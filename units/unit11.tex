% יחידה 11 - לכסון
%====================================
\section{יחידה 11: לכסון}

\subsection{משפט קנטור}
%====================================

\begin{thmbox}
\textbf{משפט קנטור (כללי)}

לכל קבוצה $A$: $|A| < |\mathcal{P}(A)|$

במילים: קבוצת החזקה תמיד גדולה יותר מהקבוצה המקורית.
\end{thmbox}

\begin{proofbox}
\textbf{הוכחה:}

$|A| \leq |\mathcal{P}(A)|$: הפונקציה $f: A \to \mathcal{P}(A)$, $f(a) = \{a\}$ היא חח``ע.

$|A| \neq |\mathcal{P}(A)|$: נניח בשלילה שקיים זיווג $g: A \to \mathcal{P}(A)$.

נגדיר $D = \{x \in A \mid x \notin g(x)\}$.

$D \in \mathcal{P}(A)$, לכן מ-$g$ על קיים $d \in A$ כך ש-$g(d) = D$.

האם $d \in D$?
\begin{itemize}
    \item אם $d \in D$ אז לפי הגדרת $D$: $d \notin g(d) = D$. סתירה!
    \item אם $d \notin D$ אז לפי הגדרת $D$: $d \in g(d) = D$. סתירה!
\end{itemize}

לכן אין זיווג, ו-$|A| < |\mathcal{P}(A)|$. $\square$
\end{proofbox}

\subsection{טכניקת הלכסון (Diagonalization)}
%====================================

\begin{notebox}
\textbf{טכניקת הלכסון}

נניח בשלילה שקיים זיווג $F: \N \to A$ (כאשר $A \subseteq \N \to X$).

נבנה $h \in A$ כך ש-$h$ \textbf{שונה מכל שורה} ב``מטריצה'' של $F$:

\begin{itemize}
    \item $h(0) \neq F(0)(0)$
    \item $h(1) \neq F(1)(1)$
    \item $h(n) \neq F(n)(n)$ לכל $n$
\end{itemize}

אז $h \in A$ אבל $h \notin \text{Im}(F)$ -- סתירה להנחה ש-$F$ על.
\end{notebox}

\subsubsection{ייצוג כמטריצה}

\begin{verbatim}
        0    1    2    3    4    ...
F(0): [ a_00  a_01  a_02  a_03  a_04  ... ]
F(1): [ a_10  a_11  a_12  a_13  a_14  ... ]
F(2): [ a_20  a_21  a_22  a_23  a_24  ... ]
F(3): [ a_30  a_31  a_32  a_33  a_34  ... ]
...

h:    [ ≠a_00 ≠a_11 ≠a_22 ≠a_33 ≠a_44 ... ]
\end{verbatim}

$h$ נבנה מהאלכסון, ושונה ממנו בכל מקום!

\subsection{דוגמאות לשימוש בלכסון}
%====================================

\begin{proofbox}
\textbf{הוכחה: קבוצת הסדרות הבינאריות אינה בת-מניה}

כלומר: $|\N \to \{0,1\}| \neq \aleph_0$

נניח בשלילה שקיים זיווג $F: \N \to (\N \to \{0,1\})$.

נגדיר:
\[h = \lambda n \in \N. \; 1 - F(n)(n)\]

\textbf{$h \in \N \to \{0,1\}$}: אם $F(n)(n) = 0$ אז $h(n) = 1$, ואם $F(n)(n) = 1$ אז $h(n) = 0$.

\textbf{$h \notin \text{Im}(F)$}: נניח בשלילה שקיים $n$ כך ש-$F(n) = h$.

אז: $F(n)(n) = h(n) = 1 - F(n)(n)$ -- סתירה!

לכן $|\N \to \{0,1\}| \neq \aleph_0$.
\end{proofbox}

\begin{proofbox}
\textbf{הוכחה: קבוצת הפונקציות החח``ע מ-$\N$ ל-$\N$ אינה בת-מניה}

יהי $A = \{f \in \N \to \N \mid f \text{ חח``ע}\}$.

נניח בשלילה שקיים זיווג $F: \N \to A$. נגדיר:
\[g = \lambda n \in \N. \; \sum_{i=0}^{n} F(i)(i) + n + 1\]

\textbf{$g$ חח``ע}: לכל $n_1 \neq n_2$, נניח $n_1 > n_2$:
\[g(n_2) = \sum_{i=0}^{n_2} F(i)(i) + n_2 + 1 \leq \sum_{i=0}^{n_1} F(i)(i) + n_2 + 1 < g(n_1)\]

\textbf{סתירה}: מ-$F$ על, קיים $n$ כך ש-$F(n) = g$, אבל:
\[g(n) = \sum_{i=0}^{n} F(i)(i) + n + 1 \geq F(n)(n) + 1 > F(n)(n) = g(n)\]
סתירה!
\end{proofbox}

\subsection{מסקנות}
%====================================

\begin{thmbox}
\textbf{מסקנה: $\R$ אינה בת-מניה}

\[|\R| = |\mathcal{P}(\N)| = |\N \to \{0,1\}| > |\N| = \aleph_0\]
\end{thmbox}

\begin{thmbox}
\textbf{מסקנה: קיימות אינסוף עוצמות}

\[|\N| < |\mathcal{P}(\N)| < |\mathcal{P}(\mathcal{P}(\N))| < \ldots\]
\end{thmbox}

\subsection{שגיאות נפוצות}
%====================================

\begin{notebox}
\textbf{שגיאה 1: לשכוח להוכיח $h \in A$}

בהוכחת לכסון, לא מספיק להגדיר $h$ -- צריך להוכיח שהיא אכן איבר בקבוצה $A$!
\end{notebox}

\begin{notebox}
\textbf{שגיאה 2: בניית $h$ שלא מהאלכסון}

$h$ חייב להיות שונה מ-$F(n)$ \textbf{במקום ה-$n$} (באלכסון).

לא מספיק שהוא שונה באיזשהו מקום.
\end{notebox}


% יחידה 7 - הרכבת פונקציות ופונקציה הפוכה
%====================================
\section{יחידה 7: הרכבת פונקציות ופונקציה הפוכה}

\subsection{הרכבת פונקציות}
%====================================

\begin{defbox}
\textbf{הגדרה: הרכבת פונקציות}

תהיינה $f: A \to B$ ו-$g: B \to C$. \textbf{ההרכבה} $g \circ f$ היא פונקציה מ-$A$ ל-$C$ המוגדרת:
\[(g \circ f)(x) = g(f(x))\]

או בסימון למבדא:
\[g \circ f = \lambda x \in A. g(f(x))\]
\end{defbox}

\begin{notebox}
\textbf{סדר ההרכבה}

ב-$g \circ f$:
\begin{itemize}
    \item \textbf{קודם} מפעילים את $f$
    \item \textbf{אחר כך} מפעילים את $g$
\end{itemize}
זה הפוך מסדר הכתיבה!
\end{notebox}

\begin{exbox}
\textbf{דוגמה:}

$f(x) = x^2$, $g(x) = x + 1$ על $\R$:
\begin{itemize}
    \item $(g \circ f)(x) = g(f(x)) = g(x^2) = x^2 + 1$
    \item $(f \circ g)(x) = f(g(x)) = f(x+1) = (x+1)^2$
\end{itemize}
שימו לב: $g \circ f \neq f \circ g$
\end{exbox}

\begin{thmbox}
\textbf{משפט: תכונות הרכבת פונקציות}
\begin{enumerate}
    \item \textbf{אסוציאטיביות}: $(h \circ g) \circ f = h \circ (g \circ f)$
    \item \textbf{פונקציית הזהות}: $f \circ i_A = f = i_B \circ f$ עבור $f: A \to B$
    \item \textbf{אי-קומוטטיביות}: בדרך כלל $f \circ g \neq g \circ f$
\end{enumerate}
\end{thmbox}

\subsection{שימור תכונות בהרכבה}
%====================================

\begin{thmbox}
\textbf{משפט: הרכבה משמרת חח"ע ועל}

\begin{enumerate}
    \item אם $f$ ו-$g$ חח"ע, אז $g \circ f$ חח"ע
    \item אם $f$ ו-$g$ על, אז $g \circ f$ על
    \item אם $f$ ו-$g$ ביאקציות, אז $g \circ f$ ביאקציה
\end{enumerate}
\end{thmbox}

\begin{thmbox}
\textbf{משפט: מסקנות מהרכבה}

\begin{enumerate}
    \item אם $g \circ f$ חח"ע, אז $f$ חח"ע
    \item אם $g \circ f$ על, אז $g$ על
\end{enumerate}
\end{thmbox}

\subsection{פונקציה הפוכה}
%====================================

\begin{defbox}
\textbf{הגדרה: פונקציה הפוכה}

תהי $f: A \to B$ ביאקציה. \textbf{הפונקציה ההפוכה} $f^{-1}: B \to A$ מוגדרת:
\[f^{-1}(y) = x \iff f(x) = y\]
\end{defbox}

\begin{notebox}
\textbf{שימו לב:}
פונקציה הפוכה $f^{-1}$ קיימת \textbf{אם ורק אם} $f$ היא ביאקציה.
\begin{itemize}
    \item אם $f$ לא חח"ע: יש $y$ עם כמה מקורות -- מי יהיה $f^{-1}(y)$?
    \item אם $f$ לא על: יש $y$ בלי מקור -- מה יהיה $f^{-1}(y)$?
\end{itemize}
\end{notebox}

\begin{thmbox}
\textbf{משפט: תכונות הפונקציה ההפוכה}

אם $f: A \to B$ ביאקציה, אז:
\begin{enumerate}
    \item $f^{-1} \circ f = i_A$
    \item $f \circ f^{-1} = i_B$
    \item $(f^{-1})^{-1} = f$
    \item $f^{-1}$ גם היא ביאקציה
\end{enumerate}
\end{thmbox}

\begin{thmbox}
\textbf{משפט: היפוך הרכבה}

אם $f: A \to B$ ו-$g: B \to C$ ביאקציות, אז $g \circ f$ ביאקציה ו:
\[(g \circ f)^{-1} = f^{-1} \circ g^{-1}\]
\end{thmbox}

\begin{exbox}
\textbf{דוגמה:}

$f(x) = 2x + 3$ על $\R$:
\begin{itemize}
    \item $f$ ביאקציה
    \item $f^{-1}(y) = \frac{y - 3}{2}$
    \item בדיקה: $(f^{-1} \circ f)(x) = f^{-1}(2x + 3) = \frac{(2x+3) - 3}{2} = x$ \checkmark
\end{itemize}
\end{exbox}

\subsection{צמצום והרחבה}
%====================================

\begin{defbox}
\textbf{הגדרה: צמצום פונקציה}

תהי $f: A \to B$ ו-$X \subseteq A$. \textbf{הצמצום של $f$ ל-$X$} הוא הפונקציה:
\[f|_X = \lambda x \in X. f(x)\]
זו אותה פונקציה, רק עם תחום מצומצם.
\end{defbox}

\begin{exbox}
\textbf{דוגמה:}

$f(x) = x^2$ על $\R$:
\begin{itemize}
    \item $f|_{\R^+}$ היא $\lambda x \in \R^+. x^2$
    \item $f$ לא חח"ע, אבל $f|_{\R^+}$ כן חח"ע!
\end{itemize}
\end{exbox}

\subsection{שגיאות נפוצות}
%====================================

\begin{notebox}
\textbf{שגיאה 1: סדר ההרכבה}

ב-$g \circ f$ קודם מפעילים את $f$!

$(g \circ f)(x) = g(f(x))$ -- קודם $f$, אחר כך $g$
\end{notebox}

\begin{notebox}
\textbf{שגיאה 2: פונקציה הפוכה לא תמיד קיימת}

$f^{-1}$ קיימת \textbf{רק אם} $f$ ביאקציה.
\end{notebox}

\begin{notebox}
\textbf{שגיאה 3: בלבול בין $f^{-1}$ לבין $f^{-1}[Y]$}
\begin{itemize}
    \item $f^{-1}$ -- הפונקציה ההפוכה (קיימת רק לביאקציות)
    \item $f^{-1}[Y]$ -- מקור הקבוצה $Y$ (קיים תמיד)
\end{itemize}
\end{notebox}


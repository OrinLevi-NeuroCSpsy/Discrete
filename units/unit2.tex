% יחידה 2 - קבוצות
%====================================
\section{יחידה 2: קבוצות}

\subsection{מושגי יסוד}
%====================================

\subsubsection{מהי קבוצה?}

\begin{defbox}
\textbf{הגדרה: קבוצה}

\textbf{קבוצה} (Set) היא אוסף של עצמים, המהווה עצם בעצמו. לעצמים שמרכיבים קבוצה קוראים \textbf{איברי} הקבוצה, ועל כל אחד מהם אומרים שהוא \textbf{שייך} לקבוצה.
\end{defbox}

\textbf{סימון:} אם $t$ הוא איבר של קבוצה $A$, נכתוב $t \in A$. אם $t$ אינו איבר של $A$, נכתוב $t \notin A$.

\begin{notebox}
\textbf{נקודות חשובות:}
\begin{itemize}
    \item \textbf{אין הגבלה} על מה שיכול לשמש כאיבר בקבוצה -- כל עצם יכול להיות איבר.
    \item קבוצות אינן חייבות להיות \textbf{הומוגניות} -- ניתן לערבב סוגי עצמים שונים.
    \item קבוצות יכולות להיות \textbf{סופיות} או \textbf{אינסופיות}.
    \item קבוצות עצמן נחשבות לעצמים, ולכן \textbf{קבוצה יכולה להיות איבר של קבוצה אחרת}.
\end{itemize}
\end{notebox}

\subsubsection{עקרון האקסטנציונליות}

\begin{thmbox}
\textbf{עקרון האקסטנציונליות}

שתי קבוצות הן \textbf{שוות} אם ורק אם יש להן \textbf{בדיוק אותם איברים}.

בסימון פורמלי:
\[A = B \Leftrightarrow \forall x.(x \in A \leftrightarrow x \in B)\]
\end{thmbox}

\textbf{משמעות:} קבוצה נקבעת אך ורק על-ידי איבריה -- לא על-ידי סדר הצגתם, לא על-ידי האופן שבו תוארה, ולא על-ידי כל מאפיין אחר.

\begin{exbox}
\textbf{דוגמאות:}
\begin{itemize}
    \item $\{1, 2, 3\} = \{3, 1, 2\} = \{1, 1, 2, 3\}$ -- סדר ההצגה וכפילויות לא משנים.
    \item קבוצת המספרים הזוגיים בין 1 ל-5 שווה לקבוצה $\{2, 4\}$.
\end{itemize}
\end{exbox}

%====================================
\subsection{הכלה (תת-קבוצה)}
%====================================

\subsubsection{הגדרה}

\begin{defbox}
\textbf{הגדרה: הכלה}

נאמר ש-$A$ היא \textbf{תת-קבוצה} של $B$ (או ש-$A$ \textbf{חלקית} ל-$B$, או ש-$B$ \textbf{מכילה} את $A$) אם כל איבר של $A$ הוא גם איבר של $B$.

\textbf{סימון:} $A \subseteq B$

בסימון פורמלי:
\[A \subseteq B \Leftrightarrow \forall x.(x \in A \rightarrow x \in B)\]
\end{defbox}

\begin{defbox}
\textbf{הגדרה: הכלה ממש}

נאמר ש-$A$ היא \textbf{תת-קבוצה ממש} של $B$ (או ש-$A$ \textbf{חלקית ממש} ל-$B$) אם $A \subseteq B$ וגם $A \neq B$.

\textbf{סימון:} $A \subsetneq B$ או $A \subset B$
\end{defbox}

\subsubsection{משפטים חשובים}

\begin{thmbox}
\textbf{משפט: קשר בין שוויון להכלה}
\[A = B \Leftrightarrow (A \subseteq B \land B \subseteq A)\]

זוהי הדרך הנפוצה ביותר להוכיח שוויון בין שתי קבוצות: מראים \textbf{הכלה דו-כיוונית}.
\end{thmbox}

\begin{thmbox}
\textbf{משפט: טרנזיטיביות ההכלה}

לכל שלוש קבוצות $A, B, C$:
\[A \subseteq B \land B \subseteq C \Rightarrow A \subseteq C\]
\end{thmbox}

\textbf{הוכחה:} יהי $x \in A$. כיוון ש-$A \subseteq B$ נובע ש-$x \in B$. מהעובדה ש-$B \subseteq C$ נובע ש-$x \in C$. $\blacksquare$

\begin{warnbox}
\textbf{אזהרה: הבדל בין $\in$ ל-$\subseteq$}
\begin{itemize}
    \item \textbf{שייכות} ($\in$): יחס בין \textbf{איבר} לבין \textbf{קבוצה}.
    \item \textbf{הכלה} ($\subseteq$): יחס בין \textbf{קבוצה} לבין \textbf{קבוצה}.
\end{itemize}

לדוגמה: קבוצת המספרים הזוגיים \textbf{חלקית} לקבוצת הטבעיים, אך היא \textbf{לא שייכת} לה (כי היא עצמה אינה מספר טבעי).
\end{warnbox}

%====================================
\subsection{הקבוצה הריקה}
%====================================

\begin{defbox}
\textbf{הגדרה: הקבוצה הריקה}

\textbf{הקבוצה הריקה} (נסמנת $\emptyset$ או $\{\}$) היא הקבוצה שאין בה איברים כלל.

בסימון פורמלי:
\[\forall x. x \notin \emptyset\]
\end{defbox}

\begin{thmbox}
\textbf{משפט: הקבוצה הריקה חלקית לכל קבוצה}

לכל קבוצה $A$ מתקיים $\emptyset \subseteq A$.
\end{thmbox}

\textbf{הוכחה:} צריך להראות ש-$\forall x.(x \in \emptyset \rightarrow x \in A)$. מכיוון שאין איברים ב-$\emptyset$, האגף השמאלי של הגרירה תמיד שקר, ולכן הגרירה כולה תמיד אמת (``מהשקר נובע הכל''). $\blacksquare$

%====================================
\subsection{אקסיומות להגדרת קבוצות}
%====================================

\subsubsection{עקרון הקומפרהנסיה המוגבל}

\begin{thmbox}
\textbf{עקרון הקומפרהנסיה המוגבל}

אם $A$ היא קבוצה ו-$P$ הוא תנאי (פרדיקט), אז הביטוי
\[\{x \in A \mid P(x)\}\]
מגדיר קבוצה. זוהי קבוצת כל האיברים מ-$A$ שמקיימים את התנאי $P$.
\end{thmbox}

\begin{exbox}
\textbf{דוגמה:}

קבוצת המספרים הזוגיים:
\[\mathbb{N}_{\text{even}} = \{n \in \mathbb{N} \mid \exists k \in \mathbb{N}. n = 2k\}\]
\end{exbox}

\subsubsection{אקסיומת ההחלפה}

\begin{thmbox}
\textbf{אקסיומת ההחלפה}

אם $F$ היא פונקציה ו-$A$ היא קבוצה, אז הביטוי
\[\{F(x) \mid x \in A\}\]
מגדיר קבוצה. זוהי קבוצת כל התוצאות של הפעלת $F$ על איברי $A$.
\end{thmbox}

\begin{exbox}
\textbf{דוגמה:}

קבוצת המספרים הזוגיים (דרך נוספת):
\[\mathbb{N}_{\text{even}} = \{2n \mid n \in \mathbb{N}\}\]
\end{exbox}

\begin{notebox}
\textbf{הערה:} שני הסימונים שקולים:
\[\{F(x) \mid x \in A\} = \{y \in B \mid \exists x \in A. F(x) = y\}\]
כאשר $B$ היא קבוצה המכילה את כל הערכים האפשריים של $F$.
\end{notebox}

%====================================
\subsection{פעולות על קבוצות}
%====================================

\subsubsection{פעולות בסיסיות}

\begin{defbox}
\textbf{הגדרות: פעולות על קבוצות}

יהיו $A, B$ קבוצות.

\textbf{איחוד (Union):}
\[A \cup B = \{x \mid x \in A \lor x \in B\}\]

\textbf{חיתוך (Intersection):}
\[A \cap B = \{x \mid x \in A \land x \in B\}\]

\textbf{הפרש (Difference):}
\[A \setminus B = \{x \mid x \in A \land x \notin B\}\]

\textbf{משלים (Complement):} ביחס לקבוצת ``עולם'' $E$:
\[\overline{A} = E \setminus A = \{x \in E \mid x \notin A\}\]

\textbf{הפרש סימטרי (Symmetric Difference):}
\[A \triangle B = (A \setminus B) \cup (B \setminus A)\]
\end{defbox}

\subsubsection{תכונות הפעולות}

\begin{thmbox}
\textbf{תכונות יסודיות של פעולות על קבוצות}

\textbf{קומוטטיביות:}
\[A \cap B = B \cap A \qquad A \cup B = B \cup A\]

\textbf{אסוציאטיביות:}
\[(A \cap B) \cap C = A \cap (B \cap C) \qquad (A \cup B) \cup C = A \cup (B \cup C)\]

\textbf{דיסטריביוטיביות:}
\[A \cap (B \cup C) = (A \cap B) \cup (A \cap C)\]
\[A \cup (B \cap C) = (A \cup B) \cap (A \cup C)\]

\textbf{חוקי דה-מורגן:}
\[\overline{A \cup B} = \overline{A} \cap \overline{B} \qquad \overline{A \cap B} = \overline{A} \cup \overline{B}\]

\textbf{משלים-הפרש:}
\[A \setminus B = A \cap \overline{B}\]
\end{thmbox}

\begin{exbox}
\textbf{דוגמה: הוכחת זהות בין קבוצות}

נוכיח: $A \setminus (B \cap C) = (A \setminus B) \cup (A \setminus C)$

\textbf{שיטה 1: שימוש בזהויות}
\begin{align*}
A \setminus (B \cap C) &= A \cap \overline{B \cap C} && \text{(משלים-הפרש)} \\
&= A \cap (\overline{B} \cup \overline{C}) && \text{(דה-מורגן)} \\
&= (A \cap \overline{B}) \cup (A \cap \overline{C}) && \text{(דיסטריביוטיביות)} \\
&= (A \setminus B) \cup (A \setminus C) && \text{(משלים-הפרש)}
\end{align*}

\textbf{שיטה 2: הכלה דו-כיוונית}

($\subseteq$) יהי $x \in A \setminus (B \cap C)$. אז $x \in A$ ו-$x \notin B \cap C$, כלומר $x \notin B$ או $x \notin C$.
\begin{itemize}
    \item אם $x \notin B$ אז $x \in A \setminus B$.
    \item אם $x \notin C$ אז $x \in A \setminus C$.
\end{itemize}
בכל מקרה $x \in (A \setminus B) \cup (A \setminus C)$.

($\supseteq$) יהי $x \in (A \setminus B) \cup (A \setminus C)$.
\begin{itemize}
    \item אם $x \in A \setminus B$ אז $x \in A$ ו-$x \notin B$, ולכן $x \notin B \cap C$.
    \item אם $x \in A \setminus C$ אז $x \in A$ ו-$x \notin C$, ולכן $x \notin B \cap C$.
\end{itemize}
בכל מקרה $x \in A \setminus (B \cap C)$. $\blacksquare$
\end{exbox}

%====================================
\subsection{קבוצת החזקה}
%====================================

\begin{defbox}
\textbf{הגדרה: קבוצת החזקה}

בהינתן קבוצה $A$, \textbf{קבוצת החזקה} שלה, $\mathcal{P}(A)$, מוגדרת כקבוצת כל תתי-הקבוצות של $A$:
\[\mathcal{P}(A) = \{B \mid B \subseteq A\}\]
\end{defbox}

\begin{notebox}
\textbf{תכונה חשובה:}

לכל קבוצה $B$ מתקיים:
\[B \in \mathcal{P}(A) \Leftrightarrow B \subseteq A\]
\end{notebox}

\begin{exbox}
\textbf{דוגמאות:}
\begin{itemize}
    \item $\mathcal{P}(\emptyset) = \{\emptyset\}$
    \item $\mathcal{P}(\{1\}) = \{\emptyset, \{1\}\}$
    \item $\mathcal{P}(\{1, 2\}) = \{\emptyset, \{1\}, \{2\}, \{1, 2\}\}$
    \item $\mathcal{P}(\mathcal{P}(\{1\})) = \{\emptyset, \{\emptyset\}, \{\{1\}\}, \{\emptyset, \{1\}\}\}$
\end{itemize}
\end{exbox}

\begin{thmbox}
\textbf{משפט:}
\[A \subseteq B \Leftrightarrow \mathcal{P}(A) \subseteq \mathcal{P}(B)\]
\end{thmbox}

\textbf{הוכחה:}

($\Rightarrow$) נניח $A \subseteq B$. תהא $X \in \mathcal{P}(A)$. אז $X \subseteq A$. מטרנזיטיביות ההכלה, $X \subseteq B$, ולכן $X \in \mathcal{P}(B)$.

($\Leftarrow$) נניח $\mathcal{P}(A) \subseteq \mathcal{P}(B)$. יהי $x \in A$. אז $\{x\} \subseteq A$, ולכן $\{x\} \in \mathcal{P}(A)$. מההנחה, $\{x\} \in \mathcal{P}(B)$, ולכן $\{x\} \subseteq B$, ומכאן $x \in B$. $\blacksquare$

%====================================
\subsection{איחוד וחיתוך מוכללים}
%====================================

\subsubsection{הגדרה באמצעות קבוצה של קבוצות}

\begin{defbox}
\textbf{הגדרה: איחוד וחיתוך מוכללים}

תהי $\mathcal{F}$ קבוצה של קבוצות.

\textbf{איחוד מוכלל:}
\[\bigcup \mathcal{F} = \{x \mid \exists A \in \mathcal{F}. x \in A\}\]

\textbf{חיתוך מוכלל} (מוגדר רק אם $\mathcal{F} \neq \emptyset$):
\[\bigcap \mathcal{F} = \{x \mid \forall A \in \mathcal{F}. x \in A\}\]
\end{defbox}

\subsubsection{הגדרה באמצעות קבוצת אינדקסים}

\begin{defbox}
\textbf{הגדרה: סימון אינדקסים}

תהי $I$ קבוצת אינדקסים, ולכל $i \in I$ תהי $A_i$ קבוצה.

\textbf{איחוד:}
\[\bigcup_{i \in I} A_i = \{x \mid \exists i \in I. x \in A_i\}\]

\textbf{חיתוך} (כאשר $I \neq \emptyset$):
\[\bigcap_{i \in I} A_i = \{x \mid \forall i \in I. x \in A_i\}\]
\end{defbox}

\begin{exbox}
\textbf{דוגמאות:}

עבור $\mathcal{X} = \{\emptyset, \{\emptyset\}, \{\{\emptyset\}\}\}$:
\[\bigcup \mathcal{X} = \{\emptyset, \{\emptyset\}\} \qquad \bigcap \mathcal{X} = \emptyset\]

עבור קטעים:
\[\bigcup_{x \in \mathbb{R}^+} (0, x) = \mathbb{R}^+ = (0, \infty) \qquad \bigcap_{x \in \mathbb{R}^+} (0, x) = \emptyset\]
\end{exbox}

\begin{exbox}
\textbf{דוגמה מפורטת:}

נוכיח: $\displaystyle\bigcup_{k \geq 2} \left[\frac{1}{k}, 1 - \frac{1}{k}\right] = (0, 1)$

($\subseteq$) יהי $x$ באיחוד. אז קיים $k \geq 2$ כך ש-$\frac{1}{k} \leq x \leq 1 - \frac{1}{k}$. מכיוון ש-$k \geq 2$, מתקיים $0 < \frac{1}{k} \leq x \leq 1 - \frac{1}{k} < 1$, ולכן $x \in (0, 1)$.

($\supseteq$) יהי $x \in (0, 1)$. נבחר $k = \max\left\{\left\lceil\frac{1}{x}\right\rceil, \left\lceil\frac{1}{1-x}\right\rceil, 2\right\}$. אז $k \geq 2$, וגם $k \geq \frac{1}{x}$ (כלומר $x \geq \frac{1}{k}$) וגם $k \geq \frac{1}{1-x}$ (כלומר $x \leq 1 - \frac{1}{k}$). $\blacksquare$
\end{exbox}

%====================================
\subsection{מלכודות נפוצות}
%====================================

\begin{warnbox}
\textbf{מלכודות נפוצות:}

\textbf{1. בלבול בין $\in$ ל-$\subseteq$:}
\begin{itemize}
    \item $1 \in \{1, 2\}$ \checkmark
    \item $\{1\} \subseteq \{1, 2\}$ \checkmark
    \item $\{1\} \in \{1, 2\}$ \ding{55} (אלא אם $\{1\}$ הוא עצמו איבר)
\end{itemize}

\textbf{2. בלבול בין $\emptyset$ ל-$\{\emptyset\}$:}
\begin{itemize}
    \item $\emptyset$ היא הקבוצה הריקה (0 איברים)
    \item $\{\emptyset\}$ היא קבוצה עם איבר אחד (הקבוצה הריקה)
\end{itemize}

\textbf{3. $\in$ אינה טרנזיטיבית:}
\begin{itemize}
    \item אם $A \in B$ ו-$B \in C$, לא בהכרח $A \in C$.
    \item דוגמה נגדית: $A = \{2\}$, $B = \{\{2\}, 2\}$, $C = \{\{\{2\}, 2\}\}$.
\end{itemize}
\end{warnbox}

%====================================
\subsection{תרגילים לדוגמה}
%====================================

\begin{exbox}
\textbf{תרגיל 1:} אילו מהטענות הבאות נכונות?

\begin{enumerate}
    \item $\emptyset \in \emptyset$ -- \textbf{לא נכון} (אין איברים ב-$\emptyset$)
    \item $\emptyset \subseteq \emptyset$ -- \textbf{נכון} (הריקה חלקית לכל קבוצה)
    \item $\emptyset \in \{\emptyset\}$ -- \textbf{נכון} (הריקה היא איבר)
    \item $\emptyset \subseteq \{\emptyset\}$ -- \textbf{נכון} (הריקה חלקית לכל קבוצה)
    \item $\{2, 3\} \subseteq \{1, 2, \{2, 3\}, 4\}$ -- \textbf{לא נכון} ($3$ לא איבר בקבוצה הימנית)
    \item $\{2, 3\} \in \{1, 2, \{2, 3\}, 4\}$ -- \textbf{נכון} ($\{2,3\}$ הוא איבר)
\end{enumerate}
\end{exbox}

\begin{exbox}
\textbf{תרגיל 2:} מצאו תנאי הכרחי ומספיק לכך ש-$\mathcal{P}(A) \cup \mathcal{P}(B) = \mathcal{P}(A \cup B)$.

\textbf{תשובה:} $A \subseteq B$ או $B \subseteq A$.

\textbf{הוכחת המספיקות:} נניח $A \subseteq B$. אז $A \cup B = B$, ו-$\mathcal{P}(A) \subseteq \mathcal{P}(B)$. לכן $\mathcal{P}(A) \cup \mathcal{P}(B) = \mathcal{P}(B) = \mathcal{P}(A \cup B)$.

\textbf{הוכחת ההכרחיות:} נניח ששתי הקבוצות לא מוכלות זו בזו. אז קיימים $a \in A \setminus B$ ו-$b \in B \setminus A$. הקבוצה $\{a, b\}$ שייכת ל-$\mathcal{P}(A \cup B)$ אך לא ל-$\mathcal{P}(A)$ ולא ל-$\mathcal{P}(B)$, ולכן לא לאיחוד שלהן.
\end{exbox}

%====================================
\subsection{מקורות}
%====================================

\begin{itemize}
    \item \texttt{discrete1-the\_course\_book.txt} -- פרק ב.1 (מושגי יסוד), פרק ב.2 (הגדרת קבוצות וסימונן), פרק ב.3 (פעולות על קבוצות)
    \item \texttt{discrete1-recitations\_1-13.txt} -- תרגולים 2 ו-3
    \item \texttt{discrete\_mathematics1-extended\_formula\_sheet.txt}
\end{itemize}

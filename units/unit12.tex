% יחידה 12 - משפט קש"ב, פעולות עוצמות
%====================================
\section{יחידה 12: משפט קש"ב, פעולות עוצמות}

\subsection{משפט קנטור-שרדר-ברנשטיין}
%====================================

\begin{thmbox}
\textbf{משפט קנטור-שרדר-ברנשטיין (קש``ב)}

אם $|A| \leq |B|$ וגם $|B| \leq |A|$, אז $|A| = |B|$.

במילים: אם קיימת פונקציה חח``ע מ-$A$ ל-$B$ וגם מ-$B$ ל-$A$, אז קיים זיווג ביניהן.
\end{thmbox}

\begin{notebox}
\textbf{טכניקת סנדוויץ'}

נרצה להוכיח ש-$|X| = a$. נמצא קבוצות $Y, Z$ עבורן קל להוכיח ש-$|Y| = |Z| = a$, נוכיח שמתקיים $|Y| \leq |X| \leq |Z|$ ונקבל ש-$|X| = a$.
\end{notebox}

\begin{thmbox}
\textbf{טענה: קשר בין חח``ע לעל}

נניח $A \neq \emptyset, B$ קבוצות. אזי:

קיימת פונקציה חח``ע $f \in A \to B$ \textbf{אם ורק אם} קיימת פונקציה $g \in B \to A$ על.
\end{thmbox}

\subsection{דוגמאות לשימוש בקש"ב}
%====================================

\begin{exbox}
\textbf{הוכחה: $|[0,1)| = |[0,1]|$}

\textbf{צריך להוכיח:} $|[0,1)| \leq |[0,1]|$ וגם $|[0,1]| \leq |[0,1)|$.

$|[0,1)| \leq |[0,1]|$: הפונקציה $f_1 = \lambda x \in [0,1). x$ היא חח``ע מ-$[0,1)$ ל-$[0,1]$.

$|[0,1]| \leq |[0,1)|$: הפונקציה $f_2 = \lambda x \in [0,1]. \frac{x}{2}$ היא חח``ע מ-$[0,1]$ ל-$[0,1)$.

מקש``ב: $|[0,1)| = |[0,1]|$. $\square$
\end{exbox}

\begin{notebox}
\textbf{הערה:}

באופן דומה ניתן להראות שכל קטע הוא מעוצמה $\aleph$:
\[|[a,b]| = |[a,b)| = |(a,b]| = |(a,b)| = \aleph\]
\end{notebox}

\subsection{פעולות על עוצמות}
%====================================

\begin{defbox}
\textbf{הגדרה: פעולות על עוצמות}

לכל שתי קבוצות $A, B$:

\begin{enumerate}
    \item \textbf{חיבור:} $|A| + |B| = |A \cup B|$ עבור $A$ ו-$B$ זרות.

    עבור קבוצות לא זרות: $|A| + |B| = |(A \times \{0\}) \cup (B \times \{1\})|$

    \item \textbf{כפל:} $|A| \cdot |B| = |A \times B|$

    \item \textbf{חזקה:} $|A|^{|B|} = |B \to A|$
\end{enumerate}
\end{defbox}

\begin{thmbox}
\textbf{משפט: קבוצת החזקה}

עבור קבוצה $A$:
\[|\mathcal{P}(A)| = |A \to \{0,1\}| = 2^{|A|}\]

בפרט:
\[\aleph = |\mathcal{P}(\N)| = 2^{|\N|} = 2^{\aleph_0}\]
\end{thmbox}

\subsection{תכונות הפעולות}
%====================================

\begin{thmbox}
\textbf{משפט: תכונות הפעולות על עוצמות}

לכל שלוש עוצמות $a, b, c$ מתקיים:

\textbf{1. שימור סדר:}
\begin{itemize}
    \item $a \leq b \Rightarrow a + c \leq b + c$
    \item $a \leq b \Rightarrow a \cdot c \leq b \cdot c$
    \item $a \leq b \Rightarrow a^c \leq b^c$
\end{itemize}

\textbf{2. קומוטטיביות:}
\begin{itemize}
    \item $a + b = b + a$
    \item $a \cdot b = b \cdot a$
\end{itemize}

\textbf{3. אסוציאטיביות:}
\begin{itemize}
    \item $(a + b) + c = a + (b + c)$
    \item $(a \cdot b) \cdot c = a \cdot (b \cdot c)$
\end{itemize}

\textbf{4. דיסטריביוטיביות:}
\begin{itemize}
    \item $a \cdot (b + c) = a \cdot b + a \cdot c$
\end{itemize}
\end{thmbox}

\begin{thmbox}
\textbf{חוקי 0 ו-1}

\begin{itemize}
    \item $a + 0 = a$
    \item $a \cdot 0 = 0$
    \item $a \cdot 1 = a$
    \item $a^0 = 1$
    \item $1^a = 1$
    \item $0^a = \begin{cases} 1 & a = 0 \\ 0 & \text{אחרת} \end{cases}$
\end{itemize}
\end{thmbox}

\begin{thmbox}
\textbf{חוקי חזקות}

\begin{itemize}
    \item $(a \cdot b)^c = a^c \cdot b^c$
    \item $a^{b+c} = a^b \cdot a^c$
    \item $(a^b)^c = a^{b \cdot c}$
\end{itemize}
\end{thmbox}

\subsection{טענות חשובות}
%====================================

\begin{thmbox}
\textbf{טענה: קבוצת מנה}

אם $T$ יחס שקילות על $A$, אז $|A/T| \leq |A|$.

\textbf{הוכחה:} הפונקציה $g = \lambda x \in A. [x]_T$ היא על מ-$A$ ל-$A/T$, ולכן $|A/T| \leq |A|$.
\end{thmbox}

\begin{notebox}
\textbf{הערה חשובה}

אם $A \subseteq B$ אז $|A| \leq |B|$, אבל כאשר $A \not\subseteq B$ לא בהכרח $|A| < |B|$.
\end{notebox}


% יחידה 4 - יחסים
%====================================
\section{יחידה 4: יחסים}

\subsection{זוגות סדורים ומכפלה קרטזית}
%====================================

\begin{defbox}
\textbf{הגדרה: זוג סדור}

\textbf{זוג סדור} $\langle a, b \rangle$ הוא אובייקט מתמטי שבו יש משמעות לסדר.

\textbf{תכונה:} $\langle a, b \rangle = \langle c, d \rangle$ אם ורק אם $a = c$ ו-$b = d$.
\end{defbox}

\begin{notebox}
\textbf{הבדל מקבוצה:}
\begin{itemize}
    \item בקבוצה $\{a, b\} = \{b, a\}$ -- אין משמעות לסדר
    \item בזוג סדור $\langle a, b \rangle \neq \langle b, a \rangle$ (אלא אם $a = b$)
\end{itemize}
\end{notebox}

\begin{defbox}
\textbf{הגדרה: מכפלה קרטזית}

\textbf{מכפלה קרטזית} של קבוצות $A$ ו-$B$:
\[
A \times B = \{\langle a, b \rangle \mid a \in A \land b \in B\}
\]
\end{defbox}

\begin{exbox}
\textbf{דוגמה:}

אם $A = \{1, 2\}$ ו-$B = \{x, y\}$:
\[
A \times B = \{\langle 1, x \rangle, \langle 1, y \rangle, \langle 2, x \rangle, \langle 2, y \rangle\}
\]

\textbf{שימו לב:} $|A \times B| = |A| \cdot |B| = 2 \cdot 2 = 4$
\end{exbox}

%====================================
\subsection{יחסים בינאריים}
%====================================

\begin{defbox}
\textbf{הגדרה: יחס בינארי}

\textbf{יחס בינארי} מ-$A$ ל-$B$ הוא תת-קבוצה $R \subseteq A \times B$.

אם $\langle a, b \rangle \in R$, נסמן גם $aRb$ ונאמר ``$a$ ביחס $R$ ל-$b$''.

\textbf{יחס על קבוצה:} יחס מ-$A$ ל-$A$, כלומר $R \subseteq A \times A$.
\end{defbox}

\begin{exbox}
\textbf{דוגמאות ליחסים:}

\begin{enumerate}
    \item יחס ``קטן מ-'' על $\N$: $R_< = \{\langle n, m \rangle \mid n < m\}$
    \item יחס ``מחלק'' על $\N$: $R_| = \{\langle n, m \rangle \mid n | m\}$
    \item יחס הזהות: $I_A = \{\langle a, a \rangle \mid a \in A\}$
\end{enumerate}
\end{exbox}

%====================================
\subsection{תכונות של יחסים}
%====================================

\begin{defbox}
\textbf{הגדרה: תכונות יחסים}

יהי $R$ יחס על קבוצה $A$:

\begin{enumerate}
    \item \textbf{רפלקסיבי:} $\forall a \in A. \langle a, a \rangle \in R$
    \item \textbf{סימטרי:} $\forall a, b \in A. aRb \then bRa$
    \item \textbf{אנטי-סימטרי:} $\forall a, b \in A. (aRb \land bRa) \then a = b$
    \item \textbf{טרנזיטיבי:} $\forall a, b, c \in A. (aRb \land bRc) \then aRc$
\end{enumerate}
\end{defbox}

\begin{exbox}
\textbf{דוגמאות:}

\begin{center}
\begin{tabular}{|l|c|c|c|c|}
\hline
\rowcolor{tableheader}\color{white}\textbf{יחס} & \color{white}\textbf{רפלקסיבי} & \color{white}\textbf{סימטרי} & \color{white}\textbf{א-סימטרי} & \color{white}\textbf{טרנזיטיבי} \\
\hline
\rowcolor{tablerow1} $=$ על $\N$ & \cmark & \cmark & \cmark & \cmark \\
\hline
\rowcolor{tablerow2} $<$ על $\N$ & \xmark & \xmark & \cmark & \cmark \\
\hline
\rowcolor{tablerow1} $\leq$ על $\N$ & \cmark & \xmark & \cmark & \cmark \\
\hline
\rowcolor{tablerow2} $|$ (מחלק) על $\N^+$ & \cmark & \xmark & \cmark & \cmark \\
\hline
\end{tabular}
\end{center}
\end{exbox}

%====================================
\subsection{הרכבת יחסים}
%====================================

\begin{defbox}
\textbf{הגדרה: הרכבת יחסים}

יהיו $R \subseteq A \times B$ ו-$S \subseteq B \times C$ יחסים.

\textbf{ההרכבה} $S \circ R$ מוגדרת:
\[
S \circ R = \{\langle a, c \rangle \mid \exists b \in B. (aRb \land bSc)\}
\]
\end{defbox}

\begin{notebox}
\textbf{שימו לב לסדר!}

ב-$S \circ R$: קודם מפעילים את $R$, אח"כ את $S$.

זה דומה להרכבת פונקציות: $(g \circ f)(x) = g(f(x))$.
\end{notebox}

\begin{defbox}
\textbf{הגדרה: חזקות של יחס}

יהי $R$ יחס על $A$:
\begin{align*}
R^{(0)} &= I_A \\
R^{(n+1)} &= R^{(n)} \circ R
\end{align*}
\end{defbox}

%====================================
\subsection{יחס הופכי}
%====================================

\begin{defbox}
\textbf{הגדרה: יחס הופכי}

\textbf{היחס ההופכי} של $R \subseteq A \times B$:
\[
R^{-1} = \{\langle b, a \rangle \mid \langle a, b \rangle \in R\}
\]
\end{defbox}

\begin{thmbox}
\textbf{משפט: תכונות היחס ההופכי}

\begin{enumerate}
    \item $(R^{-1})^{-1} = R$
    \item $(S \circ R)^{-1} = R^{-1} \circ S^{-1}$
    \item $R$ סימטרי אמ"מ $R = R^{-1}$
\end{enumerate}
\end{thmbox}

%====================================
\subsection{סיכום: סוגי יחסים}
%====================================

\begin{center}
\begin{tabular}{|l|l|}
\hline
\rowcolor{tableheader}\color{white}\textbf{סוג יחס} & \color{white}\textbf{תנאים} \\
\hline
\rowcolor{tablerow1} \textbf{יחס שקילות} & רפלקסיבי + סימטרי + טרנזיטיבי \\
\hline
\rowcolor{tablerow2} \textbf{יחס סדר חלקי} & רפלקסיבי + אנטי-סימטרי + טרנזיטיבי \\
\hline
\rowcolor{tablerow1} \textbf{יחס סדר מלא} & יחס סדר חלקי + קשר \\
\hline
\end{tabular}
\end{center}

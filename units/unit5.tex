% יחידה 5 - פונקציות (מושגי יסוד)
%====================================
\section{יחידה 5: פונקציות (מושגי יסוד)}

\subsection{מושגי יסוד}
%====================================

\subsubsection{מהי פונקציה?}

\begin{defbox}
\textbf{הגדרה: פונקציה}

\textbf{פונקציה} $f$ מקבוצה $A$ לקבוצה $B$, מסומנת $f: A \to B$, היא יחס $f \subseteq A \times B$ המקיים:
\begin{enumerate}
    \item \textbf{מלאות}: לכל $a \in A$ קיים $b \in B$ כך ש-$\langle a, b \rangle \in f$
    \item \textbf{חד-ערכיות}: אם $\langle a, b_1 \rangle \in f$ וגם $\langle a, b_2 \rangle \in f$, אז $b_1 = b_2$
\end{enumerate}

בקיצור: לכל איבר ב-$A$ מתאימה בדיוק תמונה אחת ב-$B$.
\[\forall a \in A. \exists! b \in B. \langle a, b \rangle \in f\]
\end{defbox}

\begin{notebox}
\textbf{סימונים:}
\begin{itemize}
    \item $f(a) = b$ או $f: a \mapsto b$ -- הפונקציה $f$ מעתיקה את $a$ ל-$b$
    \item $A \to B$ -- קבוצת כל הפונקציות מ-$A$ ל-$B$
\end{itemize}
\end{notebox}

\subsubsection{תחום, טווח ותמונה}

\begin{defbox}
\textbf{הגדרות: תחום, טווח ותמונה}

עבור פונקציה $f: A \to B$:
\begin{itemize}
    \item \textbf{התחום} (Domain): $\Dom(f) = A$
    \item \textbf{הטווח} (Codomain): $B$ -- הקבוצה אליה הפונקציה מעתיקה
    \item \textbf{התמונה} (Image/Range): $\Img(f) = \{f(x) \mid x \in A\} = \{y \in B \mid \exists x \in A. f(x) = y\}$
\end{itemize}
\end{defbox}

\begin{notebox}
\textbf{הבחנה חשובה:}
\begin{itemize}
    \item \textbf{הטווח} $B$ הוא הקבוצה אליה הפונקציה \textbf{יכולה} להעתיק
    \item \textbf{התמונה} $\Img(f)$ היא קבוצת הערכים שהפונקציה \textbf{באמת} מקבלת
\end{itemize}
תמיד מתקיים: $\Img(f) \subseteq B$
\end{notebox}

\begin{exbox}
\textbf{דוגמה:}

עבור $f: \R \to \R$ המוגדרת $f(x) = x^2$:
\begin{itemize}
    \item $\Dom(f) = \R$
    \item הטווח הוא $\R$
    \item $\Img(f) = [0, \infty) = \{y \in \R \mid y \geq 0\}$
\end{itemize}
\end{exbox}

%====================================
\subsection{סימון למבדא}
%====================================

\begin{defbox}
\textbf{סימון למבדא}

\textbf{סימון למבדא} (Lambda notation) מאפשר להגדיר פונקציה בצורה קומפקטית:
\[\lambda x \in A. t(x)\]
מתאר את הפונקציה מ-$A$ שמתאימה לכל $x$ את הערך $t(x)$.
\end{defbox}

\begin{exbox}
\textbf{דוגמאות:}
\begin{itemize}
    \item $\lambda x \in \R. x^2$ -- הפונקציה שמעלה כל מספר ממשי בריבוע
    \item $\lambda n \in \N. 2n + 1$ -- הפונקציה שמתאימה לכל מספר טבעי את המספר האי-זוגי המתאים
    \item $\lambda x \in \R. \lambda y \in \R. x + y$ -- פונקציה של שני משתנים
\end{itemize}
\end{exbox}

\subsubsection{כללי תחשיב למבדא}

\begin{defbox}
\textbf{כלל $\alpha$ (החלפת משתנה)}

ניתן להחליף את שם המשתנה הקשור:
\[\lambda x \in A. t = \lambda y \in A. t[y/x]\]
בתנאי ש-$y$ לא מופיע חופשי ב-$t$.
\end{defbox}

\begin{defbox}
\textbf{כלל $\beta$ (הצבה)}

אם $s$ חופשי להצבה במקום $x$ ב-$t$:
\[(\lambda x \in A. t)(s) = t[s/x]\]
זהו כלל ``החישוב'' -- הצבת ארגומנט בפונקציה.
\end{defbox}

\begin{defbox}
\textbf{כלל $\eta$ (אקסטנציונליות)}
\[\lambda x \in \Dom(f). f(x) = f\]
\end{defbox}

\begin{exbox}
\textbf{דוגמה לשימוש בכללים:}
\[(\lambda x \in \R. x^2)(3) = 3^2 = 9\]
\[(\lambda x \in \R. x + 1) \circ (\lambda x \in \R. x^2) = \lambda x \in \R. x^2 + 1\]
\end{exbox}

%====================================
\subsection{תמונה ומקור של קבוצות}
%====================================

\begin{defbox}
\textbf{הגדרה: תמונה של קבוצה}

תהי $f: A \to B$ ו-$X \subseteq A$. \textbf{התמונה של $X$ תחת $f$} היא:
\[f[X] = \{f(x) \mid x \in X\} = \{y \in B \mid \exists x \in X. f(x) = y\}\]
\end{defbox}

\begin{defbox}
\textbf{הגדרה: מקור של קבוצה}

תהי $f: A \to B$ ו-$Y \subseteq B$. \textbf{המקור של $Y$ לפי $f$} הוא:
\[f^{-1}[Y] = \{x \in A \mid f(x) \in Y\}\]
\end{defbox}

\begin{exbox}
\textbf{דוגמה:}

עבור $f(x) = x^2$ על $\R$:
\begin{itemize}
    \item $f[(-\infty, 3)] = [0, 9]$
    \item $f[\{-2, -1, 0, 1, 2\}] = \{0, 1, 4\}$
    \item $f^{-1}[\{9\}] = \{-3, 3\}$
    \item $f^{-1}[[0, 1]] = [-1, 1]$
    \item $f^{-1}[\{-1\}] = \emptyset$ (אין מספר ממשי שריבועו שלילי)
\end{itemize}
\end{exbox}

%====================================
\subsection{הרכבת פונקציות}
%====================================

\begin{defbox}
\textbf{הגדרה: הרכבת פונקציות}

תהיינה $f: A \to B$ ו-$g: B \to C$. \textbf{ההרכבה} $g \circ f$ היא פונקציה מ-$A$ ל-$C$ המוגדרת:
\[(g \circ f)(x) = g(f(x))\]

או בסימון למבדא:
\[g \circ f = \lambda x \in A. g(f(x))\]
\end{defbox}

\begin{notebox}
\textbf{סדר ההרכבה}

ב-$g \circ f$:
\begin{itemize}
    \item \textbf{קודם} מפעילים את $f$
    \item \textbf{אחר כך} מפעילים את $g$
\end{itemize}
זה הפוך מסדר הכתיבה!
\end{notebox}

\begin{exbox}
\textbf{דוגמה:}

$f(x) = x^2$, $g(x) = x + 1$ על $\R$:
\begin{itemize}
    \item $(g \circ f)(x) = g(f(x)) = g(x^2) = x^2 + 1$
    \item $(f \circ g)(x) = f(g(x)) = f(x+1) = (x+1)^2$
\end{itemize}
שימו לב: $g \circ f \neq f \circ g$
\end{exbox}

\begin{thmbox}
\textbf{משפט: תכונות הרכבת פונקציות}
\begin{enumerate}
    \item \textbf{אסוציאטיביות}: $(h \circ g) \circ f = h \circ (g \circ f)$
    \item \textbf{פונקציית הזהות}: $f \circ i_A = f = i_B \circ f$ עבור $f: A \to B$, כאשר $i_A = \lambda x \in A. x$
    \item \textbf{אי-קומוטטיביות}: בדרך כלל $f \circ g \neq g \circ f$
\end{enumerate}
\end{thmbox}

% המשך החומר על פונקציות מופיע ביחידות 6 ו-7


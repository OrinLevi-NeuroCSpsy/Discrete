% יחידה 6 - יחסי שקילות וחלוקות
%====================================
\section{יחידה 6: יחסי שקילות וחלוקות}

\subsection{יחסי שקילות}
%====================================

\begin{defbox}
\textbf{הגדרה: יחס שקילות}

יחס $E$ על קבוצה $A$ נקרא \textbf{יחס שקילות} (equivalence relation) אם הוא מקיים שלושה תנאים:

\begin{enumerate}
    \item \textbf{רפלקסיביות:} $\forall x \in A. \; xEx$
    \item \textbf{סימטריות:} $\forall x, y \in A. \; xEy \Rightarrow yEx$
    \item \textbf{טרנזיטיביות:} $\forall x, y, z \in A. \; (xEy \land yEz) \Rightarrow xEz$
\end{enumerate}
\end{defbox}

\begin{exbox}
\textbf{דוגמאות ליחסי שקילות:}

\begin{enumerate}
    \item \textbf{שוויון} $=$ על כל קבוצה

    \item \textbf{דמיון משולשים} על קבוצת המשולשים

    \item \textbf{חפיפת משולשים} על קבוצת המשולשים

    \item \textbf{שקילות מודולו $n$:}
    \[a \equiv b \pmod{n} \iff n \mid (a - b)\]

    לדוגמה: $5 \equiv 11 \pmod{2}$ כי $2 \mid (11 - 5) = 6$

    \item \textbf{יחס ``באותו צבע''} על קבוצת כדורים צבעוניים
\end{enumerate}
\end{exbox}

%====================================
\subsection{מחלקות שקילות}
%====================================

\begin{defbox}
\textbf{הגדרה: מחלקת שקילות}

יהי $E$ יחס שקילות על $A$. עבור $x \in A$, \textbf{מחלקת השקילות} של $x$ לפי $E$ היא:

\[[x]_E = \{y \in A \mid xEy\}\]

כלומר, קבוצת כל האיברים השקולים ל-$x$.
\end{defbox}

\begin{notebox}
\textbf{סימונים נפוצים למחלקת שקילות:}
\begin{itemize}
    \item $[x]_E$ או $[x]$ (כשהיחס ברור מההקשר)
    \item $\overline{x}$
    \item $x/E$
\end{itemize}
\end{notebox}

\subsubsection{תכונות מחלקות שקילות}

\begin{thmbox}
\textbf{תכונות מחלקות שקילות}

יהי $E$ יחס שקילות על $A$. לכל $x, y \in A$:

\begin{enumerate}
    \item \textbf{כל איבר שייך למחלקה שלו:} $x \in [x]_E$ (מרפלקסיביות)

    \item \textbf{שקילות היא שייכות למחלקה:} $y \in [x]_E \iff xEy$

    \item \textbf{סימטריות:} $y \in [x]_E \iff x \in [y]_E$

    \item \textbf{מחלקות שוות או זרות:}
    \[[x]_E \cap [y]_E \neq \emptyset \iff [x]_E = [y]_E \iff xEy\]
\end{enumerate}
\end{thmbox}

\begin{exbox}
\textbf{דוגמה: מחלקות מודולו 3}

יחס השקילות מודולו 3 על $\Z$:

\begin{itemize}
    \item $[0]_{\equiv_3} = \{\ldots, -6, -3, 0, 3, 6, 9, \ldots\} = 3\Z$
    \item $[1]_{\equiv_3} = \{\ldots, -5, -2, 1, 4, 7, 10, \ldots\} = 3\Z + 1$
    \item $[2]_{\equiv_3} = \{\ldots, -4, -1, 2, 5, 8, 11, \ldots\} = 3\Z + 2$
\end{itemize}

שלוש מחלקות זרות שמכסות את כל $\Z$.
\end{exbox}

%====================================
\subsection{קבוצת מנה}
%====================================

\begin{defbox}
\textbf{הגדרה: קבוצת מנה}

יהי $E$ יחס שקילות על $A$. \textbf{קבוצת המנה} (quotient set) של $A$ לפי $E$ היא:

\[A/E = \{[x]_E \mid x \in A\}\]

כלומר, קבוצת כל מחלקות השקילות.
\end{defbox}

\begin{exbox}
\textbf{דוגמאות לקבוצות מנה:}

\begin{enumerate}
    \item $\Z/\equiv_2 = \{[0], [1]\}$ -- שתי מחלקות (זוגיים ואי-זוגיים)

    \item $\Z/\equiv_3 = \{[0], [1], [2]\}$ -- שלוש מחלקות

    \item $\R/\equiv$ כאשר $x \equiv y \iff |x| = |y|$:

    קבוצת המנה היא למעשה $\R_{\geq 0}$
\end{enumerate}
\end{exbox}

%====================================
\subsection{פירוק (חלוקה)}
%====================================

\begin{defbox}
\textbf{הגדרה: פירוק של קבוצה}

\textbf{פירוק} (partition) של קבוצה $A$ הוא קבוצה $F$ של תתי-קבוצות של $A$ המקיימת:

\begin{enumerate}
    \item \textbf{לא ריקות:} $\forall X \in F. \; X \neq \emptyset$
    \item \textbf{כיסוי:} $\bigcup F = A$
    \item \textbf{זרות הדדית:} $\forall X, Y \in F. \; X \neq Y \Rightarrow X \cap Y = \emptyset$
\end{enumerate}
\end{defbox}

\begin{exbox}
\textbf{דוגמאות לפירוקים:}

\begin{enumerate}
    \item $\{\{1, 3, 5\}, \{2, 4, 7, 8\}, \{6\}\}$ הוא פירוק של $\{1, 2, 3, 4, 5, 6, 7, 8\}$

    \item $\{\Z^-, \{0\}, \Z^+\}$ הוא פירוק של $\Z$
\end{enumerate}
\end{exbox}

\subsubsection{הקשר בין יחסי שקילות לפירוקים}

\begin{thmbox}
\textbf{משפט: קבוצת מנה היא פירוק}

יהי $E$ יחס שקילות על $A$. אז $A/E$ הוא פירוק של $A$.
\end{thmbox}

\begin{proofbox}
\textbf{הוכחה (רעיון):}

\begin{enumerate}
    \item \textbf{לא ריקות:} לכל $x \in A$, $x \in [x]_E$ (מרפלקסיביות), לכן $[x]_E \neq \emptyset$

    \item \textbf{כיסוי:} לכל $x \in A$, $x \in [x]_E \in A/E$, לכן $\bigcup A/E = A$

    \item \textbf{זרות:} מחלקות שונות זרות (הוכחנו קודם)
\end{enumerate}
\end{proofbox}

\begin{thmbox}
\textbf{משפט: ההתאמה בין יחסי שקילות לפירוקים}

קיימת \textbf{התאמה חח``ע ועל} בין יחסי השקילות על $A$ לבין הפירוקים של $A$:

\begin{itemize}
    \item \textbf{מיחס לפירוק:} $E \mapsto A/E$
    \item \textbf{מפירוק ליחס:} $F \mapsto E_F$ כאשר $xE_Fy \iff$ קיים $X \in F$ כך ש-$x, y \in X$
\end{itemize}
\end{thmbox}

%====================================
\subsection{אי-תלות בנציג}
%====================================

\begin{notebox}
\textbf{בעיית הנציג}

כשמגדירים פעולה או פונקציה על קבוצת המנה, לעתים ההגדרה תלויה בבחירת \textbf{נציג} מכל מחלקה.

צריך להוכיח שהתוצאה \textbf{לא תלויה בבחירת הנציג}.
\end{notebox}

\begin{exbox}
\textbf{דוגמה: חיבור על $\Z_n$}

נגדיר חיבור על $\Z/\equiv_n$:
\[[a] + [b] = [a + b]\]

\textbf{צריך להוכיח:} אם $[a] = [a']$ ו-$[b] = [b']$, אז $[a + b] = [a' + b']$.

\textbf{הוכחה:}
\begin{itemize}
    \item $[a] = [a']$ $\Rightarrow$ $n \mid (a - a')$
    \item $[b] = [b']$ $\Rightarrow$ $n \mid (b - b')$
    \item לכן $n \mid ((a + b) - (a' + b')) = (a - a') + (b - b')$
    \item לכן $[a + b] = [a' + b']$ \checkmark
\end{itemize}
\end{exbox}

\begin{notebox}
\textbf{מבנה הוכחת אי-תלות בנציג:}

\begin{enumerate}
    \item נניח $[a] = [a']$ ו-$[b] = [b']$
    \item נתרגם להגדרת היחס: $aEa'$ ו-$bEb'$
    \item נראה ש-$(a \star b) E (a' \star b')$
    \item נסיק $[a \star b] = [a' \star b']$
\end{enumerate}
\end{notebox}

%====================================
\subsection{הפונקציה הקנונית}
%====================================

\begin{defbox}
\textbf{הגדרה: הפונקציה הקנונית}

יהי $E$ יחס שקילות על $A$. \textbf{הפונקציה הקנונית} (או פונקציית המנה) היא:
\[\pi: A \to A/E, \quad \pi(x) = [x]_E\]
\end{defbox}

\begin{thmbox}
\textbf{תכונות הפונקציה הקנונית:}

\begin{enumerate}
    \item $\pi$ היא \textbf{על} (surjective)
    \item $\pi(x) = \pi(y) \iff xEy$
    \item $\text{Im}(\pi) = A/E$
\end{enumerate}
\end{thmbox}

%====================================
\subsection{טבלת סיכום}
%====================================

\begin{center}
\begin{tabular}{|l|l|l|}
\hline
\rowcolor{tableheader}\color{white}\textbf{מושג} & \color{white}\textbf{הגדרה} & \color{white}\textbf{דוגמה} \\
\hline
\rowcolor{tablerow1} \textbf{יחס שקילות} & רפלקסיבי + סימטרי + טרנזיטיבי & $\equiv_n$, $=$ \\
\hline
\rowcolor{tablerow2} \textbf{מחלקת שקילות} & $[x]_E = \{y \mid xEy\}$ & $[0]_{\equiv_3} = 3\Z$ \\
\hline
\rowcolor{tablerow1} \textbf{קבוצת מנה} & $A/E = \{[x]_E \mid x \in A\}$ & $\Z/\equiv_3$ \\
\hline
\rowcolor{tablerow2} \textbf{פירוק} & קבוצות זרות שמכסות את $A$ & $\{[0], [1], [2]\}$ \\
\hline
\end{tabular}
\end{center}

%====================================
\subsection{שגיאות נפוצות}
%====================================

\begin{notebox}
\textbf{שגיאה 1: בלבול בין מחלקה לאיבר}

\begin{itemize}
    \item $[a]$ היא \textbf{קבוצה} (מחלקת השקילות)
    \item $a$ הוא \textbf{איבר} (נציג של המחלקה)
\end{itemize}
\end{notebox}

\begin{notebox}
\textbf{שגיאה 2: הנחה שכל נציג שקול}

כאשר מוכיחים אי-תלות בנציג, צריך להוכיח עבור \textbf{כל} שני נציגים של אותה מחלקה, לא רק עבור נציגים ספציפיים.
\end{notebox}

\begin{notebox}
\textbf{שגיאה 3: שכחת לבדוק את שלושת התנאים}

יחס שקילות חייב לקיים את \textbf{שלושת} התנאים: רפלקסיביות, סימטריה, טרנזיטיביות.
\end{notebox}

%====================================
\subsection{תרגילים לתרגול}
%====================================

\begin{exbox}
\textbf{תרגיל 1:}
הוכיחו שהיחס $R$ על $\Z$ המוגדר ע``י $xRy \iff x - y$ זוגי הוא יחס שקילות. מהן מחלקות השקילות?
\end{exbox}

\begin{exbox}
\textbf{תרגיל 2:}
יהי $f: A \to B$ פונקציה. הגדירו על $A$ את היחס: $xEy \iff f(x) = f(y)$.
הוכיחו ש-$E$ יחס שקילות ותארו את מחלקות השקילות.
\end{exbox}

\begin{exbox}
\textbf{תרגיל 3:}
הוכיחו שהכפל על $\Z_n$ מוגדר היטב (אי-תלות בנציג).
\end{exbox}


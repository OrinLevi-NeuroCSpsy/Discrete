% יחידה 8 - יחסי שקילות, מחלקות
%====================================
\section{יחידה 8: יחסי שקילות, מחלקות}

\subsection{יחסי שקילות}
%====================================

\begin{defbox}
\textbf{הגדרה: יחס שקילות}

יחס $E$ על קבוצה $A$ נקרא \textbf{יחס שקילות} (equivalence relation) אם הוא מקיים שלושה תנאים:

\begin{enumerate}
    \item \textbf{רפלקסיביות:} $\forall x \in A. \; xEx$
    \item \textbf{סימטריות:} $\forall x, y \in A. \; xEy \Rightarrow yEx$
    \item \textbf{טרנזיטיביות:} $\forall x, y, z \in A. \; (xEy \land yEz) \Rightarrow xEz$
\end{enumerate}
\end{defbox}

\begin{exbox}
\textbf{דוגמאות ליחסי שקילות:}

\begin{enumerate}
    \item \textbf{שוויון} $=$ על כל קבוצה

    \item \textbf{דמיון משולשים} על קבוצת המשולשים

    \item \textbf{חפיפת משולשים} על קבוצת המשולשים

    \item \textbf{שקילות מודולו $n$:}
    \[a \equiv b \pmod{n} \iff n \mid (a - b)\]

    לדוגמה: $5 \equiv 11 \pmod{2}$ כי $2 \mid (11 - 5) = 6$

    \item \textbf{יחס ``באותו צבע''} על קבוצת כדורים צבעוניים
\end{enumerate}
\end{exbox}

\subsection{מחלקות שקילות}
%====================================

\begin{defbox}
\textbf{הגדרה: מחלקת שקילות}

יהי $E$ יחס שקילות על $A$. עבור $x \in A$, \textbf{מחלקת השקילות} של $x$ לפי $E$ היא:

\[[x]_E = \{y \in A \mid xEy\}\]

כלומר, קבוצת כל האיברים השקולים ל-$x$.
\end{defbox}

\begin{notebox}
\textbf{סימונים נפוצים למחלקת שקילות:}
\begin{itemize}
    \item $[x]_E$ או $[x]$ (כשהיחס ברור מההקשר)
    \item $\overline{x}$
    \item $x/E$
\end{itemize}
\end{notebox}

\begin{thmbox}
\textbf{תכונות מחלקות שקילות}

יהי $E$ יחס שקילות על $A$. לכל $x, y \in A$:

\begin{enumerate}
    \item \textbf{כל איבר שייך למחלקה שלו:} $x \in [x]_E$ (מרפלקסיביות)

    \item \textbf{שקילות היא שייכות למחלקה:} $y \in [x]_E \iff xEy$

    \item \textbf{סימטריות:} $y \in [x]_E \iff x \in [y]_E$

    \item \textbf{מחלקות שוות או זרות:}
    \[[x]_E \cap [y]_E \neq \emptyset \iff [x]_E = [y]_E \iff xEy\]
\end{enumerate}
\end{thmbox}

\begin{exbox}
\textbf{דוגמה: מחלקות מודולו 3}

יחס השקילות מודולו 3 על $\Z$:

\begin{itemize}
    \item $[0]_{\equiv_3} = \{\ldots, -6, -3, 0, 3, 6, 9, \ldots\} = 3\Z$
    \item $[1]_{\equiv_3} = \{\ldots, -5, -2, 1, 4, 7, 10, \ldots\} = 3\Z + 1$
    \item $[2]_{\equiv_3} = \{\ldots, -4, -1, 2, 5, 8, 11, \ldots\} = 3\Z + 2$
\end{itemize}

שלוש מחלקות זרות שמכסות את כל $\Z$.
\end{exbox}

\subsection{קבוצת מנה}
%====================================

\begin{defbox}
\textbf{הגדרה: קבוצת מנה}

יהי $E$ יחס שקילות על $A$. \textbf{קבוצת המנה} (quotient set) של $A$ לפי $E$ היא:

\[A/E = \{[x]_E \mid x \in A\}\]

כלומר, קבוצת כל מחלקות השקילות.
\end{defbox}

\begin{exbox}
\textbf{דוגמאות לקבוצות מנה:}

\begin{enumerate}
    \item $\Z/\equiv_2 = \{[0], [1]\}$ -- שתי מחלקות (זוגיים ואי-זוגיים)

    \item $\Z/\equiv_3 = \{[0], [1], [2]\}$ -- שלוש מחלקות

    \item $\R/\equiv$ כאשר $x \equiv y \iff |x| = |y|$:

    קבוצת המנה היא למעשה $\R_{\geq 0}$
\end{enumerate}
\end{exbox}

\subsection{הפונקציה הקנונית}
%====================================

\begin{defbox}
\textbf{הגדרה: הפונקציה הקנונית}

יהי $E$ יחס שקילות על $A$. \textbf{הפונקציה הקנונית} (או פונקציית המנה) היא:
\[\pi: A \to A/E, \quad \pi(x) = [x]_E\]
\end{defbox}

\begin{thmbox}
\textbf{תכונות הפונקציה הקנונית:}

\begin{enumerate}
    \item $\pi$ היא \textbf{על} (surjective)
    \item $\pi(x) = \pi(y) \iff xEy$
    \item $\text{Im}(\pi) = A/E$
\end{enumerate}
\end{thmbox}

\subsection{טבלת סיכום}
%====================================

\begin{center}
\begin{tabular}{|l|l|l|}
\hline
\rowcolor{tableheader}\color{white}\textbf{מושג} & \color{white}\textbf{הגדרה} & \color{white}\textbf{דוגמה} \\
\hline
\rowcolor{tablerow1} \textbf{יחס שקילות} & רפלקסיבי + סימטרי + טרנזיטיבי & $\equiv_n$, $=$ \\
\hline
\rowcolor{tablerow2} \textbf{מחלקת שקילות} & $[x]_E = \{y \mid xEy\}$ & $[0]_{\equiv_3} = 3\Z$ \\
\hline
\rowcolor{tablerow1} \textbf{קבוצת מנה} & $A/E = \{[x]_E \mid x \in A\}$ & $\Z/\equiv_3$ \\
\hline
\end{tabular}
\end{center}

\subsection{שגיאות נפוצות}
%====================================

\begin{notebox}
\textbf{שגיאה 1: בלבול בין מחלקה לאיבר}

\begin{itemize}
    \item $[a]$ היא \textbf{קבוצה} (מחלקת השקילות)
    \item $a$ הוא \textbf{איבר} (נציג של המחלקה)
\end{itemize}
\end{notebox}

\begin{notebox}
\textbf{שגיאה 2: שכחת לבדוק את שלושת התנאים}

יחס שקילות חייב לקיים את \textbf{שלושת} התנאים: רפלקסיביות, סימטריה, טרנזיטיביות.
\end{notebox}


% יחידה 8 - עוצמות וספירה
%====================================
\section{יחידה 8: עוצמות וספירה}

\subsection{מושגי יסוד}
%====================================

\subsubsection{שוויון עוצמות}

\begin{defbox}
\textbf{הגדרה: שוויון עוצמות}

שתי קבוצות $A$ ו-$B$ הן \textbf{שוות עוצמה} (equinumerous) אם קיים ביניהן \textbf{זיווג} (פונקציה חח``ע ועל).

\textbf{סימון:} $|A| = |B|$ או $A \sim B$
\end{defbox}

\begin{thmbox}
\textbf{טענה: יחס שקילות}

היחס $\sim$ (שוויון עוצמות) מתנהג כיחס שקילות:

\begin{enumerate}
    \item \textbf{רפלקסיביות:} $|A| = |A|$ (פונקציית הזהות היא זיווג)
    \item \textbf{סימטריות:} $|A| = |B| \Rightarrow |B| = |A|$ (הפונקציה ההפוכה היא זיווג)
    \item \textbf{טרנזיטיביות:} $|A| = |B| \land |B| = |C| \Rightarrow |A| = |C|$ (הרכבת זיווגים)
\end{enumerate}
\end{thmbox}

\subsubsection{סימונים מיוחדים}

\begin{defbox}
\textbf{הגדרה: $\aleph_0$ (אלף-אפס)}

\[\aleph_0 = |\N|\]

קבוצה $A$ נקראת \textbf{בת-מניה} (countable) אם $|A| = \aleph_0$.
\end{defbox}

\begin{defbox}
\textbf{הגדרה: $\aleph$ (עוצמת הרצף)}

\[\aleph = |\R|\]

קבוצה בעוצמה $\aleph$ נקראת \textbf{בעוצמת הרצף} (continuum).
\end{defbox}

%====================================
\subsection{קבוצות בנות-מניה}
%====================================

\subsubsection{תכונות בסיסיות}

\begin{thmbox}
\textbf{טענה: קבוצות בנות-מניה}

\begin{enumerate}
    \item $|\N^k| = |\N \times \N| = |\N| = \aleph_0$ לכל $k > 0$
    \item $|\Q| = |\Z| = \aleph_0$
    \item תת-קבוצה אינסופית של קבוצה בת-מניה היא בת-מניה
    \item איחוד בן-מניה של קבוצות בנות-מניה הוא בן-מניה
\end{enumerate}
\end{thmbox}

\begin{exbox}
\textbf{דוגמאות לקבוצות בנות-מניה:}

\begin{itemize}
    \item $\N$, $\Z$, $\Q$
    \item $\N \times \N$
    \item קבוצת כל המחרוזות הסופיות מעל א``ב סופי
    \item קבוצת כל הפולינומים עם מקדמים רציונליים
\end{itemize}
\end{exbox}

\subsubsection{טכניקות להוכחת שוויון עוצמות}

\begin{notebox}
\textbf{שיטה: בניית זיווג}

כדי להוכיח $|A| = |B|$, בונים פונקציה $f: A \to B$ ומוכיחים:

\begin{enumerate}
    \item $f$ מוגדרת היטב (לכל $a \in A$ יש ערך $f(a) \in B$)
    \item $f$ חח``ע (אם $f(a_1) = f(a_2)$ אז $a_1 = a_2$)
    \item $f$ על (לכל $b \in B$ קיים $a \in A$ כך ש-$f(a) = b$)
\end{enumerate}
\end{notebox}

\begin{exbox}
\textbf{דוגמה: $|[0,1] \cup \N| = |[0,1]|$}

נבנה זיווג $f: [0,1] \cup \N \to [0,1]$.

\textbf{הרעיון:} ``נשכן'' את $\N$ בתוך תת-קבוצה בת-מניה של $[0,1]$.

ניקח $A = \{\frac{1}{n} \mid n \in \N^+\} = \{1, \frac{1}{2}, \frac{1}{3}, \ldots\}$

נגדיר:
\[f(x) = \begin{cases}
\frac{x}{2} & x \in A \\
\frac{1}{2x-3} & x \in \N \setminus \{0,1\} \\
x & \text{otherwise}
\end{cases}\]
\end{exbox}

%====================================
\subsection{עוצמת הרצף}
%====================================

\subsubsection{קבוצות בעוצמת הרצף}

\begin{thmbox}
\textbf{טענה: עוצמת קטעים}

לכל $a < b$ ממשיים:
\[|[a,b]| = |[a,b)| = |(a,b]| = |(a,b)| = \aleph\]
\end{thmbox}

\begin{thmbox}
\textbf{טענה: קבוצות בעוצמת הרצף}

\begin{itemize}
    \item $|\R| = \aleph$
    \item $|[0,1]| = \aleph$
    \item $|\R^n| = \aleph$ לכל $n \geq 1$
    \item $|\mathcal{P}(\N)| = \aleph$
    \item $|\N \to \{0,1\}| = \aleph$
\end{itemize}
\end{thmbox}

\subsubsection{הקשר בין $\aleph_0$ ל-$\aleph$}

\begin{thmbox}
\textbf{משפט קנטור}

\[\aleph_0 < \aleph\]

כלומר: $\R$ \textbf{אינה בת-מניה}.
\end{thmbox}

%====================================
\subsection{הוכחת לכסון (Diagonalization)}
%====================================

\subsubsection{העיקרון}

\begin{thmbox}
\textbf{משפט קנטור (כללי)}

לכל קבוצה $A$: $|A| < |\mathcal{P}(A)|$

במילים: קבוצת החזקה תמיד גדולה יותר מהקבוצה המקורית.
\end{thmbox}

\begin{notebox}
\textbf{טכניקת הלכסון}

נניח בשלילה שקיים זיווג $F: \N \to A$ (כאשר $A \subseteq \N \to X$).

נבנה $h \in A$ כך ש-$h$ \textbf{שונה מכל שורה} ב``מטריצה'' של $F$:

\begin{itemize}
    \item $h(0) \neq F(0)(0)$
    \item $h(1) \neq F(1)(1)$
    \item $h(n) \neq F(n)(n)$ לכל $n$
\end{itemize}

אז $h \in A$ אבל $h \notin \text{Im}(F)$ -- סתירה להנחה ש-$F$ על.
\end{notebox}

\subsubsection{דוגמאות}

\begin{proofbox}
\textbf{הוכחה: $|\N \to \{0,1\}| \neq \aleph_0$}

נניח בשלילה שקיים זיווג $F: \N \to (\N \to \{0,1\})$.

נגדיר:
\[h = \lambda n \in \N. \; 1 - F(n)(n)\]

\textbf{$h \in \N \to \{0,1\}$}: אם $F(n)(n) = 0$ אז $h(n) = 1$, ואם $F(n)(n) = 1$ אז $h(n) = 0$.

\textbf{$h \notin \text{Im}(F)$}: מ-$F$ על, קיים $n$ כך ש-$F(n) = h$.

אז: $F(n)(n) = h(n) = 1 - F(n)(n)$ -- סתירה!

לכן $|\N \to \{0,1\}| \neq \aleph_0$.
\end{proofbox}

\begin{proofbox}
\textbf{הוכחה: קבוצת הפונקציות החח``ע מ-$\N$ ל-$\N$ אינה בת-מניה}

יהי $A = \{f \in \N \to \N \mid f \text{ חח``ע}\}$.

נניח בשלילה שקיים זיווג $F: \N \to A$. נגדיר:
\[g = \lambda n \in \N. \; \sum_{i=0}^{n} F(i)(i) + n + 1\]

\textbf{$g$ חח``ע}: לכל $n_1 \neq n_2$, נניח $n_1 > n_2$:
\[g(n_2) = \sum_{i=0}^{n_2} F(i)(i) + n_2 + 1 \leq \sum_{i=0}^{n_1} F(i)(i) + n_2 + 1 < g(n_1)\]

\textbf{סתירה}: מ-$F$ על, קיים $n$ כך ש-$F(n) = g$, אבל:
\[g(n) = \sum_{i=0}^{n} F(i)(i) + n + 1 \geq F(n)(n) + 1 > F(n)(n) = g(n)\]
סתירה!
\end{proofbox}

%====================================
\subsection{משפט קנטור-שרדר-ברנשטיין}
%====================================

\subsubsection{סדר על עוצמות}

\begin{defbox}
\textbf{הגדרה: $|A| \leq |B|$}

$|A| \leq |B|$ אם קיימת פונקציה \textbf{חח``ע} $f: A \to B$.

שקול: קיימת פונקציה \textbf{על} $g: B \to A$.
\end{defbox}

\begin{defbox}
\textbf{הגדרה: $|A| < |B|$}

$|A| < |B|$ אם $|A| \leq |B|$ וגם $|A| \neq |B|$.
\end{defbox}

\subsubsection{המשפט}

\begin{thmbox}
\textbf{משפט קנטור-שרדר-ברנשטיין (קש``ב)}

אם $|A| \leq |B|$ וגם $|B| \leq |A|$, אז $|A| = |B|$.

במילים: אם קיימת פונקציה חח``ע מ-$A$ ל-$B$ וגם מ-$B$ ל-$A$, אז קיים זיווג ביניהן.
\end{thmbox}

\begin{notebox}
\textbf{שימוש:}

כדי להוכיח $|A| = |B|$, מספיק למצוא:
\begin{itemize}
    \item פונקציה חח``ע $f: A \to B$
    \item פונקציה חח``ע $g: B \to A$
\end{itemize}

זה לפעמים קל יותר מלמצוא זיווג ישירות!
\end{notebox}

%====================================
\subsection{פעולות על עוצמות}
%====================================

\subsubsection{חיבור עוצמות}

\begin{defbox}
\textbf{הגדרה: חיבור עוצמות}

\[|A| + |B| = |A \uplus B|\]

כאשר $A \uplus B$ הוא \textbf{איחוד זר} (disjoint union):
\[A \uplus B = (\{0\} \times A) \cup (\{1\} \times B)\]
\end{defbox}

\subsubsection{כפל עוצמות}

\begin{defbox}
\textbf{הגדרה: כפל עוצמות}

\[|A| \cdot |B| = |A \times B|\]
\end{defbox}

\subsubsection{חזקת עוצמות}

\begin{defbox}
\textbf{הגדרה: חזקת עוצמות}

\[|A|^{|B|} = |B \to A|\]

כלומר: עוצמת קבוצת כל הפונקציות מ-$B$ ל-$A$.
\end{defbox}

\begin{thmbox}
\textbf{טענה: קבוצת החזקה}

\[|\mathcal{P}(A)| = |A \to \{0,1\}| = 2^{|A|}\]

\textbf{הוכחה (רעיון):} לכל $X \subseteq A$ מתאימה פונקציה אופיינית $\chi_X: A \to \{0,1\}$:
\[\chi_X(a) = \begin{cases} 1 & a \in X \\ 0 & a \notin X \end{cases}\]
\end{thmbox}

\subsubsection{תכונות}

\begin{thmbox}
\textbf{משפט: תכונות פעולות על עוצמות}

\begin{enumerate}
    \item $|A| + |B| = |B| + |A|$
    \item $|A| \cdot |B| = |B| \cdot |A|$
    \item $(|A| + |B|) + |C| = |A| + (|B| + |C|)$
    \item $(|A| \cdot |B|) \cdot |C| = |A| \cdot (|B| \cdot |C|)$
    \item $|A| \cdot (|B| + |C|) = |A| \cdot |B| + |A| \cdot |C|$
    \item $|A|^{|B| + |C|} = |A|^{|B|} \cdot |A|^{|C|}$
    \item $(|A| \cdot |B|)^{|C|} = |A|^{|C|} \cdot |B|^{|C|}$
    \item $(|A|^{|B|})^{|C|} = |A|^{|B| \cdot |C|}$
\end{enumerate}
\end{thmbox}

%====================================
\subsection{טבלת סיכום: עוצמות נפוצות}
%====================================

\begin{center}
\begin{tabular}{|l|l|l|}
\hline
\rowcolor{tableheader}\color{white}\textbf{קבוצה} & \color{white}\textbf{עוצמה} & \color{white}\textbf{הערות} \\
\hline
\rowcolor{tablerow1} $\emptyset$ & $0$ & הקבוצה הריקה \\
\hline
\rowcolor{tablerow2} $\{a\}$ & $1$ & סינגלטון \\
\hline
\rowcolor{tablerow1} $\{1, \ldots, n\}$ & $n$ & קבוצה סופית \\
\hline
\rowcolor{tablerow2} $\N$ & $\aleph_0$ & בת-מניה \\
\hline
\rowcolor{tablerow1} $\Z$ & $\aleph_0$ & בת-מניה \\
\hline
\rowcolor{tablerow2} $\Q$ & $\aleph_0$ & בת-מניה \\
\hline
\rowcolor{tablerow1} $\R$ & $\aleph = 2^{\aleph_0}$ & עוצמת הרצף \\
\hline
\rowcolor{tablerow2} $\mathcal{P}(\N)$ & $\aleph = 2^{\aleph_0}$ & עוצמת הרצף \\
\hline
\rowcolor{tablerow1} $\N \to \{0,1\}$ & $\aleph = 2^{\aleph_0}$ & עוצמת הרצף \\
\hline
\end{tabular}
\end{center}

%====================================
\subsection{שגיאות נפוצות}
%====================================

\begin{notebox}
\textbf{שגיאה 1: לשכוח להוכיח $h \in A$}

בהוכחת לכסון, לא מספיק להגדיר $h$ -- צריך להוכיח שהיא אכן איבר בקבוצה $A$!
\end{notebox}

\begin{notebox}
\textbf{שגיאה 2: בלבול בין `על' ל`חח``ע'}

\begin{itemize}
    \item $|A| \leq |B|$ דורש חח``ע מ-$A$ ל-$B$
    \item לחילופין: על מ-$B$ ל-$A$
\end{itemize}
\end{notebox}

\begin{notebox}
\textbf{שגיאה 3: הנחה שתת-קבוצה קטנה יותר}

תת-קבוצה אמיתית יכולה להיות \textbf{שווה בעוצמה} לקבוצה!

דוגמה: $\N^+ \subset \N$ אבל $|\N^+| = |\N| = \aleph_0$
\end{notebox}

%====================================
\subsection{תרגילים לתרגול}
%====================================

\begin{exbox}
\textbf{תרגיל 1:}
הוכיחו שקבוצת כל הפולינומים עם מקדמים שלמים היא בת-מניה.
\end{exbox}

\begin{exbox}
\textbf{תרגיל 2:}
הוכיחו באמצעות משפט קש``ב ש-$|(0,1)| = |[0,1]|$.
\end{exbox}

\begin{exbox}
\textbf{תרגיל 3:}
הוכיחו בשיטת הלכסון שקבוצת כל הפונקציות $f: \N \to \N$ שהן עולות ממש אינה בת-מניה.
\end{exbox}

\begin{exbox}
\textbf{תרגיל 4:}
חשבו: $\aleph_0 + \aleph_0$, $\aleph_0 \cdot \aleph_0$, $2^{\aleph_0}$.
\end{exbox}


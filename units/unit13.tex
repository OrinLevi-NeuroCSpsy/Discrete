% יחידה 13 - חשבון עוצמות, איחוד בן מניה
%====================================
\section{יחידה 13: חשבון עוצמות, איחוד בן מניה}

\subsection{חשבון עוצמות - המשך}
%====================================

\begin{thmbox}
\textbf{טענה: חיבור עוצמות אינסופיות}

לכל עוצמה אינסופית $a$:
\[a + \aleph_0 = a\]
\end{thmbox}

\begin{thmbox}
\textbf{טענה: כפל עוצמות אינסופיות}

לכל עוצמה אינסופית $a$:
\[a \cdot \aleph_0 = a\]

ובפרט:
\[\aleph_0 \cdot \aleph_0 = \aleph_0\]
\[\aleph \cdot \aleph = \aleph\]
\end{thmbox}

\subsection{חישובים חשובים}
%====================================

\begin{center}
\begin{tabular}{|l|l|l|}
\hline
\rowcolor{tableheader}\color{white}\textbf{ביטוי} & \color{white}\textbf{תוצאה} & \color{white}\textbf{הסבר} \\
\hline
\rowcolor{tablerow1} $\aleph_0 + \aleph_0$ & $\aleph_0$ & $\N \cup \N \sim \N$ \\
\hline
\rowcolor{tablerow2} $\aleph_0 \cdot \aleph_0$ & $\aleph_0$ & $\N \times \N \sim \N$ \\
\hline
\rowcolor{tablerow1} $2^{\aleph_0}$ & $\aleph$ & $\mathcal{P}(\N) \sim \R$ \\
\hline
\rowcolor{tablerow2} $\aleph + \aleph$ & $\aleph$ & $\R \cup \R \sim \R$ \\
\hline
\rowcolor{tablerow1} $\aleph \cdot \aleph$ & $\aleph$ & $\R \times \R \sim \R$ \\
\hline
\rowcolor{tablerow2} $\aleph^{\aleph_0}$ & $\aleph$ & $(\N \to \R) \sim \R$ \\
\hline
\rowcolor{tablerow1} $2^{\aleph}$ & $> \aleph$ & משפט קנטור \\
\hline
\end{tabular}
\end{center}

\subsection{איחוד בן מניה}
%====================================

\begin{thmbox}
\textbf{משפט: איחוד בן מניה}

יהי $\{A_i \mid i \in I\}$ אוסף \textbf{לכל היותר בן מניה} ($|I| \leq \aleph_0$) של קבוצות \textbf{לכל היותר בנות מניה} ($|A_i| \leq \aleph_0$ לכל $i \in I$).

אזי האיחוד $\bigcup_{i \in I} A_i$ הוא \textbf{לכל היותר בן מניה}:
\[\left| \bigcup_{i \in I} A_i \right| \leq \aleph_0\]
\end{thmbox}

\begin{notebox}
\textbf{מקרה פרטי}

איחוד בן מניה של קבוצות בנות מניה הוא בן מניה.
\end{notebox}

\subsection{יישומים}
%====================================

\begin{exbox}
\textbf{דוגמה 1: $\Q$ בת מניה}

\[\Q = \bigcup_{n \in \Z \setminus \{0\}} \left\{ \frac{m}{n} \mid m \in \Z \right\}\]

\begin{itemize}
    \item $\Z \setminus \{0\}$ בת מניה (אינדקס בן מניה)
    \item לכל $n$, הקבוצה $\left\{ \frac{m}{n} \mid m \in \Z \right\}$ בת מניה
    \item לכן $\Q$ בת מניה
\end{itemize}
\end{exbox}

\begin{exbox}
\textbf{דוגמה 2: מחרוזות בינאריות מיוחדות}

מצאו את עוצמת המחרוזות הבינאריות האינסופיות בהן 1 לא מופיע פעמיים ברצף.

\textbf{פתרון:} נסמן את הקבוצה ב-$A$.

\textbf{חסם עליון:} $A \subseteq \N \to \{0,1\}$, לכן:
\[|A| \leq |\N \to \{0,1\}| = 2^{\aleph_0}\]

\textbf{חסם תחתון:} נגדיר $F \in (\N_{\text{even}} \to \{0,1\}) \to A$ ע``י:
\[F = \lambda g \in \N_{\text{even}} \to \{0,1\}. \lambda n \in \N. \begin{cases} g(n) & n \in \N_{\text{even}} \\ 0 & n \in \N_{\text{odd}} \end{cases}\]

$F$ חח``ע, לכן $|A| \geq |\N_{\text{even}} \to \{0,1\}| = 2^{\aleph_0}$.

מקש``ב: $|A| = 2^{\aleph_0}$.
\end{exbox}

\subsection{אקסיומת הבחירה (העשרה)}
%====================================

\begin{defbox}
\textbf{הגדרה: פונקציית בחירה}

תהי $X$ קבוצה שאיבריה הם קבוצות לא-ריקות.

\textbf{פונקציית בחירה} היא פונקציה $f \in X \to \bigcup X$ המקיימת:
\[\forall A \in X. f(A) \in A\]

הרעיון: לכל איבר $A$ של $X$, הפונקציה ``בוחרת'' איבר $f(A)$ מתוך $A$.
\end{defbox}

\begin{thmbox}
\textbf{אקסיומת הבחירה}

תהי $X$ קבוצה שאיבריה הם קבוצות לא ריקות. אז קיימת פונקציית בחירה $f \in X \to \bigcup X$.
\end{thmbox}

\begin{exbox}
\textbf{דוגמה: קבוצות סופיות}

עבור $X = \{\{1,2,3\}, \{4,5\}\}$:

\begin{itemize}
    \item $\bigcup X = \{1,2,3,4,5\}$
    \item פונקציית בחירה אפשרית: $f(\{1,2,3\}) = 2$, $f(\{4,5\}) = 4$
\end{itemize}
\end{exbox}

\begin{exbox}
\textbf{דוגמה: תתי-קבוצות של $\N$}

עבור $X = \mathcal{P}(\N) \setminus \{\emptyset\}$:

ניתן להגדיר $f = \lambda A \in X. \min(A)$ (בלי אקסיומת הבחירה!)
\end{exbox}

\begin{notebox}
\textbf{מתי צריך את האקסיומה?}

עבור $X = \mathcal{P}(\R) \setminus \{\emptyset\}$, לא ניתן להגדיר פונקציית בחירה במפורש (אין מינימום ב-$\R$).

אקסיומת הבחירה מבטיחה שקיימת פונקציה כזו, מבלי להגדיר אותה במפורש.
\end{notebox}

\subsection{סיכום היחידה}
%====================================

\begin{thmbox}
\textbf{נקודות מפתח}

\begin{enumerate}
    \item \textbf{איחוד בן מניה:} איחוד של לכל היותר $\aleph_0$ קבוצות בנות מניה הוא בן מניה
    \item \textbf{חשבון עוצמות:} $\aleph_0 \cdot \aleph_0 = \aleph_0$, $\aleph \cdot \aleph = \aleph$
    \item \textbf{שיטת קש``ב:} למציאת עוצמה -- מוצאים חסם עליון ותחתון שווים
    \item \textbf{אקסיומת הבחירה:} מבטיחה קיום פונקציית בחירה לכל קבוצת קבוצות לא-ריקות
\end{enumerate}
\end{thmbox}


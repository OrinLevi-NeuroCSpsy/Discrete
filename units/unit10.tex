% יחידה 10 - עוצמות
%====================================
\section{יחידה 10: עוצמות}

\subsection{מושגי יסוד}
%====================================

\subsubsection{שוויון עוצמות}

\begin{defbox}
\textbf{הגדרה: שוויון עוצמות}

שתי קבוצות $A$ ו-$B$ הן \textbf{שוות עוצמה} (equinumerous) אם קיים ביניהן \textbf{זיווג} (פונקציה חח``ע ועל).

\textbf{סימון:} $|A| = |B|$ או $A \sim B$
\end{defbox}

\begin{thmbox}
\textbf{טענה: יחס שקילות}

היחס $\sim$ (שוויון עוצמות) מתנהג כיחס שקילות:

\begin{enumerate}
    \item \textbf{רפלקסיביות:} $|A| = |A|$ (פונקציית הזהות היא זיווג)
    \item \textbf{סימטריות:} $|A| = |B| \Rightarrow |B| = |A|$ (הפונקציה ההפוכה היא זיווג)
    \item \textbf{טרנזיטיביות:} $|A| = |B| \land |B| = |C| \Rightarrow |A| = |C|$ (הרכבת זיווגים)
\end{enumerate}
\end{thmbox}

\subsubsection{סימונים מיוחדים}

\begin{defbox}
\textbf{הגדרה: $\aleph_0$ (אלף-אפס)}

\[\aleph_0 = |\N|\]

קבוצה $A$ נקראת \textbf{בת-מניה} (countable) אם $|A| = \aleph_0$.
\end{defbox}

\begin{defbox}
\textbf{הגדרה: $\aleph$ (עוצמת הרצף)}

\[\aleph = |\R|\]

קבוצה בעוצמה $\aleph$ נקראת \textbf{בעוצמת הרצף} (continuum).
\end{defbox}

\subsection{קבוצות בנות-מניה}
%====================================

\begin{thmbox}
\textbf{טענה: קבוצות בנות-מניה}

\begin{enumerate}
    \item $|\N^k| = |\N \times \N| = |\N| = \aleph_0$ לכל $k > 0$
    \item $|\Q| = |\Z| = \aleph_0$
    \item תת-קבוצה אינסופית של קבוצה בת-מניה היא בת-מניה
    \item איחוד בן-מניה של קבוצות בנות-מניה הוא בן-מניה
\end{enumerate}
\end{thmbox}

\begin{exbox}
\textbf{דוגמאות לקבוצות בנות-מניה:}

\begin{itemize}
    \item $\N$, $\Z$, $\Q$
    \item $\N \times \N$
    \item קבוצת כל המחרוזות הסופיות מעל א``ב סופי
    \item קבוצת כל הפולינומים עם מקדמים רציונליים
\end{itemize}
\end{exbox}

\subsection{המלון של הילברט}
%====================================

\begin{exbox}
\textbf{המלון של הילברט (Hilbert's Hotel)}

\textbf{הרעיון:} מלון עם אינסוף חדרים, ממוספרים $0, 1, 2, 3, \ldots$

\textbf{תרחיש 1:} המלון מלא, מגיע אורח אחד חדש.
\begin{itemize}
    \item פתרון: כל אורח עובר לחדר הבא ($n \to n+1$)
    \item החדר 0 מתפנה לאורח החדש
    \item זיווג: $f(n) = n+1$ הוא חח``ע מ-$\N$ ל-$\N \setminus \{0\}$
\end{itemize}

\textbf{תרחיש 2:} מגיעים אינסוף אורחים חדשים.
\begin{itemize}
    \item פתרון: כל אורח עובר לחדר הזוגי ($n \to 2n$)
    \item החדרים האי-זוגיים מתפנים לאורחים החדשים
    \item זיווג: $\N \cup \N \sim \N$
\end{itemize}
\end{exbox}

\begin{thmbox}
\textbf{מסקנה}

\[\aleph_0 + 1 = \aleph_0 \quad \text{וגם} \quad \aleph_0 + \aleph_0 = \aleph_0\]

קבוצה אינסופית יכולה להיות שווה בעוצמה לתת-קבוצה ממש שלה!
\end{thmbox}

\subsection{עוצמת הרצף}
%====================================

\begin{thmbox}
\textbf{טענה: עוצמת קטעים}

לכל $a < b$ ממשיים:
\[|[a,b]| = |[a,b)| = |(a,b]| = |(a,b)| = \aleph\]
\end{thmbox}

\begin{thmbox}
\textbf{טענה: קבוצות בעוצמת הרצף}

\begin{itemize}
    \item $|\R| = \aleph$
    \item $|[0,1]| = \aleph$
    \item $|\R^n| = \aleph$ לכל $n \geq 1$
    \item $|\mathcal{P}(\N)| = \aleph$
    \item $|\N \to \{0,1\}| = \aleph$
\end{itemize}
\end{thmbox}

\begin{thmbox}
\textbf{משפט קנטור}

\[\aleph_0 < \aleph\]

כלומר: $\R$ \textbf{אינה בת-מניה}.
\end{thmbox}

\subsection{סדר על עוצמות}
%====================================

\begin{defbox}
\textbf{הגדרה: $|A| \leq |B|$}

$|A| \leq |B|$ אם קיימת פונקציה \textbf{חח``ע} $f: A \to B$.

שקול: קיימת פונקציה \textbf{על} $g: B \to A$.
\end{defbox}

\begin{defbox}
\textbf{הגדרה: $|A| < |B|$}

$|A| < |B|$ אם $|A| \leq |B|$ וגם $|A| \neq |B|$.
\end{defbox}

\subsection{טבלת סיכום: עוצמות נפוצות}
%====================================

\begin{center}
\begin{tabular}{|l|l|l|}
\hline
\rowcolor{tableheader}\color{white}\textbf{קבוצה} & \color{white}\textbf{עוצמה} & \color{white}\textbf{הערות} \\
\hline
\rowcolor{tablerow1} $\emptyset$ & $0$ & הקבוצה הריקה \\
\hline
\rowcolor{tablerow2} $\{a\}$ & $1$ & סינגלטון \\
\hline
\rowcolor{tablerow1} $\{1, \ldots, n\}$ & $n$ & קבוצה סופית \\
\hline
\rowcolor{tablerow2} $\N$ & $\aleph_0$ & בת-מניה \\
\hline
\rowcolor{tablerow1} $\Z$ & $\aleph_0$ & בת-מניה \\
\hline
\rowcolor{tablerow2} $\Q$ & $\aleph_0$ & בת-מניה \\
\hline
\rowcolor{tablerow1} $\R$ & $\aleph = 2^{\aleph_0}$ & עוצמת הרצף \\
\hline
\rowcolor{tablerow2} $\mathcal{P}(\N)$ & $\aleph = 2^{\aleph_0}$ & עוצמת הרצף \\
\hline
\end{tabular}
\end{center}


% יחידה 1 - אינדוקציה מתמטית
%====================================
\section{יחידה 1: אינדוקציה מתמטית}

\subsection{מבוא}
%====================================

\begin{defbox}
\textbf{מהי אינדוקציה?}

\textbf{אינדוקציה מתמטית} היא שיטה (שימושית מאוד) להוכחת טענות מהצורה:

\begin{quote}
לכל מספר טבעי $n$ מתקיים $P(n)$
\end{quote}

כאשר $\N = \{0, 1, 2, \ldots\}$ היא קבוצת המספרים הטבעיים, ו-$P(n)$ היא טענה כלשהי בנוגע למספר הטבעי $n$.
\end{defbox}

\begin{exbox}
\textbf{דוגמאות לטענות שמוכיחים באינדוקציה:}
\begin{itemize}
    \item לכל $n$ טבעי מתקיים: $0 + 1 + \ldots + n = \frac{n(n+1)}{2}$
    \item לכל $n$ טבעי מתקיים: $n^3 - n$ מתחלק ב-$6$
\end{itemize}
\end{exbox}

%====================================
\subsection{עקרון האינדוקציה}
%====================================

\subsubsection{אינדוקציה רגילה (פשוטה)}

\begin{thmbox}
\textbf{עקרון האינדוקציה המתמטית}

כדי להוכיח שטענה $P(n)$ מתקיימת לכל מספר טבעי $n \geq 0$, מספיק להוכיח:

\begin{enumerate}
    \item \textbf{בסיס האינדוקציה:} $P(0)$ נכון
    \item \textbf{צעד האינדוקציה:} לכל $n \geq 1$ טבעי, אם $P(n-1)$ נכון אז $P(n)$ נכון
\end{enumerate}

ההנחה $P(n-1)$ נקראת \textbf{הנחת האינדוקציה}.
\end{thmbox}

\begin{notebox}
\textbf{ניסוח שקול:}

בצעד האינדוקציה אפשר גם להוכיח:
\begin{quote}
אם $P(n)$ אז $P(n+1)$, עבור $n \geq 0$
\end{quote}
\end{notebox}

\subsubsection{מדוע האינדוקציה עובדת?}

\begin{defbox}
\textbf{אינטואיציה -- אפקט הדומינו}

אפשר לדמות את עקרון האינדוקציה לאפקט דומינו:

\begin{enumerate}
    \item \textbf{בסיס:} הקלף הראשון נופל (מוכיחים $P(0)$)
    \item \textbf{צעד:} כל קלף שנופל מפיל את הבא אחריו (אם $P(n)$ אז $P(n+1)$)
\end{enumerate}

מסקנה: כל הקלפים יפלו (הטענה נכונה לכל $n$)
\end{defbox}

%====================================
\subsection{דוגמאות בסיסיות}
%====================================

\subsubsection{דוגמה 1: סכום סדרה חשבונית}

\begin{exbox}
\textbf{הוכחה:} $0 + 1 + \ldots + n = \frac{n(n+1)}{2}$

\textbf{בסיס:} עבור $n = 0$:
\[0 = \frac{0 \cdot (0+1)}{2} = 0 \quad \checkmark\]

\textbf{צעד:} יהי $n \geq 1$ טבעי. נניח שהטענה נכונה עבור $n-1$ (הנחת האינדוקציה):
\[0 + 1 + \ldots + (n-1) = \frac{(n-1)n}{2}\]

נוכיח עבור $n$:
\begin{align*}
0 + 1 + \ldots + n &= (0 + 1 + \ldots + (n-1)) + n \\
&\stackrel{*}{=} \frac{(n-1)n}{2} + n \\
&= \frac{n^2 - n + 2n}{2} \\
&= \frac{n^2 + n}{2} = \frac{n(n+1)}{2}
\end{align*}

כש-$(*)$ היא הנחת האינדוקציה.
\end{exbox}

\subsubsection{דוגמה 2: אי-שוויון אקספוננציאלי}

\begin{exbox}
\textbf{הוכחה:} לכל $n \geq 5$ מתקיים $2^n > n^2$

\textbf{בסיס:} עבור $n = 5$:
\[2^5 = 32 > 25 = 5^2 \quad \checkmark\]

\textbf{צעד:} יהי $n \geq 5$ טבעי. נניח $2^n > n^2$ (הנחת האינדוקציה). נוכיח עבור $n+1$:
\[2^{n+1} = 2 \cdot 2^n \stackrel{*}{>} 2n^2\]

צריך להראות: $2n^2 > (n+1)^2$, כלומר $2n^2 - (n+1)^2 > 0$.
\[2n^2 - (n+1)^2 = 2n^2 - n^2 - 2n - 1 = n^2 - 2n - 1 = (n-1)^2 - 2\]

עבור $n \geq 5$: $(n-1)^2 - 2 \geq 4^2 - 2 = 14 > 0$ \checkmark
\end{exbox}

%====================================
\subsection{אינדוקציה עם בסיס שונה מ-0}
%====================================

\begin{notebox}
\textbf{התאמת הבסיס}

אם הטענה היא: ``לכל מספר טבעי $n \geq k$ מתקיים $P(n)$'', אז:

\begin{itemize}
    \item \textbf{בסיס:} מוכיחים $P(k)$
    \item \textbf{צעד:} לכל $n \geq k+1$ מוכיחים: $P(n-1) \Rightarrow P(n)$
\end{itemize}
\end{notebox}

%====================================
\subsection{אינדוקציה מלאה (שלמה)}
%====================================

\subsubsection{למה צריך אינדוקציה מלאה?}

\begin{exbox}
\textbf{הבעיה: כל $n \geq 2$ ניתן לכתיבה כמכפלה של ראשוניים}

\textbf{ניסיון כושל באינדוקציה רגילה:}
\begin{itemize}
    \item בסיס: $n=2$ הוא ראשוני \checkmark
    \item צעד: נניח $n-1 = p_1 \cdot p_2 \cdot \ldots \cdot p_k$
    \item אז $n = p_1 \cdot p_2 \cdot \ldots \cdot p_k + 1$... \textbf{נתקענו!}
\end{itemize}

הבעיה: אין קשר בין פירוק של $n-1$ לפירוק של $n$.
\end{exbox}

\subsubsection{עקרון האינדוקציה המלאה}

\begin{thmbox}
\textbf{עקרון האינדוקציה המלאה (השלמה)}

כדי להוכיח שטענה $P(n)$ מתקיימת לכל מספר טבעי $n \geq k$:

\begin{enumerate}
    \item \textbf{בסיס:} $P(k)$ נכון
    \item \textbf{צעד:} לכל $n \geq k+1$: אם $P(k), P(k+1), \ldots, P(n-1)$ כולם נכונים, אז $P(n)$ נכון
\end{enumerate}

במילים אחרות: כדי להוכיח $P(n)$ מותר להניח את $P(m)$ \textbf{לכל} $k \leq m < n$.
\end{thmbox}

\begin{notebox}
\textbf{הערה:}
עקרון האינדוקציה המלאה שקול לעקרון האינדוקציה הרגיל (ניתן להוכיח אחד מהשני). אך לפעמים נוח יותר להשתמש בגרסה המלאה.
\end{notebox}

\subsubsection{הוכחת משפט הראשוניים}

\begin{proofbox}
\textbf{הוכחה: כל $n \geq 2$ ניתן לכתיבה כמכפלה של ראשוניים}

\textbf{בסיס:} עבור $n = 2$: זו מכפלה של הראשוני $p = 2$ \checkmark

\textbf{צעד (אינדוקציה מלאה):} יהי $n \geq 3$ טבעי. נניח שהטענה נכונה לכל $m$ כך ש-$2 \leq m < n$.

\textbf{מקרה 1:} אם $n$ ראשוני -- סיימנו.

\textbf{מקרה 2:} אם $n$ פריק, קיימים $m_1, m_2$ טבעיים בטווח $2 \leq m_1, m_2 < n$ כך ש-$n = m_1 \cdot m_2$.

מהנחת האינדוקציה (המלאה!):
\begin{itemize}
    \item $m_1 = p_1 \cdot \ldots \cdot p_{k_1}$
    \item $m_2 = q_1 \cdot \ldots \cdot q_{k_2}$
\end{itemize}

לכן: $n = m_1 \cdot m_2 = p_1 \cdot \ldots \cdot p_{k_1} \cdot q_1 \cdot \ldots \cdot q_{k_2}$ \checkmark
\end{proofbox}

%====================================
\subsection{שגיאות נפוצות}
%====================================

\subsubsection{שגיאה 1: בסיס לא נכון}

\begin{notebox}
\textbf{דוגמה: ``כל הסוסים בעולם באותו צבע''}

\textbf{טענה שגויה:} לכל $n \geq 1$, כל קבוצה של $n$ סוסים -- כולם באותו צבע.

\textbf{ניסיון הוכחה:}
\begin{itemize}
    \item \textbf{בסיס:} $n=1$ -- ברור שסוס אחד הוא באותו צבע כמו עצמו \checkmark
    \item \textbf{צעד:} נניח הטענה נכונה ל-$n$, נוכיח ל-$n+1$.
    \begin{itemize}
        \item ניקח סוסים $a_1, \ldots, a_{n+1}$
        \item הקבוצה $\{a_1, \ldots, a_n\}$ -- כולם באותו צבע (הנחת האינדוקציה)
        \item הקבוצה $\{a_2, \ldots, a_{n+1}\}$ -- כולם באותו צבע (הנחת האינדוקציה)
        \item הסוס $a_2$ משותף לשתיהן, לכן כולם באותו צבע!
    \end{itemize}
\end{itemize}

\textbf{היכן השגיאה?}
הצעד נכשל במעבר מ-$n=1$ ל-$n=2$! עבור שני סוסים $a_1, a_2$:
\begin{itemize}
    \item הקבוצה $\{a_1\}$ -- סוס אחד
    \item הקבוצה $\{a_2\}$ -- סוס אחד
\end{itemize}
\textbf{אין חפיפה} בין הקבוצות! אין סוס משותף שמחבר בין הצבעים.
\end{notebox}

\subsubsection{שגיאה 2: שכחת הבסיס}

\begin{notebox}
\textbf{תמיד לוודא את הבסיס!}
\begin{itemize}
    \item הבסיס הוא חלק הכרחי מההוכחה
    \item בלי בסיס, צעד האינדוקציה לבד לא מוכיח כלום
    \item יש לבדוק שהבסיס אכן מתקיים
\end{itemize}
\end{notebox}

\subsubsection{שגיאה 3: הנחת מה שצריך להוכיח}

\begin{notebox}
\textbf{אל תניחו את $P(n)$!}

בצעד האינדוקציה:
\begin{itemize}
    \item \cmark \textbf{נכון:} מניחים $P(n-1)$ ומוכיחים $P(n)$
    \item \xmark \textbf{שגוי:} מניחים $P(n)$ (זה מה שצריך להוכיח!)
\end{itemize}
\end{notebox}

%====================================
\subsection{סיכום שיטות האינדוקציה}
%====================================

\begin{center}
\begin{tabular}{|l|l|l|l|}
\hline
\rowcolor{tableheader}\color{white}\textbf{סוג} & \color{white}\textbf{בסיס} & \color{white}\textbf{צעד} & \color{white}\textbf{מתי להשתמש} \\
\hline
\rowcolor{tablerow1} \textbf{רגילה} & $P(0)$ & $P(n-1) \Rightarrow P(n)$ & קשר ישיר בין $n$ ל-$n-1$ \\
\hline
\rowcolor{tablerow2} \textbf{מלאה} & $P(k)$ & $\forall m < n: P(m) \Rightarrow P(n)$ & צריך ערכים קטנים יותר \\
\hline
\end{tabular}
\end{center}

%====================================
\subsection{תרגילים לתרגול}
%====================================

\begin{exbox}
\textbf{תרגיל 1:}
הוכיחו באינדוקציה: לכל $n \geq 1$ מתקיים $1 + 2 + 4 + \ldots + 2^{n-1} = 2^n - 1$
\end{exbox}

\begin{exbox}
\textbf{תרגיל 2:}
הוכיחו באינדוקציה: לכל $n \geq 1$ מתקיים $1^2 + 2^2 + \ldots + n^2 = \frac{n(n+1)(2n+1)}{6}$
\end{exbox}

\begin{exbox}
\textbf{תרגיל 3:}
הוכיחו באינדוקציה שלמה: כל מספר טבעי $n \geq 12$ ניתן לייצוג כסכום של כפולות של 4 ו-5.

\textbf{רמז:} $12 = 4 \cdot 3$, $13 = 4 + 4 + 5$, $14 = 4 + 5 + 5$, $15 = 5 \cdot 3$
\end{exbox}


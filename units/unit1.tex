% יחידה 1 - לוגיקה ופסוקים
%====================================
\section{יחידה 1: לוגיקה ופסוקים}

\subsection{מבוא לקורס}
%====================================

\subsubsection{מטרת הקורס}
מטרת הקורס היא להכשיר כלים מתמטיים של מדעי המחשב:
\begin{itemize}
    \item שפה מתמטית מדויקת
    \item הוכחות מתמטיות
    \item מושגים בסיסיים (קבוצות, יחסים, פונקציות וכו')
\end{itemize}

\subsubsection{נושאי הקורס}
\begin{enumerate}
    \item קבוצות, יחסים והעתקות (כולל אינדוקציה)
    \item מושג המספר ואלגברה בסיסית
    \item שיטות מנייה והסתברות
    \item קומבינטוריקה ואלגוריתמים
\end{enumerate}

%====================================
\subsection{היגדים ולוגיקה}
%====================================

\subsubsection{מוטיבציה}
רוצים להגדיר שפה פורמלית לתיאור טענות מתמטיות. לשם כך נשתמש ב\textbf{לוגיקה} -- תורה העוסקת במבנה הפורמלי של טענות וב\textbf{ערכי אמת} שלהן.

\subsubsection{פסוק (היגד)}
\begin{defbox}
\textbf{הגדרה: פסוק}

\textbf{פסוק} (או \textbf{היגד}, באנגלית: Proposition) הוא קביעה שניתן להכריע אם היא \textbf{אמת} ($T$ -- True) או \textbf{שקר} ($F$ -- False).
\end{defbox}

\begin{notebox}
\textbf{תכונות של פסוק:}
\begin{itemize}
    \item לפסוק יש ערך אמת \textbf{יחיד} -- אמת או שקר, אך לא שניהם.
    \item פסוק מוגדר \textbf{חד-משמעית} -- אין מצב ביניים.
\end{itemize}
\end{notebox}

\subsubsection{דוגמאות לפסוקים}
\begin{exbox}
\textbf{פסוקים:}
\begin{itemize}
    \item ``$2 > 1$'' -- פסוק (אמת)
    \item ``יורד גשם'' -- פסוק (ערכו תלוי במציאות, אך יש לו ערך אמת מוגדר)
    \item ``$5$ הוא מספר ראשוני'' -- פסוק (אמת)
    \item ``$4$ הוא מספר אי-זוגי'' -- פסוק (שקר)
\end{itemize}

\textbf{לא פסוקים:}
\begin{itemize}
    \item ``$x > 5$'' -- \textbf{לא פסוק!} (תלוי בערך של $x$, זה \textbf{נוסחה פתוחה})
    \item ``האם יורד גשם?'' -- לא פסוק (שאלה)
    \item ``סגור את הדלת'' -- לא פסוק (ציווי)
\end{itemize}
\end{exbox}

\textbf{הערה חשובה:} כשכותבים ``$x$ הוא מספר גדול מ-5'', זהו \textbf{משפט פתוח} (או נוסחה פתוחה) -- ערך האמת שלו תלוי ב-$x$. רק כשנקבע ערך ספציפי ל-$x$, הופך המשפט לפסוק.

\subsubsection{פרדוקסים}
\textbf{הגדרה:} \textbf{פרדוקס} הוא משפט שלא ניתן לייחס לו ערך אמת -- אם נניח שהוא אמת, נגיע לסתירה, ואם נניח שהוא שקר, גם נגיע לסתירה.

\textbf{דוגמה -- פרדוקס ברי (Berry Paradox):}
``המספר השלם הקטן ביותר שלא ניתן להגדיר בפחות מ-20 מילים.''

הסבר הפרדוקס:
\begin{itemize}
    \item המשפט עצמו מכיל פחות מ-20 מילים
    \item לכן, המשפט מגדיר מספר בפחות מ-20 מילים
    \item אבל המספר הוגדר כמספר שלא ניתן להגדיר בפחות מ-20 מילים -- סתירה!
\end{itemize}

\textbf{דוגמה -- פרדוקס השקרן:}
``המשפט הזה הוא שקר.''
\begin{itemize}
    \item אם המשפט אמת $\Rightarrow$ הוא אומר על עצמו שהוא שקר $\Rightarrow$ סתירה
    \item אם המשפט שקר $\Rightarrow$ הוא לא שקר $\Rightarrow$ הוא אמת $\Rightarrow$ סתירה
\end{itemize}

\textbf{הערה:} פרדוקסים אינם פסוקים! הם ממחישים את הצורך בהגדרות מדויקות ובזהירות בבניית טענות לוגיות.

%====================================
\subsection{קשרים לוגיים (קונקטורים)}
%====================================

קשרים לוגיים (או קונקטורים) הם פעולות שמאפשרות לבנות פסוקים מורכבים מפסוקים פשוטים יותר.

\subsubsection{שלילה (NOT)}
\textbf{הגדרה:} אם $P$ הוא פסוק, אז $\neg P$ (``לא $P$'') הוא פסוק ששולל את $P$.

\textbf{סימון:} $\neg P$ או $\lnot P$ או $\overline{P}$

\begin{center}
\textbf{טבלת אמת לשלילה:}

\vspace{0.5em}
\begin{tabular}{|C|C|}
\hline
\rowcolor{tableheader}\color{white}\textbf{$P$} & \color{white}\textbf{$\neg P$} \\
\hline
\rowcolor{tablerow1} \truthmark{T} & \truthmark{F} \\
\hline
\rowcolor{tablerow2} \truthmark{F} & \truthmark{T} \\
\hline
\end{tabular}
\end{center}

\subsubsection{וגם (AND) -- קוניונקציה}
\textbf{הגדרה:} אם $P$ ו-$Q$ הם פסוקים, אז $P \land Q$ (``$P$ וגם $Q$'') הוא פסוק שאמיתי \textbf{רק אם} שני הפסוקים אמיתיים.

\textbf{סימון:} $P \land Q$ או $P \cdot Q$ או $P \& Q$

\begin{center}
\textbf{טבלת אמת ל-AND:}

\vspace{0.5em}
\begin{tabular}{|C|C|C|}
\hline
\rowcolor{tableheader}\color{white}\textbf{$P$} & \color{white}\textbf{$Q$} & \color{white}\textbf{$P \land Q$} \\
\hline
\rowcolor{tablerow1} \truthmark{T} & \truthmark{T} & \truthmark{T} \\
\hline
\rowcolor{tablerow2} \truthmark{T} & \truthmark{F} & \truthmark{F} \\
\hline
\rowcolor{tablerow1} \truthmark{F} & \truthmark{T} & \truthmark{F} \\
\hline
\rowcolor{tablerow2} \truthmark{F} & \truthmark{F} & \truthmark{F} \\
\hline
\end{tabular}
\end{center}

\subsubsection{או (OR) -- דיסיונקציה}
\textbf{הגדרה:} אם $P$ ו-$Q$ הם פסוקים, אז $P \lor Q$ (``$P$ או $Q$'') הוא פסוק שאמיתי אם \textbf{לפחות אחד} מהפסוקים אמיתי.

\textbf{סימון:} $P \lor Q$ או $P + Q$

\textbf{שימו לב:} זהו ``או'' \textbf{לא מוציא} (inclusive OR) -- גם אם שניהם אמיתיים, התוצאה אמת.

\begin{center}
\textbf{טבלת אמת ל-OR:}

\vspace{0.5em}
\begin{tabular}{|C|C|C|}
\hline
\rowcolor{tableheader}\color{white}\textbf{$P$} & \color{white}\textbf{$Q$} & \color{white}\textbf{$P \lor Q$} \\
\hline
\rowcolor{tablerow1} \truthmark{T} & \truthmark{T} & \truthmark{T} \\
\hline
\rowcolor{tablerow2} \truthmark{T} & \truthmark{F} & \truthmark{T} \\
\hline
\rowcolor{tablerow1} \truthmark{F} & \truthmark{T} & \truthmark{T} \\
\hline
\rowcolor{tablerow2} \truthmark{F} & \truthmark{F} & \truthmark{F} \\
\hline
\end{tabular}
\end{center}

\subsubsection{או מוציא (XOR)}
\textbf{הגדרה:} $P \oplus Q$ (``$P$ או $Q$ אך לא שניהם'') הוא פסוק שאמיתי אם \textbf{בדיוק אחד} מהפסוקים אמיתי.

\textbf{סימון:} $P \oplus Q$ או $P \veebar Q$

\begin{center}
\textbf{טבלת אמת ל-XOR:}

\vspace{0.5em}
\begin{tabular}{|C|C|C|}
\hline
\rowcolor{tableheader}\color{white}\textbf{$P$} & \color{white}\textbf{$Q$} & \color{white}\textbf{$P \oplus Q$} \\
\hline
\rowcolor{tablerow1} \truthmark{T} & \truthmark{T} & \truthmark{F} \\
\hline
\rowcolor{tablerow2} \truthmark{T} & \truthmark{F} & \truthmark{T} \\
\hline
\rowcolor{tablerow1} \truthmark{F} & \truthmark{T} & \truthmark{T} \\
\hline
\rowcolor{tablerow2} \truthmark{F} & \truthmark{F} & \truthmark{F} \\
\hline
\end{tabular}
\end{center}

\subsubsection{גרירה (אימפליקציה)}
\textbf{הגדרה:} אם $P$ ו-$Q$ הם פסוקים, אז $P \then Q$ (``$P$ גורר $Q$'' או ``אם $P$ אז $Q$'') הוא פסוק.

\textbf{סימון:} $P \then Q$ או $P \Rightarrow Q$ או $P \supset Q$

\textbf{פירוש:} ``אם $P$ אמת, אז $Q$ חייב להיות אמת''

\begin{center}
\textbf{טבלת אמת לגרירה:}

\vspace{0.5em}
\begin{tabular}{|C|C|C|}
\hline
\rowcolor{tableheader}\color{white}\textbf{$P$} & \color{white}\textbf{$Q$} & \color{white}\textbf{$P \then Q$} \\
\hline
\rowcolor{tablerow1} \truthmark{T} & \truthmark{T} & \truthmark{T} \\
\hline
\rowcolor{tablerow2} \truthmark{T} & \truthmark{F} & \truthmark{F} \\
\hline
\rowcolor{tablerow1} \truthmark{F} & \truthmark{T} & \truthmark{T} \\
\hline
\rowcolor{tablerow2} \truthmark{F} & \truthmark{F} & \truthmark{T} \\
\hline
\end{tabular}
\end{center}

\textbf{הערה חשובה על גרירה:}
\begin{itemize}
    \item הגרירה $P \then Q$ היא \textbf{שקר רק} כאשר $P$ אמת ו-$Q$ שקר.
    \item כאשר $P$ שקר, הגרירה \textbf{תמיד אמת}, ללא קשר לערך של $Q$!
    \item זה נקרא ``מהשקר נובע הכל'' (ex falso quodlibet).
    \item דוגמה: ``אם 2+2=5, אז אני נשיא ארה"ב'' -- זהו פסוק אמיתי!
\end{itemize}

\textbf{דוגמה:}
נניח $P$: ``יורד גשם'', $Q$: ``הכביש רטוב''.

$P \then Q$: ``אם יורד גשם, אז הכביש רטוב''.

\begin{itemize}
    \item אם יורד גשם והכביש רטוב -- הטענה אמת
    \item אם יורד גשם והכביש יבש -- הטענה שקר (הבטחה שנשברה)
    \item אם לא יורד גשם -- הטענה אמת (לא הבטחנו כלום על מקרה זה)
\end{itemize}

\subsubsection{שקילות (אם ורק אם)}
\textbf{הגדרה:} $P \iff Q$ (``$P$ אם ורק אם $Q$'') הוא פסוק שאמיתי כאשר ל-$P$ ול-$Q$ יש \textbf{אותו ערך אמת}.

\textbf{סימון:} $P \iff Q$ או $P \Leftrightarrow Q$ או $P \equiv Q$

\textbf{שקול ל:} $(P \then Q) \land (Q \then P)$

\begin{center}
\textbf{טבלת אמת לשקילות:}

\vspace{0.5em}
\begin{tabular}{|C|C|C|}
\hline
\rowcolor{tableheader}\color{white}\textbf{$P$} & \color{white}\textbf{$Q$} & \color{white}\textbf{$P \iff Q$} \\
\hline
\rowcolor{tablerow1} \truthmark{T} & \truthmark{T} & \truthmark{T} \\
\hline
\rowcolor{tablerow2} \truthmark{T} & \truthmark{F} & \truthmark{F} \\
\hline
\rowcolor{tablerow1} \truthmark{F} & \truthmark{T} & \truthmark{F} \\
\hline
\rowcolor{tablerow2} \truthmark{F} & \truthmark{F} & \truthmark{T} \\
\hline
\end{tabular}
\end{center}

%====================================
\subsection{טבלאות אמת}
%====================================

\textbf{הגדרה:} \textbf{טבלת אמת} היא טבלה המציגה את ערך האמת של פסוק מורכב עבור כל השילובים האפשריים של ערכי האמת של הפסוקים המרכיבים אותו.

\textbf{בניית טבלת אמת:} אם יש $n$ פסוקים אטומיים, יש $2^n$ שורות בטבלת האמת (כל השילובים האפשריים של $T$ ו-$F$).

\textbf{דוגמה: טבלת אמת ל-$(P \land Q) \then R$}

יש 3 משתנים, לכן $2^3 = 8$ שורות:

\begin{center}
\begin{tabular}{|C|C|C|C|C|}
\hline
\rowcolor{tableheader}\color{white}\textbf{$P$} & \color{white}\textbf{$Q$} & \color{white}\textbf{$R$} & \color{white}\textbf{$P \land Q$} & \color{white}\textbf{$(P \land Q) \then R$} \\
\hline
\rowcolor{tablerow1} \truthmark{T} & \truthmark{T} & \truthmark{T} & \truthmark{T} & \truthmark{T} \\
\hline
\rowcolor{tablerow2} \truthmark{T} & \truthmark{T} & \truthmark{F} & \truthmark{T} & \truthmark{F} \\
\hline
\rowcolor{tablerow1} \truthmark{T} & \truthmark{F} & \truthmark{T} & \truthmark{F} & \truthmark{T} \\
\hline
\rowcolor{tablerow2} \truthmark{T} & \truthmark{F} & \truthmark{F} & \truthmark{F} & \truthmark{T} \\
\hline
\rowcolor{tablerow1} \truthmark{F} & \truthmark{T} & \truthmark{T} & \truthmark{F} & \truthmark{T} \\
\hline
\rowcolor{tablerow2} \truthmark{F} & \truthmark{T} & \truthmark{F} & \truthmark{F} & \truthmark{T} \\
\hline
\rowcolor{tablerow1} \truthmark{F} & \truthmark{F} & \truthmark{T} & \truthmark{F} & \truthmark{T} \\
\hline
\rowcolor{tablerow2} \truthmark{F} & \truthmark{F} & \truthmark{F} & \truthmark{F} & \truthmark{T} \\
\hline
\end{tabular}
\end{center}

%====================================
\subsection{סיווג פסוקים}
%====================================

\textbf{טאוטולוגיה:} פסוק ש\textbf{תמיד אמת}, לכל השמה של ערכי אמת למשתנים.

דוגמה: $P \lor \neg P$ (חוק השלישי הנמנע)

\textbf{סתירה:} (או קונטרדיקציה) פסוק ש\textbf{תמיד שקר}, לכל השמה.

דוגמה: $P \land \neg P$

\textbf{פסוק ספיק:} (או מתקיים, contingent) פסוק שאינו טאוטולוגיה ואינו סתירה -- יש השמות שבהן הוא אמת ויש השמות שבהן הוא שקר.

דוגמה: $P \then Q$

\textbf{דוגמה: הוכחה שפסוק הוא טאוטולוגיה}

נוכיח ש-$P \lor \neg P$ הוא טאוטולוגיה:

\begin{center}
\begin{tabular}{|C|C|C|}
\hline
\rowcolor{tableheader}\color{white}\textbf{$P$} & \color{white}\textbf{$\neg P$} & \color{white}\textbf{$P \lor \neg P$} \\
\hline
\rowcolor{tablerow1} \truthmark{T} & \truthmark{F} & \truthmark{T} \\
\hline
\rowcolor{tablerow2} \truthmark{F} & \truthmark{T} & \truthmark{T} \\
\hline
\end{tabular}
\end{center}

בכל השורות התוצאה $T$, לכן זוהי טאוטולוגיה.

%====================================
\subsection{שקילות לוגית}
%====================================

\textbf{הגדרה:} שני פסוקים $P$ ו-$Q$ הם \textbf{שקולים לוגית} (ונסמן $P \equiv Q$) אם יש להם \textbf{אותם ערכי אמת} בכל טבלת האמת.

\textbf{שקול לכך:} $P \iff Q$ היא טאוטולוגיה.

\subsubsection{שקילויות חשובות}

\textbf{חוקי דה-מורגן:}
\begin{align}
\neg(P \land Q) &\equiv \neg P \lor \neg Q \\
\neg(P \lor Q) &\equiv \neg P \land \neg Q
\end{align}

\textbf{חוקי הכפלה והפיצול:}
\begin{align}
P \land (Q \lor R) &\equiv (P \land Q) \lor (P \land R) \quad \text{(חוק הכפלה)} \\
P \lor (Q \land R) &\equiv (P \lor Q) \land (P \lor R) \quad \text{(חוק הפיצול)}
\end{align}

\textbf{שקילויות נוספות:}
\begin{align}
P \then Q &\equiv \neg P \lor Q \\
P \then Q &\equiv \neg Q \then \neg P \quad \text{(קונטרפוזיציה)} \\
P \iff Q &\equiv (P \then Q) \land (Q \then P) \\
\neg(\neg P) &\equiv P \quad \text{(שלילה כפולה)}
\end{align}

\textbf{חוקי החילוף (קומוטטיביות):}
\begin{align}
P \land Q &\equiv Q \land P \\
P \lor Q &\equiv Q \lor P
\end{align}

\textbf{חוקי הקיבוץ (אסוציאטיביות):}
\begin{align}
(P \land Q) \land R &\equiv P \land (Q \land R) \\
(P \lor Q) \lor R &\equiv P \lor (Q \lor R)
\end{align}

\textbf{חוקי הזהות:}
\begin{align}
P \land T &\equiv P \\
P \lor F &\equiv P \\
P \land F &\equiv F \\
P \lor T &\equiv T
\end{align}

\textbf{חוקי האידמפוטנטיות:}
\begin{align}
P \land P &\equiv P \\
P \lor P &\equiv P
\end{align}

\textbf{חוקי הבליעה:}
\begin{align}
P \land (P \lor Q) &\equiv P \\
P \lor (P \land Q) &\equiv P
\end{align}

%====================================
\subsection{דוגמאות ותרגילים}
%====================================

\textbf{דוגמה 1: הוכחת שקילות באמצעות טבלת אמת}

נוכיח ש-$P \then Q \equiv \neg P \lor Q$:

\begin{center}
\begin{tabular}{|C|C|C|C|C|}
\hline
\rowcolor{tableheader}\color{white}\textbf{$P$} & \color{white}\textbf{$Q$} & \color{white}\textbf{$\neg P$} & \color{white}\textbf{$P \then Q$} & \color{white}\textbf{$\neg P \lor Q$} \\
\hline
\rowcolor{tablerow1} \truthmark{T} & \truthmark{T} & \truthmark{F} & \truthmark{T} & \truthmark{T} \\
\hline
\rowcolor{tablerow2} \truthmark{T} & \truthmark{F} & \truthmark{F} & \truthmark{F} & \truthmark{F} \\
\hline
\rowcolor{tablerow1} \truthmark{F} & \truthmark{T} & \truthmark{T} & \truthmark{T} & \truthmark{T} \\
\hline
\rowcolor{tablerow2} \truthmark{F} & \truthmark{F} & \truthmark{T} & \truthmark{T} & \truthmark{T} \\
\hline
\end{tabular}
\end{center}

העמודות $P \then Q$ ו-$\neg P \lor Q$ זהות, לכן הפסוקים שקולים.

\textbf{דוגמה 2: הוכחת שקילות באמצעות חוקים}

נוכיח ש-$\neg(P \then Q) \equiv P \land \neg Q$:

\begin{align*}
\neg(P \then Q) &\equiv \neg(\neg P \lor Q) && \text{(שקילות גרירה)} \\
&\equiv \neg(\neg P) \land \neg Q && \text{(דה-מורגן)} \\
&\equiv P \land \neg Q && \text{(שלילה כפולה)}
\end{align*}

\textbf{דוגמה 3: קונטרפוזיציה}

הטענה ``אם יורד גשם אז הכביש רטוב'' שקולה ל:

``אם הכביש לא רטוב אז לא יורד גשם''

זוהי \textbf{קונטרפוזיציה}: $P \then Q \equiv \neg Q \then \neg P$

%====================================
\subsection{סיכום: טבלת הקשרים הלוגיים}
%====================================

\begin{center}
\begin{tabular}{|C|C|C|C|C|C|C|C|}
\hline
\rowcolor{tableheader}\color{white}\textbf{$P$} & \color{white}\textbf{$Q$} & \color{white}\textbf{$\neg P$} & \color{white}\textbf{$P \land Q$} & \color{white}\textbf{$P \lor Q$} & \color{white}\textbf{$P \oplus Q$} & \color{white}\textbf{$P \then Q$} & \color{white}\textbf{$P \iff Q$} \\
\hline
\rowcolor{tablerow1} \truthmark{T} & \truthmark{T} & \truthmark{F} & \truthmark{T} & \truthmark{T} & \truthmark{F} & \truthmark{T} & \truthmark{T} \\
\hline
\rowcolor{tablerow2} \truthmark{T} & \truthmark{F} & \truthmark{F} & \truthmark{F} & \truthmark{T} & \truthmark{T} & \truthmark{F} & \truthmark{F} \\
\hline
\rowcolor{tablerow1} \truthmark{F} & \truthmark{T} & \truthmark{T} & \truthmark{F} & \truthmark{T} & \truthmark{T} & \truthmark{T} & \truthmark{F} \\
\hline
\rowcolor{tablerow2} \truthmark{F} & \truthmark{F} & \truthmark{T} & \truthmark{F} & \truthmark{F} & \truthmark{F} & \truthmark{T} & \truthmark{T} \\
\hline
\end{tabular}
\end{center}

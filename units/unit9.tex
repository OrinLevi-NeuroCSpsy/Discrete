% יחידה 9 - חלוקה, אי תלות בנציג, יחסי סדר
%====================================
\section{יחידה 9: חלוקה, אי תלות בנציג, יחסי סדר}

\subsection{פירוק (חלוקה)}
%====================================

\begin{defbox}
\textbf{הגדרה: פירוק של קבוצה}

\textbf{פירוק} (partition) של קבוצה $A$ הוא קבוצה $\mathcal{F}$ של תתי-קבוצות של $A$ המקיימת:

\begin{enumerate}
    \item \textbf{לא ריקות:} $\forall X \in \mathcal{F}. \; X \neq \emptyset$
    \item \textbf{כיסוי:} $\bigcup \mathcal{F} = A$
    \item \textbf{זרות הדדית:} $\forall X, Y \in \mathcal{F}. \; X \neq Y \Rightarrow X \cap Y = \emptyset$
\end{enumerate}
\end{defbox}

\begin{exbox}
\textbf{דוגמאות לפירוקים:}

\begin{enumerate}
    \item $\{\{1, 3, 5\}, \{2, 4, 7, 8\}, \{6\}\}$ הוא פירוק של $\{1, 2, 3, 4, 5, 6, 7, 8\}$
    \item $\{\Z^-, \{0\}, \Z^+\}$ הוא פירוק של $\Z$
    \item $\{[0], [1], [2]\}$ הוא פירוק של $\Z$ לפי $\equiv_3$
\end{enumerate}
\end{exbox}

\begin{thmbox}
\textbf{משפט: קבוצת מנה היא פירוק}

יהי $E$ יחס שקילות על $A$. אז $A/E$ הוא פירוק של $A$.
\end{thmbox}

\begin{thmbox}
\textbf{משפט: ההתאמה בין יחסי שקילות לפירוקים}

קיימת \textbf{התאמה חח``ע ועל} בין יחסי השקילות על $A$ לבין הפירוקים של $A$:

\begin{itemize}
    \item \textbf{מיחס לפירוק:} $E \mapsto A/E$
    \item \textbf{מפירוק ליחס:} $\mathcal{F} \mapsto E_\mathcal{F}$ כאשר $xE_\mathcal{F}y \iff$ קיים $X \in \mathcal{F}$ כך ש-$x, y \in X$
\end{itemize}
\end{thmbox}

%====================================
\subsection{אי-תלות בנציג}
%====================================

\begin{notebox}
\textbf{בעיית הנציג}

כשמגדירים פעולה או פונקציה על קבוצת המנה, לעתים ההגדרה תלויה בבחירת \textbf{נציג} מכל מחלקה.

צריך להוכיח שהתוצאה \textbf{לא תלויה בבחירת הנציג}.
\end{notebox}

\begin{exbox}
\textbf{דוגמה: חיבור על $\Z_n$}

נגדיר חיבור על $\Z/\equiv_n$:
\[[a] + [b] = [a + b]\]

\textbf{צריך להוכיח:} אם $[a] = [a']$ ו-$[b] = [b']$, אז $[a + b] = [a' + b']$.

\textbf{הוכחה:}
\begin{itemize}
    \item $[a] = [a']$ $\Rightarrow$ $n \mid (a - a')$
    \item $[b] = [b']$ $\Rightarrow$ $n \mid (b - b')$
    \item לכן $n \mid ((a + b) - (a' + b')) = (a - a') + (b - b')$
    \item לכן $[a + b] = [a' + b']$ \checkmark
\end{itemize}
\end{exbox}

\begin{notebox}
\textbf{מבנה הוכחת אי-תלות בנציג}

\begin{enumerate}
    \item נניח $[a] = [a']$ ו-$[b] = [b']$
    \item נתרגם להגדרת היחס: $aEa'$ ו-$bEb'$
    \item נראה ש-$(a \star b) E (a' \star b')$
    \item נסיק $[a \star b] = [a' \star b']$
\end{enumerate}
\end{notebox}

%====================================
\subsection{יחסי סדר}
%====================================

\subsubsection{יחס סדר חלקי}

\begin{defbox}
\textbf{הגדרה: יחס סדר חלקי}

יחס $R$ על קבוצה $A$ נקרא \textbf{יחס סדר חלקי} (partial order) אם הוא מקיים:

\begin{enumerate}
    \item \textbf{רפלקסיביות:} $\forall x \in A. \; xRx$
    \item \textbf{אנטי-סימטריות:} $(xRy \land yRx) \Rightarrow x = y$
    \item \textbf{טרנזיטיביות:} $(xRy \land yRz) \Rightarrow xRz$
\end{enumerate}

מסמנים לעתים $\leq$ או $\preceq$ ליחס סדר חלקי.
\end{defbox}

\begin{exbox}
\textbf{דוגמאות ליחסי סדר חלקי:}

\begin{itemize}
    \item $\leq$ על $\R$
    \item $\subseteq$ על $\mathcal{P}(A)$ (יחס ההכלה)
    \item יחס ההתחלקות $\mid$ על $\N^+$: $a \mid b \iff \exists k \in \N. b = ak$
\end{itemize}
\end{exbox}

\subsubsection{יחס סדר מלא}

\begin{defbox}
\textbf{הגדרה: יחס סדר מלא}

יחס סדר חלקי $R$ על $A$ נקרא \textbf{יחס סדר מלא} (total/linear order) אם בנוסף הוא \textbf{קשר}:

\[\forall x, y \in A. \; xRy \lor yRx\]

כלומר, כל שני איברים ניתנים להשוואה.
\end{defbox}

\subsubsection{איברים מיוחדים}

\begin{defbox}
\textbf{הגדרות: איברים מיוחדים}

יהי $(A, \leq)$ קבוצה סדורה חלקית. עבור $a \in A$:

\begin{itemize}
    \item \textbf{מינימלי:} אין איבר קטן ממנו ממש -- $\neg \exists x \in A. x < a$
    \item \textbf{מקסימלי:} אין איבר גדול ממנו ממש -- $\neg \exists x \in A. a < x$
    \item \textbf{מינימום} (הקטן ביותר): $\forall x \in A. a \leq x$
    \item \textbf{מקסימום} (הגדול ביותר): $\forall x \in A. x \leq a$
\end{itemize}
\end{defbox}

\begin{notebox}
\textbf{הבדל חשוב: מינימלי vs מינימום}

\begin{itemize}
    \item \textbf{מינימום:} אם קיים, הוא \textbf{יחיד}
    \item \textbf{מינימלי:} יכולים להיות \textbf{כמה}, או \textbf{אף אחד}
    \item בסדר \textbf{מלא:} מינימלי = מינימום
\end{itemize}
\end{notebox}

\subsubsection{חסמים}

\begin{defbox}
\textbf{הגדרות: חסמים}

יהי $(A, \leq)$ קבוצה סדורה ו-$B \subseteq A$:

\begin{itemize}
    \item \textbf{חסם עליון} של $B$: איבר $a \in A$ כך ש-$\forall b \in B. b \leq a$
    \item \textbf{חסם תחתון} של $B$: איבר $a \in A$ כך ש-$\forall b \in B. a \leq b$
    \item \textbf{סופרמום:} החסם העליון \textbf{הקטן ביותר} -- $\sup(B)$
    \item \textbf{אינפימום:} החסם התחתון \textbf{הגדול ביותר} -- $\inf(B)$
\end{itemize}
\end{defbox}

%====================================
\subsection{טבלת סיכום: שקילות vs סדר}
%====================================

\begin{center}
\begin{tabular}{|l|c|c|}
\hline
\rowcolor{tableheader}\color{white}\textbf{תכונה} & \color{white}\textbf{יחס שקילות} & \color{white}\textbf{יחס סדר חלקי} \\
\hline
\rowcolor{tablerow1} רפלקסיבי & \checkmark & \checkmark \\
\hline
\rowcolor{tablerow2} סימטרי & \checkmark & $\times$ \\
\hline
\rowcolor{tablerow1} אנטי-סימטרי & $\times$ & \checkmark \\
\hline
\rowcolor{tablerow2} טרנזיטיבי & \checkmark & \checkmark \\
\hline
\end{tabular}
\end{center}



% ===============================
% Language (Hebrew + English)
% ===============================
\usepackage{fontspec}
\usepackage{polyglossia}
\setmainlanguage{hebrew}
\setotherlanguage{english}
\newfontfamily\hebrewfont{DavidCLM-Medium}[
  Path = /Users/orinlevi/Library/Fonts/,
  Extension = .otf,
  BoldFont = DavidCLM-Bold,
  ItalicFont = DavidCLM-MediumItalic,
  BoldItalicFont = DavidCLM-BoldItalic,
]
\newfontfamily\hebrewfonttt{DavidCLM-Medium}[
  Path = /Users/orinlevi/Library/Fonts/,
  Extension = .otf,
]
\newfontfamily\figurelatinfont{Times New Roman}

% ===============================
% Mathematics
% ===============================
\usepackage{amsmath, amssymb, amsthm}
\usepackage{mathtools}

% ===============================
% Page Layout
% ===============================
\usepackage[a4paper,margin=2.5cm]{geometry}

% ===============================
% Lists
% ===============================
\usepackage{enumitem}
\setlist[itemize]{itemsep=0.3em}

% ===============================
% Tables and Colors
% ===============================
\usepackage[table,xcdraw]{xcolor}
\usepackage{longtable}
\usepackage{booktabs}
\usepackage{colortbl}
\usepackage{array}

% סגנון לטבלאות אמת יפות
\newcommand{\truthmark}[1]{\textbf{#1}}
\newcolumntype{C}{>{\centering\arraybackslash}m{1.2cm}}

% צבעים לטבלאות - ורוד פסטלי
\definecolor{tableheader}{RGB}{219,112,147}
\definecolor{tablerow1}{RGB}{255,228,235}
\definecolor{tablerow2}{RGB}{255,245,248}
\definecolor{tableborder}{RGB}{199,92,127}

% הגדרות מסגרת לטבלאות
\setlength{\arrayrulewidth}{1.5pt}
\arrayrulecolor{tableborder}

% ===============================
% Graphics (TikZ + pgfplots)
% ===============================
\usepackage{float}
\usepackage{caption}
\usepackage{pifont}
\usepackage{pgfplots}
\pgfplotsset{compat=1.18}

\usepackage{tikz}
\usetikzlibrary{shapes.geometric, arrows.meta, positioning, calc, decorations.pathreplacing}

% ===============================
% Colored Boxes (mdframed - works with Hebrew)
% ===============================
\usepackage[framemethod=tikz]{mdframed}

% הגדרה - כחול
\newmdenv[
  linecolor=blue!75!black,
  backgroundcolor=blue!5,
  linewidth=2pt,
  roundcorner=5pt,
  innertopmargin=10pt,
  innerbottommargin=10pt,
  innerrightmargin=10pt,
  innerleftmargin=10pt,
  skipabove=12pt,
  skipbelow=12pt,
  nobreak=true
]{defbox}

% משפט - ירוק
\newmdenv[
  linecolor=green!75!black,
  backgroundcolor=green!5,
  linewidth=2pt,
  roundcorner=5pt,
  innertopmargin=10pt,
  innerbottommargin=10pt,
  innerrightmargin=10pt,
  innerleftmargin=10pt,
  skipabove=12pt,
  skipbelow=12pt,
  nobreak=true
]{thmbox}

% דוגמה - כתום
\newmdenv[
  linecolor=orange!75!black,
  backgroundcolor=orange!5,
  linewidth=2pt,
  roundcorner=5pt,
  innertopmargin=10pt,
  innerbottommargin=10pt,
  innerrightmargin=10pt,
  innerleftmargin=10pt,
  skipabove=12pt,
  skipbelow=12pt,
  nobreak=true
]{exbox}

% הערה - צהוב
\newmdenv[
  linecolor=yellow!75!black,
  backgroundcolor=yellow!10,
  linewidth=2pt,
  roundcorner=5pt,
  innertopmargin=10pt,
  innerbottommargin=10pt,
  innerrightmargin=10pt,
  innerleftmargin=10pt,
  skipabove=12pt,
  skipbelow=12pt,
  nobreak=true
]{notebox}

% הוכחה - אפור
\newmdenv[
  linecolor=gray!75!black,
  backgroundcolor=gray!5,
  linewidth=2pt,
  roundcorner=5pt,
  innertopmargin=10pt,
  innerbottommargin=10pt,
  innerrightmargin=10pt,
  innerleftmargin=10pt,
  skipabove=12pt,
  skipbelow=12pt,
  nobreak=true
]{proofbox}

% ===============================
% Theorem Environments
% ===============================
\theoremstyle{definition}
\newtheorem{definition}{הגדרה}[section]
\newtheorem{example}{דוגמה}[section]
\newtheorem{exercise}{תרגיל}[section]

\theoremstyle{plain}
\newtheorem{theorem}{משפט}[section]
\newtheorem{lemma}[theorem]{למה}
\newtheorem{corollary}[theorem]{מסקנה}
\newtheorem{proposition}[theorem]{טענה}

\theoremstyle{remark}
\newtheorem{remark}{הערה}[section]
\newtheorem{note}{הערה}[section]

% ===============================
% Custom Commands - Set Theory
% ===============================
\newcommand{\N}{\mathbb{N}}
\newcommand{\Z}{\mathbb{Z}}
\newcommand{\Q}{\mathbb{Q}}
\newcommand{\R}{\mathbb{R}}
\newcommand{\C}{\mathbb{C}}
\newcommand{\powerset}{\mathcal{P}}
\newcommand{\almark}{\aleph}
\newcommand{\card}[1]{\left|#1\right|}

% Set operations
\newcommand{\union}{\cup}
\newcommand{\intersect}{\cap}
\renewcommand{\setminus}{\smallsetminus}
\newcommand{\symdiff}{\triangle}

% Relations and functions
\newcommand{\dom}{\text{Dom}}
\newcommand{\range}{\text{Range}}
\newcommand{\im}{\text{Im}}
\newcommand{\id}{\text{Id}}

% Logic
\newcommand{\then}{\rightarrow}
\renewcommand{\iff}{\leftrightarrow}
\newcommand{\xmark}{\ding{55}}
\newcommand{\cmark}{\ding{51}}

% ===============================
% Hyperref (must be last)
% ===============================
\usepackage{hyperref}
\hypersetup{
  colorlinks=true,
  linkcolor=blue,
  citecolor=green,
  filecolor=magenta,
  urlcolor=cyan
}

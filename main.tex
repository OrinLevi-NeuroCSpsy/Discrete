% !TeX program = xelatex
% לבניית תוכן עניינים: הרץ xelatex פעמיים (או Recompile ב-IDE).
\documentclass[12pt]{article}


% ===============================
% Language (Hebrew + English)
% ===============================
\usepackage{fontspec}
\usepackage{polyglossia}
\setmainlanguage{hebrew}
\setotherlanguage{english}
\newfontfamily\hebrewfont{DavidCLM-Medium}[
  Path = /Users/orinlevi/Library/Fonts/,
  Extension = .otf,
  BoldFont = DavidCLM-Bold,
  ItalicFont = DavidCLM-MediumItalic,
  BoldItalicFont = DavidCLM-BoldItalic,
]
\newfontfamily\hebrewfonttt{DavidCLM-Medium}[
  Path = /Users/orinlevi/Library/Fonts/,
  Extension = .otf,
]
\newfontfamily\figurelatinfont{Times New Roman}

% ===============================
% Mathematics
% ===============================
\usepackage{amsmath, amssymb, amsthm}
\usepackage{mathtools}

% ===============================
% Page Layout
% ===============================
\usepackage[a4paper,margin=2.5cm]{geometry}

% ===============================
% Lists
% ===============================
\usepackage{enumitem}
\setlist[itemize]{itemsep=0.3em}

% ===============================
% Tables and Colors
% ===============================
\usepackage[table,xcdraw]{xcolor}
\usepackage{longtable}
\usepackage{booktabs}
\usepackage{colortbl}
\usepackage{array}

% סגנון לטבלאות אמת יפות
\newcommand{\truthmark}[1]{\textbf{#1}}
\newcolumntype{C}{>{\centering\arraybackslash}m{1.2cm}}

% צבעים לטבלאות - ורוד פסטלי
\definecolor{tableheader}{RGB}{219,112,147}
\definecolor{tablerow1}{RGB}{255,228,235}
\definecolor{tablerow2}{RGB}{255,245,248}
\definecolor{tableborder}{RGB}{199,92,127}

% הגדרות מסגרת לטבלאות
\setlength{\arrayrulewidth}{1.5pt}
\arrayrulecolor{tableborder}

% ===============================
% Graphics (TikZ + pgfplots)
% ===============================
\usepackage{float}
\usepackage{caption}
\usepackage{pifont}
\usepackage{pgfplots}
\pgfplotsset{compat=1.18}

\usepackage{tikz}
\usetikzlibrary{shapes.geometric, arrows.meta, positioning, calc, decorations.pathreplacing}

% ===============================
% Colored Boxes (mdframed - works with Hebrew)
% ===============================
\usepackage[framemethod=tikz]{mdframed}

% הגדרה - כחול
\newmdenv[
  linecolor=blue!75!black,
  backgroundcolor=blue!5,
  linewidth=2pt,
  roundcorner=5pt,
  innertopmargin=10pt,
  innerbottommargin=10pt,
  innerrightmargin=10pt,
  innerleftmargin=10pt,
  skipabove=12pt,
  skipbelow=12pt,
  nobreak=true
]{defbox}

% משפט - ירוק
\newmdenv[
  linecolor=green!75!black,
  backgroundcolor=green!5,
  linewidth=2pt,
  roundcorner=5pt,
  innertopmargin=10pt,
  innerbottommargin=10pt,
  innerrightmargin=10pt,
  innerleftmargin=10pt,
  skipabove=12pt,
  skipbelow=12pt,
  nobreak=true
]{thmbox}

% דוגמה - כתום
\newmdenv[
  linecolor=orange!75!black,
  backgroundcolor=orange!5,
  linewidth=2pt,
  roundcorner=5pt,
  innertopmargin=10pt,
  innerbottommargin=10pt,
  innerrightmargin=10pt,
  innerleftmargin=10pt,
  skipabove=12pt,
  skipbelow=12pt,
  nobreak=true
]{exbox}

% הערה - צהוב
\newmdenv[
  linecolor=yellow!75!black,
  backgroundcolor=yellow!10,
  linewidth=2pt,
  roundcorner=5pt,
  innertopmargin=10pt,
  innerbottommargin=10pt,
  innerrightmargin=10pt,
  innerleftmargin=10pt,
  skipabove=12pt,
  skipbelow=12pt,
  nobreak=true
]{notebox}

% הוכחה - אפור
\newmdenv[
  linecolor=gray!75!black,
  backgroundcolor=gray!5,
  linewidth=2pt,
  roundcorner=5pt,
  innertopmargin=10pt,
  innerbottommargin=10pt,
  innerrightmargin=10pt,
  innerleftmargin=10pt,
  skipabove=12pt,
  skipbelow=12pt,
  nobreak=true
]{proofbox}

% ===============================
% Theorem Environments
% ===============================
\theoremstyle{definition}
\newtheorem{definition}{הגדרה}[section]
\newtheorem{example}{דוגמה}[section]
\newtheorem{exercise}{תרגיל}[section]

\theoremstyle{plain}
\newtheorem{theorem}{משפט}[section]
\newtheorem{lemma}[theorem]{למה}
\newtheorem{corollary}[theorem]{מסקנה}
\newtheorem{proposition}[theorem]{טענה}

\theoremstyle{remark}
\newtheorem{remark}{הערה}[section]
\newtheorem{note}{הערה}[section]

% ===============================
% Custom Commands - Set Theory
% ===============================
\newcommand{\N}{\mathbb{N}}
\newcommand{\Z}{\mathbb{Z}}
\newcommand{\Q}{\mathbb{Q}}
\newcommand{\R}{\mathbb{R}}
\newcommand{\C}{\mathbb{C}}
\newcommand{\powerset}{\mathcal{P}}
\newcommand{\almark}{\aleph}
\newcommand{\card}[1]{\left|#1\right|}

% Set operations
\newcommand{\union}{\cup}
\newcommand{\intersect}{\cap}
\renewcommand{\setminus}{\smallsetminus}
\newcommand{\symdiff}{\triangle}

% Relations and functions
\newcommand{\dom}{\text{Dom}}
\newcommand{\range}{\text{Range}}
\newcommand{\im}{\text{Im}}
\newcommand{\id}{\text{Id}}

% Logic
\newcommand{\then}{\rightarrow}
\renewcommand{\iff}{\leftrightarrow}
\newcommand{\xmark}{\ding{55}}
\newcommand{\cmark}{\ding{51}}

% ===============================
% Hyperref (must be last)
% ===============================
\usepackage{hyperref}
\hypersetup{
  colorlinks=true,
  linkcolor=blue,
  citecolor=green,
  filecolor=magenta,
  urlcolor=cyan
}

% Additional macros for Discrete Math

% Ordered pair
\newcommand{\pair}[2]{\langle #1, #2 \rangle}

% Equivalence class
\newcommand{\eqclass}[1]{[#1]}

% Quotient set
\newcommand{\quotient}[2]{#1 / #2}

% Restriction
\newcommand{\restrict}[2]{#1|_{#2}}

% Composition
\newcommand{\comp}{\circ}

% Countable infinity
\newcommand{\alef}{\aleph_0}

% Continuum
\newcommand{\cont}{\mathfrak{c}}


\begin{document}

\begin{center}
\begin{english}
{\LARGE \textbf{Discrete Mathematics 1}}\\[0.6cm]
{\Large Orin Levi}\\[0.3cm]
\end{english}
\end{center}

\vspace{0.5cm}

\setcounter{tocdepth}{2}
\tableofcontents
\clearpage

% יחידה 1 - לוגיקה ופסוקים
%====================================
\section{מבוא לקורס}
%====================================

\subsection{מטרת הקורס}
\begin{notebox}
מטרת הקורס היא להכשיר כלים מתמטיים של מדעי המחשב:
\begin{itemize}
    \item שפה מתמטית מדויקת
    \item הוכחות מתמטיות
    \item מושגים בסיסיים (קבוצות, יחסים, פונקציות וכו')
\end{itemize}
\end{notebox}

\subsection{נושאי הקורס}
\begin{enumerate}
    \item קבוצות, יחסים והעתקות (כולל אינדוקציה)
    \item מושג המספר ואלגברה בסיסית
    \item שיטות מנייה והסתברות
    \item קומבינטוריקה ואלגוריתמים
\end{enumerate}

%====================================
\section{היגדים ולוגיקה}
%====================================

\subsection{מוטיבציה}
רוצים להגדיר שפה פורמלית לתיאור טענות מתמטיות. לשם כך נשתמש ב\textbf{לוגיקה} -- תורה העוסקת במבנה הפורמלי של טענות וב\textbf{ערכי אמת} שלהן.

\subsection{פסוק (היגד)}
\begin{defbox}[title=הגדרה: פסוק]
\textbf{פסוק} (או \textbf{היגד}, באנגלית: Proposition) הוא קביעה שניתן להכריע אם היא \textbf{אמת} ($T$ -- True) או \textbf{שקר} ($F$ -- False).
\end{defbox}

\begin{notebox}[title=תכונות של פסוק]
\begin{itemize}
    \item לפסוק יש ערך אמת \textbf{יחיד} -- אמת או שקר, אך לא שניהם.
    \item פסוק מוגדר \textbf{חד-משמעית} -- אין מצב ביניים.
\end{itemize}
\end{notebox}

\subsection{דוגמאות לפסוקים}
\begin{exbox}
\textbf{פסוקים:}
\begin{itemize}
    \item ``$2 > 1$'' -- פסוק (אמת)
    \item ``יורד גשם'' -- פסוק (ערכו תלוי במציאות, אך יש לו ערך אמת מוגדר)
    \item ``$5$ הוא מספר ראשוני'' -- פסוק (אמת)
    \item ``$4$ הוא מספר אי-זוגי'' -- פסוק (שקר)
\end{itemize}

\textbf{לא פסוקים:}
\begin{itemize}
    \item ``$x > 5$'' -- \textbf{לא פסוק!} (תלוי בערך של $x$, זה \textbf{נוסחה פתוחה})
    \item ``האם יורד גשם?'' -- לא פסוק (שאלה)
    \item ``סגור את הדלת'' -- לא פסוק (ציווי)
\end{itemize}
\end{exbox}

\begin{notebox}[title=הערה חשובה]
כשכותבים ``$x$ הוא מספר גדול מ-5'', זהו \textbf{משפט פתוח} (או נוסחה פתוחה) -- ערך האמת שלו תלוי ב-$x$. רק כשנקבע ערך ספציפי ל-$x$, הופך המשפט לפסוק.
\end{notebox}

\subsection{פרדוקסים}
\begin{defbox}[title=פרדוקס]
\textbf{פרדוקס} הוא משפט שלא ניתן לייחס לו ערך אמת -- אם נניח שהוא אמת, נגיע לסתירה, ואם נניח שהוא שקר, גם נגיע לסתירה.
\end{defbox}

\begin{exbox}[title=פרדוקס ברי (Berry Paradox)]
``המספר השלם הקטן ביותר שלא ניתן להגדיר בפחות מ-20 מילים.''

\textbf{הסבר הפרדוקס:}
\begin{itemize}
    \item המשפט עצמו מכיל פחות מ-20 מילים
    \item לכן, המשפט מגדיר מספר בפחות מ-20 מילים
    \item אבל המספר הוגדר כמספר שלא ניתן להגדיר בפחות מ-20 מילים -- סתירה!
\end{itemize}
\end{exbox}

\begin{exbox}[title=פרדוקס השקרן]
``המשפט הזה הוא שקר.''
\begin{itemize}
    \item אם המשפט אמת $\Rightarrow$ הוא אומר על עצמו שהוא שקר $\Rightarrow$ סתירה
    \item אם המשפט שקר $\Rightarrow$ הוא לא שקר $\Rightarrow$ הוא אמת $\Rightarrow$ סתירה
\end{itemize}
\end{exbox}

\begin{notebox}
פרדוקסים אינם פסוקים! הם ממחישים את הצורך בהגדרות מדויקות ובזהירות בבניית טענות לוגיות.
\end{notebox}

%====================================
\section{קשרים לוגיים (קונקטורים)}
%====================================

קשרים לוגיים (או קונקטורים) הם פעולות שמאפשרות לבנות פסוקים מורכבים מפסוקים פשוטים יותר.

\subsection{שלילה (NOT)}
\begin{defbox}[title=שלילה]
אם $P$ הוא פסוק, אז $\neg P$ (``לא $P$'') הוא פסוק ששולל את $P$.

\textbf{סימון:} $\neg P$ או $\lnot P$ או $\overline{P}$
\end{defbox}

\begin{center}
\textbf{טבלת אמת לשלילה:}

\begin{tabular}{|c|c|}
\hline
$P$ & $\neg P$ \\
\hline
$T$ & $F$ \\
$F$ & $T$ \\
\hline
\end{tabular}
\end{center}

\subsection{וגם (AND) -- קוניונקציה}
\begin{defbox}[title=וגם (קוניונקציה)]
אם $P$ ו-$Q$ הם פסוקים, אז $P \land Q$ (``$P$ וגם $Q$'') הוא פסוק שאמיתי \textbf{רק אם} שני הפסוקים אמיתיים.

\textbf{סימון:} $P \land Q$ או $P \cdot Q$ או $P \& Q$
\end{defbox}

\begin{center}
\textbf{טבלת אמת ל-AND:}

\begin{tabular}{|c|c|c|}
\hline
$P$ & $Q$ & $P \land Q$ \\
\hline
$T$ & $T$ & $T$ \\
$T$ & $F$ & $F$ \\
$F$ & $T$ & $F$ \\
$F$ & $F$ & $F$ \\
\hline
\end{tabular}
\end{center}

\subsection{או (OR) -- דיסיונקציה}
\begin{defbox}[title=או (דיסיונקציה)]
אם $P$ ו-$Q$ הם פסוקים, אז $P \lor Q$ (``$P$ או $Q$'') הוא פסוק שאמיתי אם \textbf{לפחות אחד} מהפסוקים אמיתי.

\textbf{סימון:} $P \lor Q$ או $P + Q$

\textbf{שימו לב:} זהו ``או'' \textbf{לא מוציא} (inclusive OR) -- גם אם שניהם אמיתיים, התוצאה אמת.
\end{defbox}

\begin{center}
\textbf{טבלת אמת ל-OR:}

\begin{tabular}{|c|c|c|}
\hline
$P$ & $Q$ & $P \lor Q$ \\
\hline
$T$ & $T$ & $T$ \\
$T$ & $F$ & $T$ \\
$F$ & $T$ & $T$ \\
$F$ & $F$ & $F$ \\
\hline
\end{tabular}
\end{center}

\subsection{או מוציא (XOR)}
\begin{defbox}[title=או מוציא]
$P \oplus Q$ (``$P$ או $Q$ אך לא שניהם'') הוא פסוק שאמיתי אם \textbf{בדיוק אחד} מהפסוקים אמיתי.

\textbf{סימון:} $P \oplus Q$ או $P \veebar Q$
\end{defbox}

\begin{center}
\textbf{טבלת אמת ל-XOR:}

\begin{tabular}{|c|c|c|}
\hline
$P$ & $Q$ & $P \oplus Q$ \\
\hline
$T$ & $T$ & $F$ \\
$T$ & $F$ & $T$ \\
$F$ & $T$ & $T$ \\
$F$ & $F$ & $F$ \\
\hline
\end{tabular}
\end{center}

\subsection{גרירה (אימפליקציה)}
\begin{defbox}[title=גרירה (אימפליקציה)]
אם $P$ ו-$Q$ הם פסוקים, אז $P \then Q$ (``$P$ גורר $Q$'' או ``אם $P$ אז $Q$'') הוא פסוק.

\textbf{סימון:} $P \then Q$ או $P \Rightarrow Q$ או $P \supset Q$

\textbf{פירוש:} ``אם $P$ אמת, אז $Q$ חייב להיות אמת''
\end{defbox}

\begin{center}
\textbf{טבלת אמת לגרירה:}

\begin{tabular}{|c|c|c|}
\hline
$P$ & $Q$ & $P \then Q$ \\
\hline
$T$ & $T$ & $T$ \\
$T$ & $F$ & $F$ \\
$F$ & $T$ & $T$ \\
$F$ & $F$ & $T$ \\
\hline
\end{tabular}
\end{center}

\begin{notebox}[title=הערה חשובה על גרירה]
\begin{itemize}
    \item הגרירה $P \then Q$ היא \textbf{שקר רק} כאשר $P$ אמת ו-$Q$ שקר.
    \item כאשר $P$ שקר, הגרירה \textbf{תמיד אמת}, ללא קשר לערך של $Q$!
    \item זה נקרא ``מהשקר נובע הכל'' (ex falso quodlibet).
    \item דוגמה: ``אם 2+2=5, אז אני נשיא ארה"ב'' -- זהו פסוק אמיתי!
\end{itemize}
\end{notebox}

\begin{exbox}[title=דוגמה]
נניח $P$: ``יורד גשם'', $Q$: ``הכביש רטוב''.

$P \then Q$: ``אם יורד גשם, אז הכביש רטוב''.

\begin{itemize}
    \item אם יורד גשם והכביש רטוב -- הטענה אמת
    \item אם יורד גשם והכביש יבש -- הטענה שקר (הבטחה שנשברה)
    \item אם לא יורד גשם -- הטענה אמת (לא הבטחנו כלום על מקרה זה)
\end{itemize}
\end{exbox}

\subsection{שקילות (אם ורק אם)}
\begin{defbox}[title=שקילות]
$P \iff Q$ (``$P$ אם ורק אם $Q$'') הוא פסוק שאמיתי כאשר ל-$P$ ול-$Q$ יש \textbf{אותו ערך אמת}.

\textbf{סימון:} $P \iff Q$ או $P \Leftrightarrow Q$ או $P \equiv Q$

\textbf{שקול ל:} $(P \then Q) \land (Q \then P)$
\end{defbox}

\begin{center}
\textbf{טבלת אמת לשקילות:}

\begin{tabular}{|c|c|c|}
\hline
$P$ & $Q$ & $P \iff Q$ \\
\hline
$T$ & $T$ & $T$ \\
$T$ & $F$ & $F$ \\
$F$ & $T$ & $F$ \\
$F$ & $F$ & $T$ \\
\hline
\end{tabular}
\end{center}

%====================================
\section{טבלאות אמת}
%====================================

\begin{defbox}[title=טבלת אמת]
\textbf{טבלת אמת} היא טבלה המציגה את ערך האמת של פסוק מורכב עבור כל השילובים האפשריים של ערכי האמת של הפסוקים המרכיבים אותו.
\end{defbox}

\begin{notebox}[title=בניית טבלת אמת]
אם יש $n$ פסוקים אטומיים, יש $2^n$ שורות בטבלת האמת (כל השילובים האפשריים של $T$ ו-$F$).
\end{notebox}

\begin{exbox}[title=דוגמה: טבלת אמת ל-$(P \land Q) \then R$]
יש 3 משתנים, לכן $2^3 = 8$ שורות:

\begin{center}
\begin{tabular}{|c|c|c|c|c|}
\hline
$P$ & $Q$ & $R$ & $P \land Q$ & $(P \land Q) \then R$ \\
\hline
$T$ & $T$ & $T$ & $T$ & $T$ \\
$T$ & $T$ & $F$ & $T$ & $F$ \\
$T$ & $F$ & $T$ & $F$ & $T$ \\
$T$ & $F$ & $F$ & $F$ & $T$ \\
$F$ & $T$ & $T$ & $F$ & $T$ \\
$F$ & $T$ & $F$ & $F$ & $T$ \\
$F$ & $F$ & $T$ & $F$ & $T$ \\
$F$ & $F$ & $F$ & $F$ & $T$ \\
\hline
\end{tabular}
\end{center}
\end{exbox}

%====================================
\section{סיווג פסוקים}
%====================================

\begin{defbox}[title=טאוטולוגיה]
\textbf{טאוטולוגיה} היא פסוק ש\textbf{תמיד אמת}, לכל השמה של ערכי אמת למשתנים.

דוגמה: $P \lor \neg P$ (חוק השלישי הנמנע)
\end{defbox}

\begin{defbox}[title=סתירה]
\textbf{סתירה} (או קונטרדיקציה) היא פסוק ש\textbf{תמיד שקר}, לכל השמה.

דוגמה: $P \land \neg P$
\end{defbox}

\begin{defbox}[title=פסוק ספיק]
\textbf{פסוק ספיק} (או מתקיים, contingent) הוא פסוק שאינו טאוטולוגיה ואינו סתירה -- יש השמות שבהן הוא אמת ויש השמות שבהן הוא שקר.

דוגמה: $P \then Q$
\end{defbox}

\begin{exbox}[title=דוגמה: הוכחה שפסוק הוא טאוטולוגיה]
נוכיח ש-$P \lor \neg P$ הוא טאוטולוגיה:

\begin{center}
\begin{tabular}{|c|c|c|}
\hline
$P$ & $\neg P$ & $P \lor \neg P$ \\
\hline
$T$ & $F$ & $T$ \\
$F$ & $T$ & $T$ \\
\hline
\end{tabular}
\end{center}

בכל השורות התוצאה $T$, לכן זוהי טאוטולוגיה.
\end{exbox}

%====================================
\section{שקילות לוגית}
%====================================

\begin{defbox}[title=שקילות לוגית]
שני פסוקים $P$ ו-$Q$ הם \textbf{שקולים לוגית} (ונסמן $P \equiv Q$) אם יש להם \textbf{אותם ערכי אמת} בכל טבלת האמת.

\textbf{שקול לכך:} $P \iff Q$ היא טאוטולוגיה.
\end{defbox}

\subsection{שקילויות חשובות}

\begin{thmbox}[title=חוקי דה-מורגן]
\begin{align}
\neg(P \land Q) &\equiv \neg P \lor \neg Q \\
\neg(P \lor Q) &\equiv \neg P \land \neg Q
\end{align}
\end{thmbox}

\begin{thmbox}[title=חוקי הכפלה והפיצול]
\begin{align}
P \land (Q \lor R) &\equiv (P \land Q) \lor (P \land R) \quad \text{(חוק הכפלה)} \\
P \lor (Q \land R) &\equiv (P \lor Q) \land (P \lor R) \quad \text{(חוק הפיצול)}
\end{align}
\end{thmbox}

\begin{thmbox}[title=שקילויות נוספות]
\begin{align}
P \then Q &\equiv \neg P \lor Q \\
P \then Q &\equiv \neg Q \then \neg P \quad \text{(קונטרפוזיציה)} \\
P \iff Q &\equiv (P \then Q) \land (Q \then P) \\
\neg(\neg P) &\equiv P \quad \text{(שלילה כפולה)}
\end{align}
\end{thmbox}

\begin{thmbox}[title=חוקי החילוף (קומוטטיביות)]
\begin{align}
P \land Q &\equiv Q \land P \\
P \lor Q &\equiv Q \lor P
\end{align}
\end{thmbox}

\begin{thmbox}[title=חוקי הקיבוץ (אסוציאטיביות)]
\begin{align}
(P \land Q) \land R &\equiv P \land (Q \land R) \\
(P \lor Q) \lor R &\equiv P \lor (Q \lor R)
\end{align}
\end{thmbox}

\begin{thmbox}[title=חוקי הזהות]
\begin{align}
P \land T &\equiv P \\
P \lor F &\equiv P \\
P \land F &\equiv F \\
P \lor T &\equiv T
\end{align}
\end{thmbox}

\begin{thmbox}[title=חוקי האידמפוטנטיות]
\begin{align}
P \land P &\equiv P \\
P \lor P &\equiv P
\end{align}
\end{thmbox}

\begin{thmbox}[title=חוקי הבליעה]
\begin{align}
P \land (P \lor Q) &\equiv P \\
P \lor (P \land Q) &\equiv P
\end{align}
\end{thmbox}

%====================================
\section{דוגמאות ותרגילים}
%====================================

\begin{exbox}[title=דוגמה 1: הוכחת שקילות באמצעות טבלת אמת]
נוכיח ש-$P \then Q \equiv \neg P \lor Q$:

\begin{center}
\begin{tabular}{|c|c|c|c|c|}
\hline
$P$ & $Q$ & $\neg P$ & $P \then Q$ & $\neg P \lor Q$ \\
\hline
$T$ & $T$ & $F$ & $T$ & $T$ \\
$T$ & $F$ & $F$ & $F$ & $F$ \\
$F$ & $T$ & $T$ & $T$ & $T$ \\
$F$ & $F$ & $T$ & $T$ & $T$ \\
\hline
\end{tabular}
\end{center}

העמודות $P \then Q$ ו-$\neg P \lor Q$ זהות, לכן הפסוקים שקולים.
\end{exbox}

\begin{exbox}[title=דוגמה 2: הוכחת שקילות באמצעות חוקים]
נוכיח ש-$\neg(P \then Q) \equiv P \land \neg Q$:

\begin{align*}
\neg(P \then Q) &\equiv \neg(\neg P \lor Q) && \text{(שקילות גרירה)} \\
&\equiv \neg(\neg P) \land \neg Q && \text{(דה-מורגן)} \\
&\equiv P \land \neg Q && \text{(שלילה כפולה)}
\end{align*}
\end{exbox}

\begin{exbox}[title=דוגמה 3: קונטרפוזיציה]
הטענה ``אם יורד גשם אז הכביש רטוב'' שקולה ל:

``אם הכביש לא רטוב אז לא יורד גשם''

זוהי \textbf{קונטרפוזיציה}: $P \then Q \equiv \neg Q \then \neg P$
\end{exbox}

%====================================
\section{סיכום: טבלת הקשרים הלוגיים}
%====================================

\begin{center}
\begin{tabular}{|c|c|c|c|c|c|c|c|}
\hline
$P$ & $Q$ & $\neg P$ & $P \land Q$ & $P \lor Q$ & $P \oplus Q$ & $P \then Q$ & $P \iff Q$ \\
\hline
$T$ & $T$ & $F$ & $T$ & $T$ & $F$ & $T$ & $T$ \\
$T$ & $F$ & $F$ & $F$ & $T$ & $T$ & $F$ & $F$ \\
$F$ & $T$ & $T$ & $F$ & $T$ & $T$ & $T$ & $F$ \\
$F$ & $F$ & $T$ & $F$ & $F$ & $F$ & $T$ & $T$ \\
\hline
\end{tabular}
\end{center}

\clearpage
% יחידה 2 - קבוצות
%====================================
\section{יחידה 2: קבוצות}

\subsection{מושגי יסוד}
%====================================

\subsubsection{מהי קבוצה?}

\begin{defbox}
\textbf{הגדרה: קבוצה}

\textbf{קבוצה} (Set) היא אוסף של עצמים, המהווה עצם בעצמו. לעצמים שמרכיבים קבוצה קוראים \textbf{איברי} הקבוצה, ועל כל אחד מהם אומרים שהוא \textbf{שייך} לקבוצה.
\end{defbox}

\textbf{סימון:} אם $t$ הוא איבר של קבוצה $A$, נכתוב $t \in A$. אם $t$ אינו איבר של $A$, נכתוב $t \notin A$.

\begin{notebox}
\textbf{נקודות חשובות:}
\begin{itemize}
    \item \textbf{אין הגבלה} על מה שיכול לשמש כאיבר בקבוצה -- כל עצם יכול להיות איבר.
    \item קבוצות אינן חייבות להיות \textbf{הומוגניות} -- ניתן לערבב סוגי עצמים שונים.
    \item קבוצות יכולות להיות \textbf{סופיות} או \textbf{אינסופיות}.
    \item קבוצות עצמן נחשבות לעצמים, ולכן \textbf{קבוצה יכולה להיות איבר של קבוצה אחרת}.
\end{itemize}
\end{notebox}

\subsubsection{עקרון האקסטנציונליות}

\begin{thmbox}
\textbf{עקרון האקסטנציונליות}

שתי קבוצות הן \textbf{שוות} אם ורק אם יש להן \textbf{בדיוק אותם איברים}.

בסימון פורמלי:
\[A = B \Leftrightarrow \forall x.(x \in A \leftrightarrow x \in B)\]
\end{thmbox}

\textbf{משמעות:} קבוצה נקבעת אך ורק על-ידי איבריה -- לא על-ידי סדר הצגתם, לא על-ידי האופן שבו תוארה, ולא על-ידי כל מאפיין אחר.

\begin{exbox}
\textbf{דוגמאות:}
\begin{itemize}
    \item $\{1, 2, 3\} = \{3, 1, 2\} = \{1, 1, 2, 3\}$ -- סדר ההצגה וכפילויות לא משנים.
    \item קבוצת המספרים הזוגיים בין 1 ל-5 שווה לקבוצה $\{2, 4\}$.
\end{itemize}
\end{exbox}

%====================================
\subsection{הכלה (תת-קבוצה)}
%====================================

\subsubsection{הגדרה}

\begin{defbox}
\textbf{הגדרה: הכלה}

נאמר ש-$A$ היא \textbf{תת-קבוצה} של $B$ (או ש-$A$ \textbf{חלקית} ל-$B$, או ש-$B$ \textbf{מכילה} את $A$) אם כל איבר של $A$ הוא גם איבר של $B$.

\textbf{סימון:} $A \subseteq B$

בסימון פורמלי:
\[A \subseteq B \Leftrightarrow \forall x.(x \in A \rightarrow x \in B)\]
\end{defbox}

\begin{defbox}
\textbf{הגדרה: הכלה ממש}

נאמר ש-$A$ היא \textbf{תת-קבוצה ממש} של $B$ (או ש-$A$ \textbf{חלקית ממש} ל-$B$) אם $A \subseteq B$ וגם $A \neq B$.

\textbf{סימון:} $A \subsetneq B$ או $A \subset B$
\end{defbox}

\subsubsection{משפטים חשובים}

\begin{thmbox}
\textbf{משפט: קשר בין שוויון להכלה}
\[A = B \Leftrightarrow (A \subseteq B \land B \subseteq A)\]

זוהי הדרך הנפוצה ביותר להוכיח שוויון בין שתי קבוצות: מראים \textbf{הכלה דו-כיוונית}.
\end{thmbox}

\begin{thmbox}
\textbf{משפט: טרנזיטיביות ההכלה}

לכל שלוש קבוצות $A, B, C$:
\[A \subseteq B \land B \subseteq C \Rightarrow A \subseteq C\]
\end{thmbox}

\textbf{הוכחה:} יהי $x \in A$. כיוון ש-$A \subseteq B$ נובע ש-$x \in B$. מהעובדה ש-$B \subseteq C$ נובע ש-$x \in C$. $\blacksquare$

\begin{warnbox}
\textbf{אזהרה: הבדל בין $\in$ ל-$\subseteq$}
\begin{itemize}
    \item \textbf{שייכות} ($\in$): יחס בין \textbf{איבר} לבין \textbf{קבוצה}.
    \item \textbf{הכלה} ($\subseteq$): יחס בין \textbf{קבוצה} לבין \textbf{קבוצה}.
\end{itemize}

לדוגמה: קבוצת המספרים הזוגיים \textbf{חלקית} לקבוצת הטבעיים, אך היא \textbf{לא שייכת} לה (כי היא עצמה אינה מספר טבעי).
\end{warnbox}

%====================================
\subsection{הקבוצה הריקה}
%====================================

\begin{defbox}
\textbf{הגדרה: הקבוצה הריקה}

\textbf{הקבוצה הריקה} (נסמנת $\emptyset$ או $\{\}$) היא הקבוצה שאין בה איברים כלל.

בסימון פורמלי:
\[\forall x. x \notin \emptyset\]
\end{defbox}

\begin{thmbox}
\textbf{משפט: הקבוצה הריקה חלקית לכל קבוצה}

לכל קבוצה $A$ מתקיים $\emptyset \subseteq A$.
\end{thmbox}

\textbf{הוכחה:} צריך להראות ש-$\forall x.(x \in \emptyset \rightarrow x \in A)$. מכיוון שאין איברים ב-$\emptyset$, האגף השמאלי של הגרירה תמיד שקר, ולכן הגרירה כולה תמיד אמת (``מהשקר נובע הכל''). $\blacksquare$

%====================================
\subsection{אקסיומות להגדרת קבוצות}
%====================================

\subsubsection{עקרון הקומפרהנסיה המוגבל}

\begin{thmbox}
\textbf{עקרון הקומפרהנסיה המוגבל}

אם $A$ היא קבוצה ו-$P$ הוא תנאי (פרדיקט), אז הביטוי
\[\{x \in A \mid P(x)\}\]
מגדיר קבוצה. זוהי קבוצת כל האיברים מ-$A$ שמקיימים את התנאי $P$.
\end{thmbox}

\begin{exbox}
\textbf{דוגמה:}

קבוצת המספרים הזוגיים:
\[\mathbb{N}_{\text{even}} = \{n \in \mathbb{N} \mid \exists k \in \mathbb{N}. n = 2k\}\]
\end{exbox}

\subsubsection{אקסיומת ההחלפה}

\begin{thmbox}
\textbf{אקסיומת ההחלפה}

אם $F$ היא פונקציה ו-$A$ היא קבוצה, אז הביטוי
\[\{F(x) \mid x \in A\}\]
מגדיר קבוצה. זוהי קבוצת כל התוצאות של הפעלת $F$ על איברי $A$.
\end{thmbox}

\begin{exbox}
\textbf{דוגמה:}

קבוצת המספרים הזוגיים (דרך נוספת):
\[\mathbb{N}_{\text{even}} = \{2n \mid n \in \mathbb{N}\}\]
\end{exbox}

\begin{notebox}
\textbf{הערה:} שני הסימונים שקולים:
\[\{F(x) \mid x \in A\} = \{y \in B \mid \exists x \in A. F(x) = y\}\]
כאשר $B$ היא קבוצה המכילה את כל הערכים האפשריים של $F$.
\end{notebox}

%====================================
\subsection{פעולות על קבוצות}
%====================================

\subsubsection{פעולות בסיסיות}

\begin{defbox}
\textbf{הגדרות: פעולות על קבוצות}

יהיו $A, B$ קבוצות.

\textbf{איחוד (Union):}
\[A \cup B = \{x \mid x \in A \lor x \in B\}\]

\textbf{חיתוך (Intersection):}
\[A \cap B = \{x \mid x \in A \land x \in B\}\]

\textbf{הפרש (Difference):}
\[A \setminus B = \{x \mid x \in A \land x \notin B\}\]

\textbf{משלים (Complement):} ביחס לקבוצת ``עולם'' $E$:
\[\overline{A} = E \setminus A = \{x \in E \mid x \notin A\}\]

\textbf{הפרש סימטרי (Symmetric Difference):}
\[A \triangle B = (A \setminus B) \cup (B \setminus A)\]
\end{defbox}

\subsubsection{תכונות הפעולות}

\begin{thmbox}
\textbf{תכונות יסודיות של פעולות על קבוצות}

\textbf{קומוטטיביות:}
\[A \cap B = B \cap A \qquad A \cup B = B \cup A\]

\textbf{אסוציאטיביות:}
\[(A \cap B) \cap C = A \cap (B \cap C) \qquad (A \cup B) \cup C = A \cup (B \cup C)\]

\textbf{דיסטריביוטיביות:}
\[A \cap (B \cup C) = (A \cap B) \cup (A \cap C)\]
\[A \cup (B \cap C) = (A \cup B) \cap (A \cup C)\]

\textbf{חוקי דה-מורגן:}
\[\overline{A \cup B} = \overline{A} \cap \overline{B} \qquad \overline{A \cap B} = \overline{A} \cup \overline{B}\]

\textbf{משלים-הפרש:}
\[A \setminus B = A \cap \overline{B}\]
\end{thmbox}

\begin{exbox}
\textbf{דוגמה: הוכחת זהות בין קבוצות}

נוכיח: $A \setminus (B \cap C) = (A \setminus B) \cup (A \setminus C)$

\textbf{שיטה 1: שימוש בזהויות}
\begin{align*}
A \setminus (B \cap C) &= A \cap \overline{B \cap C} && \text{(משלים-הפרש)} \\
&= A \cap (\overline{B} \cup \overline{C}) && \text{(דה-מורגן)} \\
&= (A \cap \overline{B}) \cup (A \cap \overline{C}) && \text{(דיסטריביוטיביות)} \\
&= (A \setminus B) \cup (A \setminus C) && \text{(משלים-הפרש)}
\end{align*}

\textbf{שיטה 2: הכלה דו-כיוונית}

($\subseteq$) יהי $x \in A \setminus (B \cap C)$. אז $x \in A$ ו-$x \notin B \cap C$, כלומר $x \notin B$ או $x \notin C$.
\begin{itemize}
    \item אם $x \notin B$ אז $x \in A \setminus B$.
    \item אם $x \notin C$ אז $x \in A \setminus C$.
\end{itemize}
בכל מקרה $x \in (A \setminus B) \cup (A \setminus C)$.

($\supseteq$) יהי $x \in (A \setminus B) \cup (A \setminus C)$.
\begin{itemize}
    \item אם $x \in A \setminus B$ אז $x \in A$ ו-$x \notin B$, ולכן $x \notin B \cap C$.
    \item אם $x \in A \setminus C$ אז $x \in A$ ו-$x \notin C$, ולכן $x \notin B \cap C$.
\end{itemize}
בכל מקרה $x \in A \setminus (B \cap C)$. $\blacksquare$
\end{exbox}

%====================================
\subsection{קבוצת החזקה}
%====================================

\begin{defbox}
\textbf{הגדרה: קבוצת החזקה}

בהינתן קבוצה $A$, \textbf{קבוצת החזקה} שלה, $\mathcal{P}(A)$, מוגדרת כקבוצת כל תתי-הקבוצות של $A$:
\[\mathcal{P}(A) = \{B \mid B \subseteq A\}\]
\end{defbox}

\begin{notebox}
\textbf{תכונה חשובה:}

לכל קבוצה $B$ מתקיים:
\[B \in \mathcal{P}(A) \Leftrightarrow B \subseteq A\]
\end{notebox}

\begin{exbox}
\textbf{דוגמאות:}
\begin{itemize}
    \item $\mathcal{P}(\emptyset) = \{\emptyset\}$
    \item $\mathcal{P}(\{1\}) = \{\emptyset, \{1\}\}$
    \item $\mathcal{P}(\{1, 2\}) = \{\emptyset, \{1\}, \{2\}, \{1, 2\}\}$
    \item $\mathcal{P}(\mathcal{P}(\{1\})) = \{\emptyset, \{\emptyset\}, \{\{1\}\}, \{\emptyset, \{1\}\}\}$
\end{itemize}
\end{exbox}

\begin{thmbox}
\textbf{משפט:}
\[A \subseteq B \Leftrightarrow \mathcal{P}(A) \subseteq \mathcal{P}(B)\]
\end{thmbox}

\textbf{הוכחה:}

($\Rightarrow$) נניח $A \subseteq B$. תהא $X \in \mathcal{P}(A)$. אז $X \subseteq A$. מטרנזיטיביות ההכלה, $X \subseteq B$, ולכן $X \in \mathcal{P}(B)$.

($\Leftarrow$) נניח $\mathcal{P}(A) \subseteq \mathcal{P}(B)$. יהי $x \in A$. אז $\{x\} \subseteq A$, ולכן $\{x\} \in \mathcal{P}(A)$. מההנחה, $\{x\} \in \mathcal{P}(B)$, ולכן $\{x\} \subseteq B$, ומכאן $x \in B$. $\blacksquare$

%====================================
\subsection{איחוד וחיתוך מוכללים}
%====================================

\subsubsection{הגדרה באמצעות קבוצה של קבוצות}

\begin{defbox}
\textbf{הגדרה: איחוד וחיתוך מוכללים}

תהי $\mathcal{F}$ קבוצה של קבוצות.

\textbf{איחוד מוכלל:}
\[\bigcup \mathcal{F} = \{x \mid \exists A \in \mathcal{F}. x \in A\}\]

\textbf{חיתוך מוכלל} (מוגדר רק אם $\mathcal{F} \neq \emptyset$):
\[\bigcap \mathcal{F} = \{x \mid \forall A \in \mathcal{F}. x \in A\}\]
\end{defbox}

\subsubsection{הגדרה באמצעות קבוצת אינדקסים}

\begin{defbox}
\textbf{הגדרה: סימון אינדקסים}

תהי $I$ קבוצת אינדקסים, ולכל $i \in I$ תהי $A_i$ קבוצה.

\textbf{איחוד:}
\[\bigcup_{i \in I} A_i = \{x \mid \exists i \in I. x \in A_i\}\]

\textbf{חיתוך} (כאשר $I \neq \emptyset$):
\[\bigcap_{i \in I} A_i = \{x \mid \forall i \in I. x \in A_i\}\]
\end{defbox}

\begin{exbox}
\textbf{דוגמאות:}

עבור $\mathcal{X} = \{\emptyset, \{\emptyset\}, \{\{\emptyset\}\}\}$:
\[\bigcup \mathcal{X} = \{\emptyset, \{\emptyset\}\} \qquad \bigcap \mathcal{X} = \emptyset\]

עבור קטעים:
\[\bigcup_{x \in \mathbb{R}^+} (0, x) = \mathbb{R}^+ = (0, \infty) \qquad \bigcap_{x \in \mathbb{R}^+} (0, x) = \emptyset\]
\end{exbox}

\begin{exbox}
\textbf{דוגמה מפורטת:}

נוכיח: $\displaystyle\bigcup_{k \geq 2} \left[\frac{1}{k}, 1 - \frac{1}{k}\right] = (0, 1)$

($\subseteq$) יהי $x$ באיחוד. אז קיים $k \geq 2$ כך ש-$\frac{1}{k} \leq x \leq 1 - \frac{1}{k}$. מכיוון ש-$k \geq 2$, מתקיים $0 < \frac{1}{k} \leq x \leq 1 - \frac{1}{k} < 1$, ולכן $x \in (0, 1)$.

($\supseteq$) יהי $x \in (0, 1)$. נבחר $k = \max\left\{\left\lceil\frac{1}{x}\right\rceil, \left\lceil\frac{1}{1-x}\right\rceil, 2\right\}$. אז $k \geq 2$, וגם $k \geq \frac{1}{x}$ (כלומר $x \geq \frac{1}{k}$) וגם $k \geq \frac{1}{1-x}$ (כלומר $x \leq 1 - \frac{1}{k}$). $\blacksquare$
\end{exbox}

%====================================
\subsection{מלכודות נפוצות}
%====================================

\begin{warnbox}
\textbf{מלכודות נפוצות:}

\textbf{1. בלבול בין $\in$ ל-$\subseteq$:}
\begin{itemize}
    \item $1 \in \{1, 2\}$ \checkmark
    \item $\{1\} \subseteq \{1, 2\}$ \checkmark
    \item $\{1\} \in \{1, 2\}$ \ding{55} (אלא אם $\{1\}$ הוא עצמו איבר)
\end{itemize}

\textbf{2. בלבול בין $\emptyset$ ל-$\{\emptyset\}$:}
\begin{itemize}
    \item $\emptyset$ היא הקבוצה הריקה (0 איברים)
    \item $\{\emptyset\}$ היא קבוצה עם איבר אחד (הקבוצה הריקה)
\end{itemize}

\textbf{3. $\in$ אינה טרנזיטיבית:}
\begin{itemize}
    \item אם $A \in B$ ו-$B \in C$, לא בהכרח $A \in C$.
    \item דוגמה נגדית: $A = \{2\}$, $B = \{\{2\}, 2\}$, $C = \{\{\{2\}, 2\}\}$.
\end{itemize}
\end{warnbox}

%====================================
\subsection{תרגילים לדוגמה}
%====================================

\begin{exbox}
\textbf{תרגיל 1:} אילו מהטענות הבאות נכונות?

\begin{enumerate}
    \item $\emptyset \in \emptyset$ -- \textbf{לא נכון} (אין איברים ב-$\emptyset$)
    \item $\emptyset \subseteq \emptyset$ -- \textbf{נכון} (הריקה חלקית לכל קבוצה)
    \item $\emptyset \in \{\emptyset\}$ -- \textbf{נכון} (הריקה היא איבר)
    \item $\emptyset \subseteq \{\emptyset\}$ -- \textbf{נכון} (הריקה חלקית לכל קבוצה)
    \item $\{2, 3\} \subseteq \{1, 2, \{2, 3\}, 4\}$ -- \textbf{לא נכון} ($3$ לא איבר בקבוצה הימנית)
    \item $\{2, 3\} \in \{1, 2, \{2, 3\}, 4\}$ -- \textbf{נכון} ($\{2,3\}$ הוא איבר)
\end{enumerate}
\end{exbox}

\begin{exbox}
\textbf{תרגיל 2:} מצאו תנאי הכרחי ומספיק לכך ש-$\mathcal{P}(A) \cup \mathcal{P}(B) = \mathcal{P}(A \cup B)$.

\textbf{תשובה:} $A \subseteq B$ או $B \subseteq A$.

\textbf{הוכחת המספיקות:} נניח $A \subseteq B$. אז $A \cup B = B$, ו-$\mathcal{P}(A) \subseteq \mathcal{P}(B)$. לכן $\mathcal{P}(A) \cup \mathcal{P}(B) = \mathcal{P}(B) = \mathcal{P}(A \cup B)$.

\textbf{הוכחת ההכרחיות:} נניח ששתי הקבוצות לא מוכלות זו בזו. אז קיימים $a \in A \setminus B$ ו-$b \in B \setminus A$. הקבוצה $\{a, b\}$ שייכת ל-$\mathcal{P}(A \cup B)$ אך לא ל-$\mathcal{P}(A)$ ולא ל-$\mathcal{P}(B)$, ולכן לא לאיחוד שלהן.
\end{exbox}

%====================================
\subsection{מקורות}
%====================================

\begin{itemize}
    \item \texttt{discrete1-the\_course\_book.txt} -- פרק ב.1 (מושגי יסוד), פרק ב.2 (הגדרת קבוצות וסימונן), פרק ב.3 (פעולות על קבוצות)
    \item \texttt{discrete1-recitations\_1-13.txt} -- תרגולים 2 ו-3
    \item \texttt{discrete\_mathematics1-extended\_formula\_sheet.txt}
\end{itemize}
\clearpage
\input{units/unit3}\clearpage
% יחידה 4 - יחסים
%====================================
\section{יחידה 4: יחסים}

\subsection{זוגות סדורים ומכפלה קרטזית}
%====================================

\begin{defbox}
\textbf{הגדרה: זוג סדור}

\textbf{זוג סדור} $\langle a, b \rangle$ הוא אובייקט מתמטי שבו יש משמעות לסדר.

\textbf{תכונה:} $\langle a, b \rangle = \langle c, d \rangle$ אם ורק אם $a = c$ ו-$b = d$.
\end{defbox}

\begin{notebox}
\textbf{הבדל מקבוצה:}
\begin{itemize}
    \item בקבוצה $\{a, b\} = \{b, a\}$ -- אין משמעות לסדר
    \item בזוג סדור $\langle a, b \rangle \neq \langle b, a \rangle$ (אלא אם $a = b$)
\end{itemize}
\end{notebox}

\begin{defbox}
\textbf{הגדרה: מכפלה קרטזית}

\textbf{מכפלה קרטזית} של קבוצות $A$ ו-$B$:
\[
A \times B = \{\langle a, b \rangle \mid a \in A \land b \in B\}
\]
\end{defbox}

\begin{exbox}
\textbf{דוגמה:}

אם $A = \{1, 2\}$ ו-$B = \{x, y\}$:
\[
A \times B = \{\langle 1, x \rangle, \langle 1, y \rangle, \langle 2, x \rangle, \langle 2, y \rangle\}
\]

\textbf{שימו לב:} $|A \times B| = |A| \cdot |B| = 2 \cdot 2 = 4$
\end{exbox}

%====================================
\subsection{יחסים בינאריים}
%====================================

\begin{defbox}
\textbf{הגדרה: יחס בינארי}

\textbf{יחס בינארי} מ-$A$ ל-$B$ הוא תת-קבוצה $R \subseteq A \times B$.

אם $\langle a, b \rangle \in R$, נסמן גם $aRb$ ונאמר ``$a$ ביחס $R$ ל-$b$''.

\textbf{יחס על קבוצה:} יחס מ-$A$ ל-$A$, כלומר $R \subseteq A \times A$.
\end{defbox}

\begin{exbox}
\textbf{דוגמאות ליחסים:}

\begin{enumerate}
    \item יחס ``קטן מ-'' על $\N$: $R_< = \{\langle n, m \rangle \mid n < m\}$
    \item יחס ``מחלק'' על $\N$: $R_| = \{\langle n, m \rangle \mid n | m\}$
    \item יחס הזהות: $I_A = \{\langle a, a \rangle \mid a \in A\}$
\end{enumerate}
\end{exbox}

%====================================
\subsection{תכונות של יחסים}
%====================================

\begin{defbox}
\textbf{הגדרה: תכונות יחסים}

יהי $R$ יחס על קבוצה $A$:

\begin{enumerate}
    \item \textbf{רפלקסיבי:} $\forall a \in A. \langle a, a \rangle \in R$
    \item \textbf{סימטרי:} $\forall a, b \in A. aRb \then bRa$
    \item \textbf{אנטי-סימטרי:} $\forall a, b \in A. (aRb \land bRa) \then a = b$
    \item \textbf{טרנזיטיבי:} $\forall a, b, c \in A. (aRb \land bRc) \then aRc$
\end{enumerate}
\end{defbox}

\begin{exbox}
\textbf{דוגמאות:}

\begin{center}
\begin{tabular}{|l|c|c|c|c|}
\hline
\rowcolor{tableheader}\color{white}\textbf{יחס} & \color{white}\textbf{רפלקסיבי} & \color{white}\textbf{סימטרי} & \color{white}\textbf{א-סימטרי} & \color{white}\textbf{טרנזיטיבי} \\
\hline
\rowcolor{tablerow1} $=$ על $\N$ & \cmark & \cmark & \cmark & \cmark \\
\hline
\rowcolor{tablerow2} $<$ על $\N$ & \xmark & \xmark & \cmark & \cmark \\
\hline
\rowcolor{tablerow1} $\leq$ על $\N$ & \cmark & \xmark & \cmark & \cmark \\
\hline
\rowcolor{tablerow2} $|$ (מחלק) על $\N^+$ & \cmark & \xmark & \cmark & \cmark \\
\hline
\end{tabular}
\end{center}
\end{exbox}

%====================================
\subsection{הרכבת יחסים}
%====================================

\begin{defbox}
\textbf{הגדרה: הרכבת יחסים}

יהיו $R \subseteq A \times B$ ו-$S \subseteq B \times C$ יחסים.

\textbf{ההרכבה} $S \circ R$ מוגדרת:
\[
S \circ R = \{\langle a, c \rangle \mid \exists b \in B. (aRb \land bSc)\}
\]
\end{defbox}

\begin{notebox}
\textbf{שימו לב לסדר!}

ב-$S \circ R$: קודם מפעילים את $R$, אח"כ את $S$.

זה דומה להרכבת פונקציות: $(g \circ f)(x) = g(f(x))$.
\end{notebox}

\begin{defbox}
\textbf{הגדרה: חזקות של יחס}

יהי $R$ יחס על $A$:
\begin{align*}
R^{(0)} &= I_A \\
R^{(n+1)} &= R^{(n)} \circ R
\end{align*}
\end{defbox}

%====================================
\subsection{יחס הופכי}
%====================================

\begin{defbox}
\textbf{הגדרה: יחס הופכי}

\textbf{היחס ההופכי} של $R \subseteq A \times B$:
\[
R^{-1} = \{\langle b, a \rangle \mid \langle a, b \rangle \in R\}
\]
\end{defbox}

\begin{thmbox}
\textbf{משפט: תכונות היחס ההופכי}

\begin{enumerate}
    \item $(R^{-1})^{-1} = R$
    \item $(S \circ R)^{-1} = R^{-1} \circ S^{-1}$
    \item $R$ סימטרי אמ"מ $R = R^{-1}$
\end{enumerate}
\end{thmbox}

%====================================
\subsection{סיכום: סוגי יחסים}
%====================================

\begin{center}
\begin{tabular}{|l|l|}
\hline
\rowcolor{tableheader}\color{white}\textbf{סוג יחס} & \color{white}\textbf{תנאים} \\
\hline
\rowcolor{tablerow1} \textbf{יחס שקילות} & רפלקסיבי + סימטרי + טרנזיטיבי \\
\hline
\rowcolor{tablerow2} \textbf{יחס סדר חלקי} & רפלקסיבי + אנטי-סימטרי + טרנזיטיבי \\
\hline
\rowcolor{tablerow1} \textbf{יחס סדר מלא} & יחס סדר חלקי + קשר \\
\hline
\end{tabular}
\end{center}
\clearpage
% יחידה 5 - פונקציות (מושגי יסוד)
%====================================
\section{יחידה 5: פונקציות (מושגי יסוד)}

\subsection{מושגי יסוד}
%====================================

\subsubsection{מהי פונקציה?}

\begin{defbox}
\textbf{הגדרה: פונקציה}

\textbf{פונקציה} $f$ מקבוצה $A$ לקבוצה $B$, מסומנת $f: A \to B$, היא יחס $f \subseteq A \times B$ המקיים:
\begin{enumerate}
    \item \textbf{מלאות}: לכל $a \in A$ קיים $b \in B$ כך ש-$\langle a, b \rangle \in f$
    \item \textbf{חד-ערכיות}: אם $\langle a, b_1 \rangle \in f$ וגם $\langle a, b_2 \rangle \in f$, אז $b_1 = b_2$
\end{enumerate}

בקיצור: לכל איבר ב-$A$ מתאימה בדיוק תמונה אחת ב-$B$.
\[\forall a \in A. \exists! b \in B. \langle a, b \rangle \in f\]
\end{defbox}

\begin{notebox}
\textbf{סימונים:}
\begin{itemize}
    \item $f(a) = b$ או $f: a \mapsto b$ -- הפונקציה $f$ מעתיקה את $a$ ל-$b$
    \item $A \to B$ -- קבוצת כל הפונקציות מ-$A$ ל-$B$
\end{itemize}
\end{notebox}

\subsubsection{תחום, טווח ותמונה}

\begin{defbox}
\textbf{הגדרות: תחום, טווח ותמונה}

עבור פונקציה $f: A \to B$:
\begin{itemize}
    \item \textbf{התחום} (Domain): $\Dom(f) = A$
    \item \textbf{הטווח} (Codomain): $B$ -- הקבוצה אליה הפונקציה מעתיקה
    \item \textbf{התמונה} (Image/Range): $\Img(f) = \{f(x) \mid x \in A\} = \{y \in B \mid \exists x \in A. f(x) = y\}$
\end{itemize}
\end{defbox}

\begin{notebox}
\textbf{הבחנה חשובה:}
\begin{itemize}
    \item \textbf{הטווח} $B$ הוא הקבוצה אליה הפונקציה \textbf{יכולה} להעתיק
    \item \textbf{התמונה} $\Img(f)$ היא קבוצת הערכים שהפונקציה \textbf{באמת} מקבלת
\end{itemize}
תמיד מתקיים: $\Img(f) \subseteq B$
\end{notebox}

\begin{exbox}
\textbf{דוגמה:}

עבור $f: \R \to \R$ המוגדרת $f(x) = x^2$:
\begin{itemize}
    \item $\Dom(f) = \R$
    \item הטווח הוא $\R$
    \item $\Img(f) = [0, \infty) = \{y \in \R \mid y \geq 0\}$
\end{itemize}
\end{exbox}

%====================================
\subsection{סימון למבדא}
%====================================

\begin{defbox}
\textbf{סימון למבדא}

\textbf{סימון למבדא} (Lambda notation) מאפשר להגדיר פונקציה בצורה קומפקטית:
\[\lambda x \in A. t(x)\]
מתאר את הפונקציה מ-$A$ שמתאימה לכל $x$ את הערך $t(x)$.
\end{defbox}

\begin{exbox}
\textbf{דוגמאות:}
\begin{itemize}
    \item $\lambda x \in \R. x^2$ -- הפונקציה שמעלה כל מספר ממשי בריבוע
    \item $\lambda n \in \N. 2n + 1$ -- הפונקציה שמתאימה לכל מספר טבעי את המספר האי-זוגי המתאים
    \item $\lambda x \in \R. \lambda y \in \R. x + y$ -- פונקציה של שני משתנים
\end{itemize}
\end{exbox}

\subsubsection{כללי תחשיב למבדא}

\begin{defbox}
\textbf{כלל $\alpha$ (החלפת משתנה)}

ניתן להחליף את שם המשתנה הקשור:
\[\lambda x \in A. t = \lambda y \in A. t[y/x]\]
בתנאי ש-$y$ לא מופיע חופשי ב-$t$.
\end{defbox}

\begin{defbox}
\textbf{כלל $\beta$ (הצבה)}

אם $s$ חופשי להצבה במקום $x$ ב-$t$:
\[(\lambda x \in A. t)(s) = t[s/x]\]
זהו כלל ``החישוב'' -- הצבת ארגומנט בפונקציה.
\end{defbox}

\begin{defbox}
\textbf{כלל $\eta$ (אקסטנציונליות)}
\[\lambda x \in \Dom(f). f(x) = f\]
\end{defbox}

\begin{exbox}
\textbf{דוגמה לשימוש בכללים:}
\[(\lambda x \in \R. x^2)(3) = 3^2 = 9\]
\[(\lambda x \in \R. x + 1) \circ (\lambda x \in \R. x^2) = \lambda x \in \R. x^2 + 1\]
\end{exbox}

%====================================
\subsection{תמונה ומקור של קבוצות}
%====================================

\begin{defbox}
\textbf{הגדרה: תמונה של קבוצה}

תהי $f: A \to B$ ו-$X \subseteq A$. \textbf{התמונה של $X$ תחת $f$} היא:
\[f[X] = \{f(x) \mid x \in X\} = \{y \in B \mid \exists x \in X. f(x) = y\}\]
\end{defbox}

\begin{defbox}
\textbf{הגדרה: מקור של קבוצה}

תהי $f: A \to B$ ו-$Y \subseteq B$. \textbf{המקור של $Y$ לפי $f$} הוא:
\[f^{-1}[Y] = \{x \in A \mid f(x) \in Y\}\]
\end{defbox}

\begin{exbox}
\textbf{דוגמה:}

עבור $f(x) = x^2$ על $\R$:
\begin{itemize}
    \item $f[(-\infty, 3)] = [0, 9]$
    \item $f[\{-2, -1, 0, 1, 2\}] = \{0, 1, 4\}$
    \item $f^{-1}[\{9\}] = \{-3, 3\}$
    \item $f^{-1}[[0, 1]] = [-1, 1]$
    \item $f^{-1}[\{-1\}] = \emptyset$ (אין מספר ממשי שריבועו שלילי)
\end{itemize}
\end{exbox}

%====================================
\subsection{הרכבת פונקציות}
%====================================

\begin{defbox}
\textbf{הגדרה: הרכבת פונקציות}

תהיינה $f: A \to B$ ו-$g: B \to C$. \textbf{ההרכבה} $g \circ f$ היא פונקציה מ-$A$ ל-$C$ המוגדרת:
\[(g \circ f)(x) = g(f(x))\]

או בסימון למבדא:
\[g \circ f = \lambda x \in A. g(f(x))\]
\end{defbox}

\begin{notebox}
\textbf{סדר ההרכבה}

ב-$g \circ f$:
\begin{itemize}
    \item \textbf{קודם} מפעילים את $f$
    \item \textbf{אחר כך} מפעילים את $g$
\end{itemize}
זה הפוך מסדר הכתיבה!
\end{notebox}

\begin{exbox}
\textbf{דוגמה:}

$f(x) = x^2$, $g(x) = x + 1$ על $\R$:
\begin{itemize}
    \item $(g \circ f)(x) = g(f(x)) = g(x^2) = x^2 + 1$
    \item $(f \circ g)(x) = f(g(x)) = f(x+1) = (x+1)^2$
\end{itemize}
שימו לב: $g \circ f \neq f \circ g$
\end{exbox}

\begin{thmbox}
\textbf{משפט: תכונות הרכבת פונקציות}
\begin{enumerate}
    \item \textbf{אסוציאטיביות}: $(h \circ g) \circ f = h \circ (g \circ f)$
    \item \textbf{פונקציית הזהות}: $f \circ i_A = f = i_B \circ f$ עבור $f: A \to B$, כאשר $i_A = \lambda x \in A. x$
    \item \textbf{אי-קומוטטיביות}: בדרך כלל $f \circ g \neq g \circ f$
\end{enumerate}
\end{thmbox}

% המשך החומר על פונקציות מופיע ביחידות 6 ו-7

\clearpage
% יחידה 6 - יחסי שקילות וחלוקות
%====================================
\section{יחידה 6: יחסי שקילות וחלוקות}

\subsection{יחסי שקילות}
%====================================

\begin{defbox}
\textbf{הגדרה: יחס שקילות}

יחס $E$ על קבוצה $A$ נקרא \textbf{יחס שקילות} (equivalence relation) אם הוא מקיים שלושה תנאים:

\begin{enumerate}
    \item \textbf{רפלקסיביות:} $\forall x \in A. \; xEx$
    \item \textbf{סימטריות:} $\forall x, y \in A. \; xEy \Rightarrow yEx$
    \item \textbf{טרנזיטיביות:} $\forall x, y, z \in A. \; (xEy \land yEz) \Rightarrow xEz$
\end{enumerate}
\end{defbox}

\begin{exbox}
\textbf{דוגמאות ליחסי שקילות:}

\begin{enumerate}
    \item \textbf{שוויון} $=$ על כל קבוצה

    \item \textbf{דמיון משולשים} על קבוצת המשולשים

    \item \textbf{חפיפת משולשים} על קבוצת המשולשים

    \item \textbf{שקילות מודולו $n$:}
    \[a \equiv b \pmod{n} \iff n \mid (a - b)\]

    לדוגמה: $5 \equiv 11 \pmod{2}$ כי $2 \mid (11 - 5) = 6$

    \item \textbf{יחס ``באותו צבע''} על קבוצת כדורים צבעוניים
\end{enumerate}
\end{exbox}

%====================================
\subsection{מחלקות שקילות}
%====================================

\begin{defbox}
\textbf{הגדרה: מחלקת שקילות}

יהי $E$ יחס שקילות על $A$. עבור $x \in A$, \textbf{מחלקת השקילות} של $x$ לפי $E$ היא:

\[[x]_E = \{y \in A \mid xEy\}\]

כלומר, קבוצת כל האיברים השקולים ל-$x$.
\end{defbox}

\begin{notebox}
\textbf{סימונים נפוצים למחלקת שקילות:}
\begin{itemize}
    \item $[x]_E$ או $[x]$ (כשהיחס ברור מההקשר)
    \item $\overline{x}$
    \item $x/E$
\end{itemize}
\end{notebox}

\subsubsection{תכונות מחלקות שקילות}

\begin{thmbox}
\textbf{תכונות מחלקות שקילות}

יהי $E$ יחס שקילות על $A$. לכל $x, y \in A$:

\begin{enumerate}
    \item \textbf{כל איבר שייך למחלקה שלו:} $x \in [x]_E$ (מרפלקסיביות)

    \item \textbf{שקילות היא שייכות למחלקה:} $y \in [x]_E \iff xEy$

    \item \textbf{סימטריות:} $y \in [x]_E \iff x \in [y]_E$

    \item \textbf{מחלקות שוות או זרות:}
    \[[x]_E \cap [y]_E \neq \emptyset \iff [x]_E = [y]_E \iff xEy\]
\end{enumerate}
\end{thmbox}

\begin{exbox}
\textbf{דוגמה: מחלקות מודולו 3}

יחס השקילות מודולו 3 על $\Z$:

\begin{itemize}
    \item $[0]_{\equiv_3} = \{\ldots, -6, -3, 0, 3, 6, 9, \ldots\} = 3\Z$
    \item $[1]_{\equiv_3} = \{\ldots, -5, -2, 1, 4, 7, 10, \ldots\} = 3\Z + 1$
    \item $[2]_{\equiv_3} = \{\ldots, -4, -1, 2, 5, 8, 11, \ldots\} = 3\Z + 2$
\end{itemize}

שלוש מחלקות זרות שמכסות את כל $\Z$.
\end{exbox}

%====================================
\subsection{קבוצת מנה}
%====================================

\begin{defbox}
\textbf{הגדרה: קבוצת מנה}

יהי $E$ יחס שקילות על $A$. \textbf{קבוצת המנה} (quotient set) של $A$ לפי $E$ היא:

\[A/E = \{[x]_E \mid x \in A\}\]

כלומר, קבוצת כל מחלקות השקילות.
\end{defbox}

\begin{exbox}
\textbf{דוגמאות לקבוצות מנה:}

\begin{enumerate}
    \item $\Z/\equiv_2 = \{[0], [1]\}$ -- שתי מחלקות (זוגיים ואי-זוגיים)

    \item $\Z/\equiv_3 = \{[0], [1], [2]\}$ -- שלוש מחלקות

    \item $\R/\equiv$ כאשר $x \equiv y \iff |x| = |y|$:

    קבוצת המנה היא למעשה $\R_{\geq 0}$
\end{enumerate}
\end{exbox}

%====================================
\subsection{פירוק (חלוקה)}
%====================================

\begin{defbox}
\textbf{הגדרה: פירוק של קבוצה}

\textbf{פירוק} (partition) של קבוצה $A$ הוא קבוצה $F$ של תתי-קבוצות של $A$ המקיימת:

\begin{enumerate}
    \item \textbf{לא ריקות:} $\forall X \in F. \; X \neq \emptyset$
    \item \textbf{כיסוי:} $\bigcup F = A$
    \item \textbf{זרות הדדית:} $\forall X, Y \in F. \; X \neq Y \Rightarrow X \cap Y = \emptyset$
\end{enumerate}
\end{defbox}

\begin{exbox}
\textbf{דוגמאות לפירוקים:}

\begin{enumerate}
    \item $\{\{1, 3, 5\}, \{2, 4, 7, 8\}, \{6\}\}$ הוא פירוק של $\{1, 2, 3, 4, 5, 6, 7, 8\}$

    \item $\{\Z^-, \{0\}, \Z^+\}$ הוא פירוק של $\Z$
\end{enumerate}
\end{exbox}

\subsubsection{הקשר בין יחסי שקילות לפירוקים}

\begin{thmbox}
\textbf{משפט: קבוצת מנה היא פירוק}

יהי $E$ יחס שקילות על $A$. אז $A/E$ הוא פירוק של $A$.
\end{thmbox}

\begin{proofbox}
\textbf{הוכחה (רעיון):}

\begin{enumerate}
    \item \textbf{לא ריקות:} לכל $x \in A$, $x \in [x]_E$ (מרפלקסיביות), לכן $[x]_E \neq \emptyset$

    \item \textbf{כיסוי:} לכל $x \in A$, $x \in [x]_E \in A/E$, לכן $\bigcup A/E = A$

    \item \textbf{זרות:} מחלקות שונות זרות (הוכחנו קודם)
\end{enumerate}
\end{proofbox}

\begin{thmbox}
\textbf{משפט: ההתאמה בין יחסי שקילות לפירוקים}

קיימת \textbf{התאמה חח``ע ועל} בין יחסי השקילות על $A$ לבין הפירוקים של $A$:

\begin{itemize}
    \item \textbf{מיחס לפירוק:} $E \mapsto A/E$
    \item \textbf{מפירוק ליחס:} $F \mapsto E_F$ כאשר $xE_Fy \iff$ קיים $X \in F$ כך ש-$x, y \in X$
\end{itemize}
\end{thmbox}

%====================================
\subsection{אי-תלות בנציג}
%====================================

\begin{notebox}
\textbf{בעיית הנציג}

כשמגדירים פעולה או פונקציה על קבוצת המנה, לעתים ההגדרה תלויה בבחירת \textbf{נציג} מכל מחלקה.

צריך להוכיח שהתוצאה \textbf{לא תלויה בבחירת הנציג}.
\end{notebox}

\begin{exbox}
\textbf{דוגמה: חיבור על $\Z_n$}

נגדיר חיבור על $\Z/\equiv_n$:
\[[a] + [b] = [a + b]\]

\textbf{צריך להוכיח:} אם $[a] = [a']$ ו-$[b] = [b']$, אז $[a + b] = [a' + b']$.

\textbf{הוכחה:}
\begin{itemize}
    \item $[a] = [a']$ $\Rightarrow$ $n \mid (a - a')$
    \item $[b] = [b']$ $\Rightarrow$ $n \mid (b - b')$
    \item לכן $n \mid ((a + b) - (a' + b')) = (a - a') + (b - b')$
    \item לכן $[a + b] = [a' + b']$ \checkmark
\end{itemize}
\end{exbox}

\begin{notebox}
\textbf{מבנה הוכחת אי-תלות בנציג:}

\begin{enumerate}
    \item נניח $[a] = [a']$ ו-$[b] = [b']$
    \item נתרגם להגדרת היחס: $aEa'$ ו-$bEb'$
    \item נראה ש-$(a \star b) E (a' \star b')$
    \item נסיק $[a \star b] = [a' \star b']$
\end{enumerate}
\end{notebox}

%====================================
\subsection{הפונקציה הקנונית}
%====================================

\begin{defbox}
\textbf{הגדרה: הפונקציה הקנונית}

יהי $E$ יחס שקילות על $A$. \textbf{הפונקציה הקנונית} (או פונקציית המנה) היא:
\[\pi: A \to A/E, \quad \pi(x) = [x]_E\]
\end{defbox}

\begin{thmbox}
\textbf{תכונות הפונקציה הקנונית:}

\begin{enumerate}
    \item $\pi$ היא \textbf{על} (surjective)
    \item $\pi(x) = \pi(y) \iff xEy$
    \item $\text{Im}(\pi) = A/E$
\end{enumerate}
\end{thmbox}

%====================================
\subsection{טבלת סיכום}
%====================================

\begin{center}
\begin{tabular}{|l|l|l|}
\hline
\rowcolor{tableheader}\color{white}\textbf{מושג} & \color{white}\textbf{הגדרה} & \color{white}\textbf{דוגמה} \\
\hline
\rowcolor{tablerow1} \textbf{יחס שקילות} & רפלקסיבי + סימטרי + טרנזיטיבי & $\equiv_n$, $=$ \\
\hline
\rowcolor{tablerow2} \textbf{מחלקת שקילות} & $[x]_E = \{y \mid xEy\}$ & $[0]_{\equiv_3} = 3\Z$ \\
\hline
\rowcolor{tablerow1} \textbf{קבוצת מנה} & $A/E = \{[x]_E \mid x \in A\}$ & $\Z/\equiv_3$ \\
\hline
\rowcolor{tablerow2} \textbf{פירוק} & קבוצות זרות שמכסות את $A$ & $\{[0], [1], [2]\}$ \\
\hline
\end{tabular}
\end{center}

%====================================
\subsection{שגיאות נפוצות}
%====================================

\begin{notebox}
\textbf{שגיאה 1: בלבול בין מחלקה לאיבר}

\begin{itemize}
    \item $[a]$ היא \textbf{קבוצה} (מחלקת השקילות)
    \item $a$ הוא \textbf{איבר} (נציג של המחלקה)
\end{itemize}
\end{notebox}

\begin{notebox}
\textbf{שגיאה 2: הנחה שכל נציג שקול}

כאשר מוכיחים אי-תלות בנציג, צריך להוכיח עבור \textbf{כל} שני נציגים של אותה מחלקה, לא רק עבור נציגים ספציפיים.
\end{notebox}

\begin{notebox}
\textbf{שגיאה 3: שכחת לבדוק את שלושת התנאים}

יחס שקילות חייב לקיים את \textbf{שלושת} התנאים: רפלקסיביות, סימטריה, טרנזיטיביות.
\end{notebox}

%====================================
\subsection{תרגילים לתרגול}
%====================================

\begin{exbox}
\textbf{תרגיל 1:}
הוכיחו שהיחס $R$ על $\Z$ המוגדר ע``י $xRy \iff x - y$ זוגי הוא יחס שקילות. מהן מחלקות השקילות?
\end{exbox}

\begin{exbox}
\textbf{תרגיל 2:}
יהי $f: A \to B$ פונקציה. הגדירו על $A$ את היחס: $xEy \iff f(x) = f(y)$.
הוכיחו ש-$E$ יחס שקילות ותארו את מחלקות השקילות.
\end{exbox}

\begin{exbox}
\textbf{תרגיל 3:}
הוכיחו שהכפל על $\Z_n$ מוגדר היטב (אי-תלות בנציג).
\end{exbox}

\clearpage
% יחידה 7 - יחסי סדר
%====================================
\section{יחידה 7: יחסי סדר}

\subsection{יחס סדר חלקי}
%====================================

\begin{defbox}
\textbf{הגדרה: יחס סדר חלקי}

יחס $R$ על קבוצה $A$ נקרא \textbf{יחס סדר חלקי} (partial order) אם הוא מקיים:

\begin{enumerate}
    \item \textbf{רפלקסיביות:} $\forall x \in A. \; xRx$
    \item \textbf{אנטי-סימטריות:} $(xRy \land yRx) \Rightarrow x = y$
    \item \textbf{טרנזיטיביות:} $(xRy \land yRz) \Rightarrow xRz$
\end{enumerate}

מסמנים לעתים $\leq$ או $\preceq$ ליחס סדר חלקי.
\end{defbox}

\begin{notebox}
\textbf{סימון:} הזוג $(A, \leq)$ נקרא \textbf{קבוצה סדורה חלקית} (poset -- partially ordered set).
\end{notebox}

\begin{exbox}
\textbf{דוגמאות ליחסי סדר חלקי:}

\begin{itemize}
    \item $\leq$ על $\R$ (או על $\N, \Z, \Q$)
    \item $\subseteq$ על $\mathcal{P}(A)$ (יחס ההכלה על קבוצת החזקה)
    \item יחס ההתחלקות $|$ על $\N^+$: $a \mid b \iff \exists k \in \N. \; b = ak$
    \item יחס ``$f$ גדולה נקודתית מ-$g$'' על קבוצת הפונקציות
\end{itemize}
\end{exbox}

%====================================
\subsection{יחס סדר מלא}
%====================================

\begin{defbox}
\textbf{הגדרה: יחס סדר מלא}

יחס סדר חלקי $R$ על $A$ נקרא \textbf{יחס סדר מלא} (total/linear order) אם בנוסף הוא \textbf{קשר} (connected):

\[\forall x, y \in A. \; xRy \lor yRx\]

כלומר, כל שני איברים ניתנים להשוואה.
\end{defbox}

\begin{exbox}
\textbf{דוגמאות:}

\begin{itemize}
    \item $\leq$ על $\R$ -- יחס סדר \textbf{מלא}
    \item $\subseteq$ על $\mathcal{P}(A)$ -- יחס סדר \textbf{חלקי} (לא מלא):
    \begin{itemize}
        \item $\{1\} \not\subseteq \{2\}$ וגם $\{2\} \not\subseteq \{1\}$
    \end{itemize}
    \item $|$ (מחלק) על $\N^+$ -- \textbf{חלקי}: $2 \nmid 3$ וגם $3 \nmid 2$
\end{itemize}
\end{exbox}

%====================================
\subsection{יחס סדר חזק}
%====================================

\begin{defbox}
\textbf{הגדרה: יחס סדר חלקי חזק}

\textbf{יחס סדר חלקי חזק} (strict partial order) הוא יחס שהוא:

\begin{enumerate}
    \item \textbf{אי-רפלקסיבי:} $\forall x. \; \neg(xRx)$
    \item \textbf{טרנזיטיבי:} $(xRy \land yRz) \Rightarrow xRz$
\end{enumerate}

(אנטי-סימטריות חזקה נובעת מהטרנזיטיביות והאי-רפלקסיביות)
\end{defbox}

\begin{exbox}
\textbf{דוגמאות:}
\begin{itemize}
    \item $<$ על $\R$
    \item $\subsetneq$ (הכלה ממש)
\end{itemize}
\end{exbox}

\begin{thmbox}
\textbf{קשר בין סדר חלקי לסדר חזק}

אם $\leq$ יחס סדר חלקי על $A$, אז:
\[< \; = \; \{(x, y) \in A^2 \mid x \leq y \land x \neq y\}\]
הוא יחס סדר חלקי חזק.

ולהפך: אם $<$ יחס סדר חלקי חזק, אז:
\[\leq \; = \; < \cup I_A\]
הוא יחס סדר חלקי (כאשר $I_A$ הוא יחס הזהות).
\end{thmbox}

%====================================
\subsection{איברים מיוחדים}
%====================================

\begin{defbox}
\textbf{הגדרות: איברים מיוחדים בקבוצה סדורה}

יהי $(A, \leq)$ קבוצה סדורה חלקית. עבור $a \in A$:

\textbf{מינימלי ומקסימלי:}
\begin{itemize}
    \item \textbf{מינימלי:} אין איבר קטן ממנו ממש
    \[\neg \exists x \in A. \; x < a\]
    \item \textbf{מקסימלי:} אין איבר גדול ממנו ממש
    \[\neg \exists x \in A. \; a < x\]
\end{itemize}

\textbf{מינימום ומקסימום:}
\begin{itemize}
    \item \textbf{מינימום} (הקטן ביותר): קטן או שווה לכל איבר
    \[\forall x \in A. \; a \leq x\]
    \item \textbf{מקסימום} (הגדול ביותר): גדול או שווה לכל איבר
    \[\forall x \in A. \; x \leq a\]
\end{itemize}
\end{defbox}

\begin{notebox}
\textbf{הבדל חשוב: מינימלי vs מינימום}

\begin{itemize}
    \item \textbf{מינימום:} אם קיים, הוא \textbf{יחיד}
    \item \textbf{מינימלי:} יכולים להיות \textbf{כמה}, או \textbf{אף אחד}
    \item בסדר \textbf{מלא}: מינימלי = מינימום
    \item בסדר \textbf{חלקי}: יכול להיות מינימלי בלי מינימום
\end{itemize}
\end{notebox}

\begin{exbox}
\textbf{דוגמה: הכלה על קבוצת חזקה}

$(\mathcal{P}(\{1,2,3\}) \setminus \{\emptyset\}, \subseteq)$

\begin{itemize}
    \item \textbf{מינימליים:} $\{1\}, \{2\}, \{3\}$ (אין קטנים מהם)
    \item \textbf{מינימום:} אין! (אף איבר לא מוכל בכולם)
    \item \textbf{מקסימום:} $\{1,2,3\}$
\end{itemize}
\end{exbox}

%====================================
\subsection{חסמים}
%====================================

\begin{defbox}
\textbf{הגדרות: חסמים}

יהי $(A, \leq)$ קבוצה סדורה ו-$B \subseteq A$:

\textbf{חסמים עליונים ותחתונים:}
\begin{itemize}
    \item \textbf{חסם עליון} של $B$: איבר $a \in A$ כך ש-$\forall b \in B. \; b \leq a$
    \item \textbf{חסם תחתון} של $B$: איבר $a \in A$ כך ש-$\forall b \in B. \; a \leq b$
\end{itemize}

\textbf{סופרמום ואינפימום:}
\begin{itemize}
    \item \textbf{סופרמום} (חסם עליון הדוק): החסם העליון \textbf{הקטן ביותר} -- $\sup(B)$
    \item \textbf{אינפימום} (חסם תחתון הדוק): החסם התחתון \textbf{הגדול ביותר} -- $\inf(B)$
\end{itemize}
\end{defbox}

\begin{exbox}
\textbf{דוגמה:} $B = (0, 1) \subseteq \R$ עם הסדר הרגיל:

\begin{itemize}
    \item \textbf{חסמים עליונים:} כל $x \geq 1$
    \item \textbf{חסמים תחתונים:} כל $x \leq 0$
    \item \textbf{סופרמום:} $\sup(B) = 1$ (לא שייך ל-$B$!)
    \item \textbf{אינפימום:} $\inf(B) = 0$ (לא שייך ל-$B$!)
    \item \textbf{מקסימום:} אין
    \item \textbf{מינימום:} אין
\end{itemize}
\end{exbox}

%====================================
\subsection{קבוצות סדורות טובות}
%====================================

\begin{defbox}
\textbf{הגדרה: קבוצה סדורה טובה}

קבוצה סדורה $(A, \leq)$ נקראת \textbf{סדורה טובה} (well-ordered) אם:

\begin{enumerate}
    \item $\leq$ הוא יחס סדר \textbf{מלא}
    \item לכל תת-קבוצה לא ריקה $B \subseteq A$ יש \textbf{מינימום}
\end{enumerate}
\end{defbox}

\begin{thmbox}
\textbf{משפט: סדר טוב על הטבעיים}

$(\N, \leq)$ היא קבוצה סדורה טובה.

זהו \textbf{עקרון הסדר הטוב} (Well-Ordering Principle).
\end{thmbox}

\begin{notebox}
\textbf{שקילות לאינדוקציה:}

עקרון הסדר הטוב שקול לעקרון האינדוקציה על $\N$.
\end{notebox}

%====================================
\subsection{דיאגרמת האסה}
%====================================

\begin{defbox}
\textbf{הגדרה: דיאגרמת האסה}

\textbf{דיאגרמת האסה} (Hasse diagram) היא ייצוג גרפי של יחס סדר חלקי:

\begin{itemize}
    \item כל איבר מיוצג כנקודה
    \item אם $x < y$ ואין $z$ כך ש-$x < z < y$, מציירים קו מ-$x$ ל-$y$
    \item איברים גדולים יותר מצוירים למעלה
\end{itemize}
\end{defbox}

\begin{exbox}
\textbf{דוגמה: מחלקי 12}

דיאגרמת האסה של $(D_{12}, |)$ כאשר $D_{12} = \{1, 2, 3, 4, 6, 12\}$:

\begin{verbatim}
       12
      /  \
     4    6
     |   / \
     2  /   3
      \/
       1
\end{verbatim}
\end{exbox}

%====================================
\subsection{טבלת סיכום -- סוגי יחסי סדר}
%====================================

\begin{center}
\begin{tabular}{|l|l|l|}
\hline
\rowcolor{tableheader}\color{white}\textbf{סוג יחס} & \color{white}\textbf{תכונות} & \color{white}\textbf{דוגמאות} \\
\hline
\rowcolor{tablerow1} \textbf{סדר חלקי} & רפלקסיבי, אנטי-סימטרי, טרנזיטיבי & $\leq$, $\subseteq$, $|$ \\
\hline
\rowcolor{tablerow2} \textbf{סדר מלא} & סדר חלקי + קשר (טוטאלי) & $\leq$ על $\R$ \\
\hline
\rowcolor{tablerow1} \textbf{סדר חזק} & אי-רפלקסיבי, טרנזיטיבי & $<$, $\subsetneq$ \\
\hline
\rowcolor{tablerow2} \textbf{סדר טוב} & סדר מלא + לכל ת``ק יש מינימום & $\leq$ על $\N$ \\
\hline
\end{tabular}
\end{center}

%====================================
\subsection{טבלת סיכום -- איברים מיוחדים}
%====================================

\begin{center}
\begin{tabular}{|l|l|l|l|}
\hline
\rowcolor{tableheader}\color{white}\textbf{מושג} & \color{white}\textbf{הגדרה} & \color{white}\textbf{יחידות?} & \color{white}\textbf{בסדר מלא} \\
\hline
\rowcolor{tablerow1} \textbf{מינימום} & $\forall x. \; a \leq x$ & כן (אם קיים) & = מינימלי \\
\hline
\rowcolor{tablerow2} \textbf{מינימלי} & $\neg \exists x. \; x < a$ & לא בהכרח & = מינימום \\
\hline
\rowcolor{tablerow1} \textbf{אינפימום} & חסם תחתון גדול ביותר & כן (אם קיים) & = מינימום אם שייך \\
\hline
\rowcolor{tablerow2} \textbf{חסם תחתון} & $\forall b \in B. \; a \leq b$ & לא בהכרח & -- \\
\hline
\end{tabular}
\end{center}

%====================================
\subsection{שגיאות נפוצות}
%====================================

\begin{notebox}
\textbf{שגיאה 1: בלבול בין מינימלי למינימום}

\begin{itemize}
    \item \textbf{מינימום:} קטן מ\textbf{כולם} -- יחיד אם קיים
    \item \textbf{מינימלי:} אין קטן \textbf{ממנו} -- יכולים להיות כמה
\end{itemize}
\end{notebox}

\begin{notebox}
\textbf{שגיאה 2: הנחה שסדר חלקי הוא מלא}

רוב יחסי הסדר הם חלקיים! לא כל שני איברים ניתנים להשוואה.
\end{notebox}

\begin{notebox}
\textbf{שגיאה 3: בלבול בין סופרמום למקסימום}

\begin{itemize}
    \item \textbf{מקסימום:} שייך לקבוצה
    \item \textbf{סופרמום:} לא בהכרח שייך לקבוצה
\end{itemize}
\end{notebox}

%====================================
\subsection{תרגילים לתרגול}
%====================================

\begin{exbox}
\textbf{תרגיל 1:}
ציירו את דיאגרמת האסה של $(\mathcal{P}(\{a, b, c\}), \subseteq)$.
מצאו את כל האיברים המינימליים והמקסימליים.
\end{exbox}

\begin{exbox}
\textbf{תרגיל 2:}
תהי $A = \{2, 3, 4, 6, 8, 12, 24\}$ עם יחס ההתחלקות.
מצאו את הסופרמום והאינפימום של $B = \{4, 6\}$ אם קיימים.
\end{exbox}

\begin{exbox}
\textbf{תרגיל 3:}
הוכיחו שאם $(A, \leq)$ קבוצה סדורה טובה ו-$B \subseteq A$ לא ריקה, אז לכל תת-קבוצה לא ריקה של $B$ יש מינימום.
\end{exbox}

\clearpage
% יחידה 8 - יחסי שקילות, מחלקות
%====================================
\section{יחידה 8: יחסי שקילות, מחלקות}

\subsection{יחסי שקילות}
%====================================

\begin{defbox}
\textbf{הגדרה: יחס שקילות}

יחס $E$ על קבוצה $A$ נקרא \textbf{יחס שקילות} (equivalence relation) אם הוא מקיים שלושה תנאים:

\begin{enumerate}
    \item \textbf{רפלקסיביות:} $\forall x \in A. \; xEx$
    \item \textbf{סימטריות:} $\forall x, y \in A. \; xEy \Rightarrow yEx$
    \item \textbf{טרנזיטיביות:} $\forall x, y, z \in A. \; (xEy \land yEz) \Rightarrow xEz$
\end{enumerate}
\end{defbox}

\begin{exbox}
\textbf{דוגמאות ליחסי שקילות:}

\begin{enumerate}
    \item \textbf{שוויון} $=$ על כל קבוצה

    \item \textbf{דמיון משולשים} על קבוצת המשולשים

    \item \textbf{חפיפת משולשים} על קבוצת המשולשים

    \item \textbf{שקילות מודולו $n$:}
    \[a \equiv b \pmod{n} \iff n \mid (a - b)\]

    לדוגמה: $5 \equiv 11 \pmod{2}$ כי $2 \mid (11 - 5) = 6$

    \item \textbf{יחס ``באותו צבע''} על קבוצת כדורים צבעוניים
\end{enumerate}
\end{exbox}

\subsection{מחלקות שקילות}
%====================================

\begin{defbox}
\textbf{הגדרה: מחלקת שקילות}

יהי $E$ יחס שקילות על $A$. עבור $x \in A$, \textbf{מחלקת השקילות} של $x$ לפי $E$ היא:

\[[x]_E = \{y \in A \mid xEy\}\]

כלומר, קבוצת כל האיברים השקולים ל-$x$.
\end{defbox}

\begin{notebox}
\textbf{סימונים נפוצים למחלקת שקילות:}
\begin{itemize}
    \item $[x]_E$ או $[x]$ (כשהיחס ברור מההקשר)
    \item $\overline{x}$
    \item $x/E$
\end{itemize}
\end{notebox}

\begin{thmbox}
\textbf{תכונות מחלקות שקילות}

יהי $E$ יחס שקילות על $A$. לכל $x, y \in A$:

\begin{enumerate}
    \item \textbf{כל איבר שייך למחלקה שלו:} $x \in [x]_E$ (מרפלקסיביות)

    \item \textbf{שקילות היא שייכות למחלקה:} $y \in [x]_E \iff xEy$

    \item \textbf{סימטריות:} $y \in [x]_E \iff x \in [y]_E$

    \item \textbf{מחלקות שוות או זרות:}
    \[[x]_E \cap [y]_E \neq \emptyset \iff [x]_E = [y]_E \iff xEy\]
\end{enumerate}
\end{thmbox}

\begin{exbox}
\textbf{דוגמה: מחלקות מודולו 3}

יחס השקילות מודולו 3 על $\Z$:

\begin{itemize}
    \item $[0]_{\equiv_3} = \{\ldots, -6, -3, 0, 3, 6, 9, \ldots\} = 3\Z$
    \item $[1]_{\equiv_3} = \{\ldots, -5, -2, 1, 4, 7, 10, \ldots\} = 3\Z + 1$
    \item $[2]_{\equiv_3} = \{\ldots, -4, -1, 2, 5, 8, 11, \ldots\} = 3\Z + 2$
\end{itemize}

שלוש מחלקות זרות שמכסות את כל $\Z$.
\end{exbox}

\subsection{קבוצת מנה}
%====================================

\begin{defbox}
\textbf{הגדרה: קבוצת מנה}

יהי $E$ יחס שקילות על $A$. \textbf{קבוצת המנה} (quotient set) של $A$ לפי $E$ היא:

\[A/E = \{[x]_E \mid x \in A\}\]

כלומר, קבוצת כל מחלקות השקילות.
\end{defbox}

\begin{exbox}
\textbf{דוגמאות לקבוצות מנה:}

\begin{enumerate}
    \item $\Z/\equiv_2 = \{[0], [1]\}$ -- שתי מחלקות (זוגיים ואי-זוגיים)

    \item $\Z/\equiv_3 = \{[0], [1], [2]\}$ -- שלוש מחלקות

    \item $\R/\equiv$ כאשר $x \equiv y \iff |x| = |y|$:

    קבוצת המנה היא למעשה $\R_{\geq 0}$
\end{enumerate}
\end{exbox}

\subsection{הפונקציה הקנונית}
%====================================

\begin{defbox}
\textbf{הגדרה: הפונקציה הקנונית}

יהי $E$ יחס שקילות על $A$. \textbf{הפונקציה הקנונית} (או פונקציית המנה) היא:
\[\pi: A \to A/E, \quad \pi(x) = [x]_E\]
\end{defbox}

\begin{thmbox}
\textbf{תכונות הפונקציה הקנונית:}

\begin{enumerate}
    \item $\pi$ היא \textbf{על} (surjective)
    \item $\pi(x) = \pi(y) \iff xEy$
    \item $\text{Im}(\pi) = A/E$
\end{enumerate}
\end{thmbox}

\subsection{טבלת סיכום}
%====================================

\begin{center}
\begin{tabular}{|l|l|l|}
\hline
\rowcolor{tableheader}\color{white}\textbf{מושג} & \color{white}\textbf{הגדרה} & \color{white}\textbf{דוגמה} \\
\hline
\rowcolor{tablerow1} \textbf{יחס שקילות} & רפלקסיבי + סימטרי + טרנזיטיבי & $\equiv_n$, $=$ \\
\hline
\rowcolor{tablerow2} \textbf{מחלקת שקילות} & $[x]_E = \{y \mid xEy\}$ & $[0]_{\equiv_3} = 3\Z$ \\
\hline
\rowcolor{tablerow1} \textbf{קבוצת מנה} & $A/E = \{[x]_E \mid x \in A\}$ & $\Z/\equiv_3$ \\
\hline
\end{tabular}
\end{center}

\subsection{שגיאות נפוצות}
%====================================

\begin{notebox}
\textbf{שגיאה 1: בלבול בין מחלקה לאיבר}

\begin{itemize}
    \item $[a]$ היא \textbf{קבוצה} (מחלקת השקילות)
    \item $a$ הוא \textbf{איבר} (נציג של המחלקה)
\end{itemize}
\end{notebox}

\begin{notebox}
\textbf{שגיאה 2: שכחת לבדוק את שלושת התנאים}

יחס שקילות חייב לקיים את \textbf{שלושת} התנאים: רפלקסיביות, סימטריה, טרנזיטיביות.
\end{notebox}

\clearpage
% הערה: קבצי LaTeX מכסים את כל 13 היחידות:
% unit1-5: אינדוקציה, לוגיקה, קבוצות, יחסים, פונקציות
% unit6: שקילות וחלוקות (יחידות 8-9 באתר)
% unit7: יחסי סדר (חלק מיחידה 9 באתר)
% unit8: עוצמות, לכסון, קש"ב, חשבון עוצמות (יחידות 10-13 באתר)

\end{document}
